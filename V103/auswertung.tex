% This work is licensed under the Creative Commons
% Attribution-NonCommercial 3.0 Unported License. To view a copy of this
% license, visit http://creativecommons.org/licenses/by-nc/3.0/.

\section{Auswertung}

Aus den gewonnenen Daten, die nachfolgend vorgestellt werden, kann man
den Elastizitätsmodul der untersuchten Metalle und Legierungen
errechnen. Um eine lineare Regression anwenden zu können, stellt man die
Zusammenhänge \eqref{eq:durchbiegung-einseitig} und
\eqref{eq:durchbiegung-beidseitig} so dar:
%
\begin{align}
  \label{eq:lineare-durchbiegung-einseitig}
  D(x) &= f(Lx^2-\frac{x^3}{3}) & f(u) &:= \frac{F}{2EI}u
\end{align}
Dies ist für die einseitige Aufhängung des Stabes. Für die beidseitige
Aufhängung ergibt sich:
\begin{align}
  \label{eq:lineare-durchbiegung-beidseitig1}
  D(x) &= g(3L^2x-4x^3) & g(u) &:= \frac{F}{48EI}u\\
  \label{eq:lineare-durchbiegung-beidseitig2}
  D(x) &= g(4x^3-12Lx^2 + 9L^2x - L^3)
\end{align}
%
Hier können nun die Größen $A_g = F/(48EI)$ und $A_f = F/(2EI)$ bestimmt
werden. Umstellen von der beiden Größen nach $E$ liefert:
\begin{align}
  E &= \frac{F}{2A_fI} & E =& \frac{F}{48A_gI}
\end{align}
Da diese fehlerbehaftet sind, wird eine \name{Gauß}-Fehlerfortpflanzung
durchgeführt:
%
\begin{equation}
  \label{eq:gaussfehler-f}
  \Delta E = \sqrt{\left(\frac{\partial E}{\partial A_f}\cdot\Delta A_f\right)^2}
  = \frac{F\cdot\Delta A_f}{2A_f^2I}
\end{equation}
%
\begin{equation}
  \label{eq:gaussfehler-g}
  \Delta E = \sqrt{\left(\frac{\partial E}{\partial A_g}\cdot\Delta
      A_g\right)^2}
  = \frac{3}{3}
\end{equation}

\subsection{Bestimmung des Elastizitätsmoduls von Stahl}


\subsection{Bestimmung des Elastizitätsmoduls von Aluminium}


\subsection{Bestimmung des Elastizitätsmoduls von Messing}



%%% Local Variables: 
%%% mode: latex
%%% TeX-master: "protokoll"
%%% End: 

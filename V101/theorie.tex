% This work is licensed under the Creative Commons
% Attribution-NonCommercial 3.0 Unported License.  To view a copy of
% this license, visit http://creativecommons.org/licenses/by-nc/3.0/.

\section{Theorie}

Zur Beschreibung der Dynamik von Rotationsbewegungen ist es zweckmäßig
die Größen Drehmoment~$\vec{M}$, Trägheitsmoment~$I$ und
Winkelbeschleunigung~$\ddot{\phi}$ einzuführen, welche den Größen
Kraft~$F$, träge Masse~$m$ und Beschleunigung~$\ddot{x}$ aus der
Dynamik der Translationsbewegung entsprechen.  Das Trägheitsmoment eines
Massenpunktes~$m$, der sich im Abstand~$r$ und eine Rotationsachse
dreht, ist definiert als
\begin{equation}
  I = mr^2.
\end{equation}
Auf Grundlage dieser Definition kann man nun Trägheitsmomente für
ausgedehnte Körper definieren.  Ist der Körper aus $n$ Massenpunkten der
Masse~$m_i$ zusammengesetzt, die sich im Abstand~$r_i$ um eine Achse
drehen, dann gilt für das Gesamtträgheitsmoment
\begin{equation}
  I = \sum_{i = 1}^n r_i^2 m_i.
\end{equation}
Ein ausgedehnter Körper kann auch durch eine auf einem Gebiet~$G$
gegebene Massendichte~$\rho$ beschrieben sein, die jedem Punkt dieses
Gebiets eine Dichte zuordnet.  In diesem Fall ist das
Gesamtträgheitsmoment als Integral
\begin{equation}
  I = \int\nolimits_G r^2 \rho \d V
\end{equation}
gegeben, wobei $r$ den Abstand zur Drehachse kennzeichnet.

Für verschiedene einfache geometrische Körper kann das
Gesamtträgheitsmoment leicht berechnet werden.  Komplizierte Formen
lassen sich durch Aufteilen in einfache Formen berechnen, die dann
addiert werden müssen.  Gegebenfalls bieten sich auch Vereinfachungen an
(z.\,B. den Arm eines Menschen als Zylinder anzunähern).  Oft ist ein
Trägheitsmoment für eine Achse durch den Schwerpunkt bekannt, es soll
aber das Trägheitsmoment bezüglich einer anderen, zu dieser parallelen
Achse berechnet werden. Dazu ist der Satz von
\name{Steiner}\footnote{Jakob \name{Steiner} (1796--1863), schweizer
  Mathematiker (entnommen aus \textcite{wikipedia:jakob-steiner})}
geeignet.  Er besagt:
\begin{equation}
  \label{eq:steiner}
  I = I_\text{S} + m a^2,
\end{equation}
wobei $I_\text{S}$ das bekannte Trägheitsmoment bezogen auf eine Achse
durch den Schwerpunkt, $m$ die Gesamtmasse des Körpers und $a$ den
Abstand der Drehachse von der Schwerpunktsachse bezeichnet.

In diesem Versuch befinden sich die zu untersuchenden Körper in einem
oszillierendem System, so daß nach einer Auslenkung um den Winkel~$\phi$
sofort ein rücktreibendes Drehmoment, das durch die Kraft~$\vec{F}$ mit
$\vec{M} = \vec{r} \times \vec{F}$ gegeben ist, auf den Körper wirkt.
Im Fall kleiner Auslenkungen~$\phi$ kommt es zu Schwingungen mit der
Periodendauer
\begin{equation}
  \label{eq:periode}
  T = 2 \pi \sqrt{\frac{I}{D}},
\end{equation}
wobei $I$ das Gesamtträgheitsmoment und $D$ die sogenannte
Winkelrichtgröße bezeichnet, die mit dem Drehmoment $M$ über die
Beziehung
\begin{equation}
  \label{eq:drehmoment-winkelricht}
  M = D \phi
\end{equation}
zusammenhängt.


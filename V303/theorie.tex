% This work is licensed under the Creative Commons
% Attribution-NonCommercial 3.0 Unported License. To view a copy of this
% license, visit http://creativecommons.org/licenses/by-nc/3.0/.

\section{Theorie}
Mit dem Lock-In-Verstärker ist es möglich, stark verrauschte Signale zu
messen.  Dazu wird das verrauschte Signal in mehreren Schritten
bearbeitet. Als erstes wird das verrauschte Signal im Preamplifier
verstärkt, d.h. es wird der Spannungswert zu jedem Zeitpunkt mit einem
Faktor multipliziert. Anschließend wird das Signal im Bandpassfilter
durch einen Tief- und Hochpass von Frequenzen befreit, die stark von
einer einstellbaren Frequenz $\omega_0$ abweichen.  Im nächsten
Bearbeitungsschritt wird das Signal in einem Mischer mit einem
Referenzsignal einstellbarer Phase multipliziert. Die Frequenz des
Referenzsignals entspricht dabei der des zu messenden Signals.  In
diesem Versuch wird eine Sinusspannung als Referenzsignal
verwendet. D.h. das Referenzsignal hat die in Formel \eqref{eq:refsig}
wiedergegebene Form.
%
\begin{equation}
  \label{eq:refsig}
  U_\text{ref} = \widehat{U}_\text{ref} \cdot \sin(\omega \cdot t)
\end{equation}
%
Gibt es keine Phasendifferenz zwischen zu messendem Signal und
Referenzsignal, dann ist die Formel \eqref{eq:eingangssig} für das zu
messende Signal analog.
%
\begin{equation}
  \label{eq:eingangssig}
  U_\text{sig} = \widehat{U}_\text{sig} \cdot \sin(\omega \cdot t)
\end{equation}
%
Also ergibt die Multiplikation der beiden Signale folgendes:
%
\begin{equation*}
  \begin{split}
    U_\text{sig} \cdot U_\text{ref}
    &= \widehat{U}_\text{sig}\cdot \widehat{U}_\text{ref}\cdot
    \sin^2(\omega \cdot t) \\
    &= \widehat{U}_\text{sig} \cdot \widehat{U}_\text{ref}
    \cdot (1 - \cos^2(\omega \cdot t)) \\
    &= \widehat{U}_\text{sig} \cdot \widehat{U}_\text{ref} \cdot (1 -
    \frac{1}{2} \cdot (1 +
    \cos(2\cdot \omega \cdot t)))\\
    &= \frac{1}{2} \widehat{U}_\text{sig} \cdot \widehat{U}_\text{ref}
    \cdot(1 - \cos(2\cdot \omega\cdot t))
  \end{split}
\end{equation*}
%
Der darauffolgende Tiefpass lässt nur niedrige Frequenzen durch, sodass
sich die Formel \eqref{eq:gleichspannung} für die Form des
Ausgangssignals ergibt, nachdem das Signal durch den Mischer und den
Tiefpass bearbeitet wurde.
%
\begin{equation}
  \label{eq:gleichspannung}
  U_\text{out} = \frac{k}{2} \cdot \widehat{U}_\text{sig} \widehat{U}_\text{ref}
\end{equation}
%
Das endgültige Signal ist also eine Gleichspannung, die proportional zum
Produkt der Amplituden der Referenzspannung und des Signals ist.

Gibt es einen Phasenunterschied zwischen zu messendem Signal und
Referenzsignal, so muss dieser berücksichtigt werden. Bei vorhandenem
Phasenunterschied $\phi$ hat das Ausgangssignal die in Formel
\eqref{eq:mischer_phase} wiedergegebene Form.
%
\begin{equation}
  \label{eq:mischer_phase}
   U_\text{out} = \frac{1}{2} \widehat{U}_\text{sig}
   \widehat{U}_\text{ref} \cdot cos(\phi)
\end{equation}

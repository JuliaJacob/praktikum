% This work is licensed under the Creative Commons
% Attribution-NonCommercial 3.0 Unported License. To view a copy of this
% license, visit http://creativecommons.org/licenses/by-nc/3.0/.

\section{Theorie}

Hat ein Metall an zwei verschiedenen Punkten eine unterschiedliche
Temperatur, so wird diese Temperaturdifferenz durch einen Wärmetransport
aufgelöst. In Metallen geschieht dieser Transport durch die sogenannte
Wärmeleitung, die über Phononen (Quantisierte Schwingungen in der
Gitterstruktur des Metalls\footcite[vgl. hierzu][]{uni-kiel:phononen})
und die frei beweglichen Elektronen des Elektronengases passiert.

Es wird nun ein Stab der Länge $L$ mit Querschnitt $A$, Dichte $\rho$
sowie spezifischer Wärme $c$ untersucht. Der Wärmestrom $\d Q/\d t$ hat
dabei folgende Gestalt:
%
\begin{equation}
  \label{eq:waermestrom}
  \td{Q}{t} = -\kappa A\pd{T}{x}
\end{equation}
%
Hier taucht neben der Querschnittsfläche $A$ noch die materialabhängige
Wärmeleitfähigkeit $\kappa$ als Proportionalitätskonstante auf. Wird der
Wärmestrom auf die Querschnittsfläche $A$ normiert, so ergibt sich die
Wärmestromdichte $j_w$ durch
%
\begin{equation}
  \label{eq:waermestromdichte}
  j_w = -\kappa\pd{T}{x}
\end{equation}
%
Aus dieser Größe kann man mit der Kontinuitätsgleichung die
eindimensionale Wärmeleitungsgleichung ableiten.
%
\begin{equation}
  \label{eq:wellengl.}
  \pd{T}{t} = \sigma_T\pdd{T}{x}
\end{equation}
%
Hier bezeichnet $\sigma_T := \kappa/(\rho c)$ die
Temperaturleitfähigkeit des Materials. Je nach Rand- und
Anfangsbedingungen, d.\,h. auch je nach Stabgeometrie, sieht die Lösung
der Gl.~\eqref{eq:wellengl.} anders aus.

Nimmt man einen sehr langen Stab und erwärmt und kühlt ihn abwechselnd
mit der Periode $\tilde{T} = 2\pi/\omega$ so ergibt sich als Lösung die
Funktion $T$ mit
%
\begin{equation}
  \label{eq:temp.fkt.}
  T(x, t) = T_\text{max}\cdot e^{-kx}\cos(\omega t - kx), \quad k
  = \sqrt{\frac{\omega\rho c}{2\kappa}}\text{.}
\end{equation}
%
Diese Funktion beschreibt die Ausbreitung der Temperaturwelle räumlich
und zeitlich. Man kann daraus auch die Phasengeschwindigkeit der Welle
bestimmen.
%
\begin{equation}
  \label{eq:phasengeschw.}
  v_\text{Ph} = \frac{\omega}{k} = \sqrt{\frac{2\kappa\omega}{\rho c}}
\end{equation}

Zur Bestimmung der Wärmeleitfähigkeit $\kappa$ kann man die Dämpfung
heranziehen. In Gl.~\eqref{eq:temp.fkt.} kann man den Faktor mit der
Exponentialfunktion als Dämpfungsterm identifizieren. Betrachtet man nun
die Amplituden $A_1, A_2$ an zwei Meßstellen $x_1, x_2$, kann man daraus
die Wärmeleitfähigkeit $\kappa$ errechnen. Wenn man noch $\omega =
2\pi/\tilde{T}$ und $\varphi = 2\pi\Delta t/\tilde{T}$ dazu nimmt, sowie
$\Delta x := x_2-x_1$ setzt, ergibt sich:
%
\begin{equation}
  \label{eq:kappa-aus-daempfung}
  \kappa = \frac{\rho c(\Delta x)^2}{2\Delta t\log(A_1/A_2)}
\end{equation}
%
Dort bezeichnet $\Delta t$ den Gangunterschied der Temperaturwelle
zwischen den beiden Meßstellen $x_1, x_2$.

% Füge Praktikumsanleitung zu den Quellen hinzu
\nocite{v204}

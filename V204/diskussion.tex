% This work is licensed under the Creative Commons
% Attribution-NonCommercial 3.0 Unported License. To view a copy of this
% license, visit http://creativecommons.org/licenses/by-nc/3.0/.

\section{Diskussion}

In Tabelle \ref{tab:endergebnis} werden die ermittelten Werte für die
Wärmekapazität mit den gefundenen Literaturwerten verglichen.  Die
gemessenen Werte sind im Vergleich mit den Literaturwerten nicht
gut. Einzig für Messing ergibt sich eine geringe Abweichung von
\SI{1.7}{\percent}. Beim Aluminium ergeben sich \SI{32.6}{\percent} und
beim Edelstahl \SI{25}{\percent}.

Die großen Abweichungen beim Aluminium können nicht erklärt werden. Es
kann nur vermutet werden, daß das Aluminium verunreinigt ist. Beim
Edelstahl, da es sich um eine Legierung handelt, kann nicht genau
entschieden werden, was der Literaturwert für das uns vorliegende
Material war. Die Abweichung für das Edelstahl könnte also daher kommen.

Aluminium besitzt eine höhere Wärmeleitfähigkeit als Messing. Dies
konnte in den Messungen bestätigt werden.

\begin{table}
  \centering
  \begin{tabular}{|c|S|S|S|}
    \hline
    &{Messing}& {Aluminium} & {Edelstahl} \\
    \hline
    Ermitteltes $\kappa/(\si{\watt\per\metre\per\kelvin})$ &118 & 313 & 15\\
    Literaturwert $\kappa/(\si{\watt\per\metre\per\kelvin})$ &120 & 236 & 20\\
    Abweichung in \si{\percent} & 1.7 & 32.6 & 25\\
    \hline
  \end{tabular}
  \caption{Vergleich zwischen errechneten Werten und Literaturwerten für
    die genannten Metalle. Die Werte für Aluminium und Messing stammen
    von \textcite{wikipedia:waermeleitfaehigkeit}. Der Wert für Edelstahl
    ist von \textcite{schweizer-fn:waermeleitfaehigkeiten} übernommen.}
  \label{tab:endergebnis}
\end{table}
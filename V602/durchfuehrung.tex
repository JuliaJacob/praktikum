% This work is licensed under the Creative Commons
% Attribution-NonCommercial 3.0 Unported License. To view a copy of this
% license, visit http://creativecommons.org/licenses/by-nc/3.0/.

\section{Durchführung}
%
In diesem Versuch wird zur Erzeugung von Röntgenstrahlung eine
Cu-Röntgenröhre verwendet, wie sie im Theorieteil beschrieben wurde. Die
verwendete Beschleunigungsspannung beträgt \SI{35}{\kilo\volt}. Der
Emissionsstrom beträgt \SI{1}{\milli\ampere}.  Um die Röntgenstrahlung
untersuchen zu können, wird die sogenannte Drehkristallmethode
verwendet. In diesem Versuch wird ein LiF-Kristall mit einer
Gitterkonstante $d$ von \SI{201.4}{\pico\metre} in einem Winkel $\theta$
zur einfallenden Röntgenstrahlung gestellt. Dadurch ergibt sich nur bei
Erfüllung der in Formel \eqref{eq:bragg_formel} wiedergegebenen
Bragg-Bedingung eine konstruktive Interferenz der reflektierten
Röntgenstrahlung. Also wird im Winkel $\theta$ im wesentlichen bloß
Röntgenstrahlung einer Wellenlänge, und somit Photonen einer Energie,
reflektiert. In Abb. \ref{fig:bragg_bild} ist dieser Sachverhalt
verbildlicht.  Nun werden also zur Untersuchung der Röntgenstrahlung
mithilfe der Drehkristallmethode und eines Geiger-Müller-Zählrohres die
registrieren Impulse, welche durch ankommende Röntenstrahlung erzeugt
werden, in einem Zeitintervall $\Delta t$ gemessen.

Die in diesem Versuch verwendete Röntgenapparatur wird mithilfe eines
Computers über das Programm measure gesteuert. Das Programm führt eine
Messung des Spektrums in dem angegebenen Winkelbereich mit einstellbarer
Schrittweite und Messdauer pro Winkel durch. Dabei wird vom unteren
Winkelbereich bis zum oberen gemessen. Gleichzeitig erstellt das
Programm die in diesem Versuch ausgewerteten Messkurven. Mit steigender
Energie der Röntgenstrahlung gelangt auch immer mehr Intensität durch
den angebrachten Absorber. Ab einer Energie, die zum ionisieren des
Apsorbermaterials genügt, werden aufgrund dieses Ionisationsvorganges
wieder mehr Röntgenphotonen absorbiert, sodass sich eine Kante
ergibt. Die Energie der Röntgenphotonen nimmt mit steigendem
Kristallwinkel ab, weswegen die Winkel der Kanten in den aufgenommenen
Messkurven an der linken Seite des entstehenden Ausschlages abgelesen
werden.
%
\begin{equation}
2 d \sin{\theta} = n \lambda
\label{eq:bragg_formel}
\end{equation}
%
%
\begin{figure}
\centering
\includegraphics[width=0.3\textwidth]{bragg_bild.pdf}
\caption{Zur Veranschaulichung der Bragg-Bedingung. Quelle: \textcite{v602}}
\label{fig:bragg_bild}
\end{figure}
%
Als erstes wird die Bragg-Bedingung überprüft. Dies geschieht durch
Vermessung eines Winkelbereiches von \SI{26}{\degree} bis
\SI{30}{\degree} in \SI{0.1}{\degree} Schritten bei einem festgestelltem
Kristallwinkel von \SI{14}{\degree}, sodass sich bei Erfüllung der
Bragg-Bedingung, also bei \SI{28}{\degree} ein Maximum in der
Impulsanzahl im Geiger-Müller-Zählrohr ausbilden sollte.  Anschließend
wird das Röntgenspektrum der Cu-Röntgenröhre in einem Winkelbereich von
\SI{4}{\degree} bis \SI{26}{\degree} vermessen.  Als nächstes werden der
Reihe nach Absorber aus Germanium, Niob, Rubidium, Zink und Zirkonium
vor die Öffnung des Geiger-Müller-Zählrohres gesetzt und die
Röntgenspektren in verschiedenen Winkelbereichen vermessen, in denen das
auftreten von K-Kanten des Absorbermaterials vermutet wird.  Zum Schluß
wird noch das gleiche Verfahren mit Gold als Absorbermaterial
durchgeführt. Bei Gold wird das Auftauchen von zwei L-Kanten vermutet.
%
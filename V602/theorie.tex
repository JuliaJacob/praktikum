% This work is licensed under the Creative Commons
% Attribution-NonCommercial 3.0 Unported License. To view a copy of this
% license, visit http://creativecommons.org/licenses/by-nc/3.0/.

\section{Theorie}
%
\subsection{Emission von Röntgenstrahlen}
%
Die in diesem Versuch verwendeten Röntgenstrahlen werden mithilfe einer Kupfer-Röntgenröhre erzeugt. Dabei handelt es sich um eine evakuierte Röhre, in der sich eine beheizte Kathode und eine Kupferanode befinden. Durch den glühelektrischen Effekt werden Elektronen aus dem Kathodenmaterial gelöst. Durch anlegen einer Beschleunigungsspannung $U_\text{B}$ werden diese freien Elektronen zur Kupferanode hin beschleunigt. Durch das Abbremsen dieser Elektronen im Coulombfeld der Atome des Anodenmaterials wird pro Abbremsen ein Photon emittiert, welches eine Energie besitzt, die der Abnahme der kinetischen Energie des abgebremsten Elektrons entspricht. Durch Formel \eqref{eq:photonenenergie} kann man die Wellenlänge $\lambda$ der dazugehörigen elektromagnetischen Strahlung mithilfe des Planck'schen Wirkungsquantums h und der Lichtgeschindigkeit c bestimmen.
%
\begin{equation}
E = h \cdot \frac{c}{\lambda}
\label{eq:photonenenergie}
\end{equation}
%
Je nach Stärke der Abbremsung wird ein Photon mit passender Energie emittiert. Die durch Abbremsung entstehende Röntgenstrahlung ist also kontinuierlich. Dadurch ergibt sich im Frequenzspektrum der Röntgenstrahlung der sogenannte Bremsberg. Besitzen die beschleunigten Elektronen eine kinetische Energie, welche groß genug ist, sodass Atome des Anodenmatierals ionisiert werden, können Elektronen dieser ionisierten Atome von höheren Energienieveaus auf das Energienieveau vom herausgeschlagenen Elektron fallen. Dabei wird ein Photon emittiert, welches eine Energie besitzt, die exakt der Differenz der Energienieveaus entspricht. Da die Energieniveaus in einem Atom diskrete Werte besitzen, werden auf diese Weise nur Photonen mit diskreten Energien emittiert. Diese sind im Röntgenspektrum als sogenannte Kanten zu finden.


%
\subsection{Absorption von Röntgenstrahlen}
%
Bei Verwendung eines Absorbers vor einem Messgerät für Röntgenstrahlung wird die einfallende Röntgenstrahlung durch den Comptoneffekt und dem Photoeffekt geschwächt. Besitzt die einfallende Strahlung eine Energie, die ausreicht um die Atome des Absorbermaterials zu ionisieren, so zeigt sich dies als eine weitere Kante im entstehenden Röntgenspektrum, da das Absorbermaterial dann wiederum selbst Photonen aussendet. Dies geschieht in ähnlicher Weise wie bei der Kupferanode.

Bei Mehrelektronenatomen wird die Kernladung durch die Elektronen abgeschirmt, sodass die Elektronen in den äußeren Schalen eine geringere Bindungsenergie besitzen als in den inneren Schalen und somit leichter herausgeschlagen werden können. Durch Bestimmung der K-Kanten kann mithilfe von Formel \eqref{ eq:sigma_k} die Abschirmzahl $\sigma_\text{K}$ bestimmt werden. Dabei bezeichnet $\alpha$ die Feinstrukturkonstante, Z die Kernladungszahl $E_\text{K}$ die Energie der K-Kante und $R_\infty$ = \SI{13.6}{\electronvolt} die Rydberg-Energie.
%
\begin{equation}
\sigma_\text{K} = Z - \sqrt{\frac{E_\text{K}}{R_\infty} - \frac{\alpha^2 Z^4}{4}}
\label{ eq:sigma_k}
\end{equation}
% 
Des weiteren lässt sich die Abschirmkonstante $\sigma_\text{L}$ aus der Energiedifferenz $\Delta$E der $L_{II}$ und $L_{III}$ Kanten mithilfe von Formel \eqref{eq:sigma_l} bestimmen.
%
\begin{equation}
\sigma_\text{L} = Z - \sqrt{\frac{4}{\alpha} \cdot \sqrt{\frac{\Delta E}{R_\infty}} - \frac{5 \Delta E}{R_\infty}} \cdot \sqrt{ 1 + \frac{19 \alpha ^2 \Delta E}{32 R_\infty}}
\label{eq:sigma_l}
\end{equation}
%
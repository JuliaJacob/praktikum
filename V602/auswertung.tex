% This work is licensed under the Creative Commons
% Attribution-NonCommercial 3.0 Unported License. To view a copy of this
% license, visit http://creativecommons.org/licenses/by-nc/3.0/.

\section{Auswertung}

\subsection{Überprüfung der \name{Bragg}-Bedingung}

Aus Tabelle~\ref{tab:bragg-bed} kann entnehmen werden, daß das Maximum
der Zählrate unter dem Zählrohr-Winkel \SI{27.9}{\degree} erreicht
ist. Das liefert für den Glanzwinkel
%
\begin{equation}
  \theta = \SI{13.95}{\degree} .
\end{equation}
%
Der Winkel weicht also nur um \SI{0.05}{\degree} vom erwarteten Winkel
von \SI{14}{\degree} ab. Damit ist die \name{Bragg}-Bedingung bestätigt.


\begin{table}
  \centering
  \begin{tabular}{SS}
    \toprule
    {$\theta/\si{\degree}$} & {Zählrate} \\
    \midrule
%    % This work is licensed under the Creative Commons
% Attribution-NonCommercial 3.0 Unported License. To view a copy of this
% license, visit http://creativecommons.org/licenses/by-nc/3.0/.

% Z�hlrohrwinkel	Rate bei 35kV mit Gold
% 2*theta/�	R(35kV)/Imp/s

26.0  &  58.0   \\
26.1  &  62.0   \\
26.2  &  74.0   \\
26.3  &  83.0   \\
26.4  &  94.0   \\
26.5  &  102.0  \\
26.6  &  120.0  \\
26.7  &  125.0  \\
26.8  &  135.0  \\
26.9  &  139.0  \\
27.0  &  154.0  \\
27.1  &  163.0  \\
27.2  &  167.0  \\
27.3  &  173.0  \\
27.4  &  173.0  \\
27.5  &  183.0  \\
27.6  &  198.0  \\
27.7  &  197.0  \\
27.8  &  189.0  \\
27.9  &  201.0  \\
28.0  &  199.0  \\
28.1  &  199.0  \\
28.2  &  191.0  \\
28.3  &  180.0  \\
28.4  &  166.0  \\
28.5  &  154.0  \\
28.6  &  150.0  \\
28.7  &  145.0  \\
28.8  &  125.0  \\
28.9  &  108.0  \\
29.0  &  101.0  \\
29.1  &  87.0   \\
29.2  &  79.0   \\
29.3  &  64.0   \\
29.4  &  57.0   \\
29.5  &  49.0   \\
29.6  &  43.0   \\
29.7  &  38.0   \\
29.8  &  38.0   \\
29.9  &  38.0   \\
30.0  &  45.0   \\
    \bottomrule
  \end{tabular}
  \caption{Die Meßwerte zur Überprüfung der \name{Bragg}-Bedingung. Der
    angegebene Winkel ist der Zählrohrwinkel. Der Kristallwinkel war
    fest auf \SI{14}{\degree} eingestellt.}
  \label{tab:bragg-bed}
\end{table}

\subsection{Emissionspektrum der Kupferanode}



\subsection{Absorptionsspektren verschiedener Materialien}


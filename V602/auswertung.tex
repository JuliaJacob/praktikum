% This work is licensed under the Creative Commons
% Attribution-NonCommercial 3.0 Unported License. To view a copy of this
% license, visit http://creativecommons.org/licenses/by-nc/3.0/.

\section{Auswertung}

\subsection{Überprüfung der \name{Bragg}-Bedingung}

Aus Tabelle~\ref{tab:bragg-bed} kann entnommen werden, daß das Maximum
der Zählrate unter dem Zählrohr-Winkel \SI{27.9}{\degree} erreicht
ist. Das liefert für den Glanzwinkel
%
\begin{equation}
  \theta = \SI{13.95}{\degree} .
\end{equation}
%
Der Winkel weicht also nur um \SI{0.05}{\degree} vom erwarteten Winkel
von \SI{14}{\degree} ab. Damit ist die \name{Bragg}-Bedingung
bestätigt. Eine graphische Darstellung der Meßwerte aus
Tabelle~\ref{tab:bragg-bed} kann in Abbildung~\ref{fig:bragg-bedingung}
gefunden werden. Das Maximum ist hier deutlich im Bereich um
\SI{28}{\degree} ausgeprägt.

\begin{table}
  \centering\footnotesize
  \begin{tabular}{SSSSSSSS}
    \toprule
    {$\theta/\si{\degree}$} & {Zählrate} &
    {$\theta/\si{\degree}$} & {Zählrate} &
    {$\theta/\si{\degree}$} & {Zählrate} &
    {$\theta/\si{\degree}$} & {Zählrate} \\
    \midrule
26.0 & 58.0 & 27.1 & 163.0 & 28.2 & 191.0 & 29.3 & 64.0 \\
26.1 & 62.0 & 27.2 & 167.0 & 28.3 & 180.0 & 29.4 & 57.0 \\
26.2 & 74.0 & 27.3 & 173.0 & 28.4 & 166.0 & 29.5 & 49.0 \\
26.3 & 83.0 & 27.4 & 173.0 & 28.5 & 154.0 & 29.6 & 43.0 \\
26.4 & 94.0 & 27.5 & 183.0 & 28.6 & 150.0 & 29.7 & 38.0 \\
26.5 & 102.0 & 27.6 & 198.0 & 28.7 & 145.0 & 29.8 & 38.0 \\
26.6 & 120.0 & 27.7 & 197.0 & 28.8 & 125.0 & 29.9 & 38.0 \\
26.7 & 125.0 & 27.8 & 189.0 & 28.9 & 108.0 & 30.0 & 45.0 \\
26.8 & 135.0 & 27.9 & 201.0 & 29.0 & 101.0 &  &  \\
26.9 & 139.0 & 28.0 & 199.0 & 29.1 & 87.0 &  &  \\
27.0 & 154.0 & 28.1 & 199.0 & 29.2 & 79.0 &  & \\
    \bottomrule
  \end{tabular}
  \caption{Die Meßwerte zur Überprüfung der \name{Bragg}-Bedingung. Der
    angegebene Winkel ist der Zählrohrwinkel. Der Kristallwinkel war
    fest auf \SI{14}{\degree} eingestellt.}
  \label{tab:bragg-bed}
\end{table}

\begin{figure}
  \centering
  \includegraphics[width=0.7\textwidth]{bragg-bedingung}
  \caption{Plot der Meßwerte zur Untersuchung der \name{Bragg}-Bedingung}
  \label{fig:bragg-bedingung}
\end{figure}

\subsection{Emissionspektrum der Kupferanode}

Das Emissionsspektrum der verwendeten \name{Röntgen}-Röhre ist in
Abbildung~\ref{fig:cu-emission} dargestellt. Man erkennt deutlich die
beiden K-Linien. Davor befindet sich der sogenannte Bremsberg, die
flache Erhebung im Bereich von \SIrange{5}{20}{\degree}.

\begin{figure}
  \centering
  \includegraphics[width=0.7\textwidth]{cu-emission}
  \caption{Emissionsspektrum der Kupferanode. Erkennbar sind Bremsberg
    (kleine Erhebung) und die beiden scharfen Linien (rechts)}
  \label{fig:cu-emission}
\end{figure}

Zur Bestimmung der Halbwertsbreiten der beiden K-Linien werden die
Meßpunkte in einer Umgebung $U$ um die jeweilige Linie herum mithilfe
von kubischen Splines interpoliert. Mit der Interpolation $P\colon
U\to\R$ werden nun jeweils die Winkel $\theta\in U$ aus der Umgebung um
das Maximum bestimmt, an denen der Funktionswert auf die Hälfte des
Maximums gefallen ist.

Dazu wird die Funktion $H(\theta) := P(\theta) - \frac{1}{2}
\max\{P(\theta):\theta\in U\}$ definiert, die genau an den gesuchten
Winkeln Nullstellen hat. Es werden also einfach die Nullstellen von $H$
bestimmt. Der Abstand der beiden Nullstellen liefert die beiden
Halbwertsbreiten.\footnote{Zur Interpolation und Nullstellenbestimmung
  wurde die \texttt{scipy}-Bibliothek in der Version 0.11 verwendet.}

Die Nullstellen sind von der Software zu $\theta_1 =
\SI{2.047}{\degree}$ und $\theta_2 = \SI{2.097}{\degree}$ ermittelt
worden.

Das Ergebnis der Interpolationen und Halbwertsbreitenbestimmung ist in
Abbildung~\ref{fig:halbwertsbreite} zu sehen. Hier sind die Graphen der
interpolierenden Funktion $P$ in grün dargestellt. Die blauen vertikalen
Geraden schneiden die Graphen auf der Hälfte der Höhe des Maximums. Als
Ergebnis ergibt sich für die Halbwertsbreite der $\mathrm{K}_\alpha$-Linie
%
\begin{equation}
  h = \SI{0.496}{\degree}.
\end{equation}
%
Die Halbwertsbreite der $\mathrm{K}_\beta$-Linie beträgt
%
\begin{equation}
  h = \SI{0.44}{\degree}
\end{equation}

Aus den beiden Winkeln $\theta_1,\theta_2$ können durch
Gleichung~\ref{eq:photonenenergie} zugehörige Energien berechnet
werden. Diese Energiedifferenz $\Delta E$ ist ein Maß für das
Auflösungsvermögen der verwendeten Apparatur. Sie berechnet sich zu
%
\begin{equation}
  \Delta E = \SI{199}{\electronvolt}
\end{equation}
%
für die $\mathrm{K}_\alpha$-Linie und
%
\begin{equation}
  \Delta E = \SI{142}{\electronvolt}
\end{equation}
%
für die $\mathrm{K}_\beta$-Linie.

\begin{figure}
  \centering
  \includegraphics[width=0.7\textwidth]{halbwertsbreiten}
  \caption{Die Interpolation $P$ der Linien und Bestimmung der
    Halbwertsbreite. Für die $\mathrm{K}_\alpha$-Linie wurde
    $U=[\SI{20.4}{\degree}, \SI{21.2}{\degree}]$ gewählt, für die
    $\mathrm{K}_\beta$-Linie $U = [\SI{22.8}{\degree}, \SI{23.4}{\degree}]$.}
  \label{fig:halbwertsbreite}
\end{figure}


\subsection{Absorptionsspektren der verwendeten Materialien}

Aus den Abbildungen~\ref{fig:au-absorption}--\ref{fig:zr-absorption}
werden die Winkel $\theta$ der Absorptionskanten abgelesen. Mithilfe der
\name{Bragg}-Bedingung~\eqref{eq:bragg_formel} bestimmt man nun die
zugehörigen Wellenlängen, aus denen dann wiederum mithilfe von
Formel~\eqref{eq:photonenenergie} die zugehörige Absorptionsenergie bestimmt
wird. Für alle verwendeten Elemente außer Gold kann dann aus
Formel~\eqref{eq:sigma_k} die sogenannte Abschirmzahl
$\sigma_\mathrm{K}$ errechnet werden. In Tabelle~\ref{tab:absorption}
sind die berechneten Werte zu finden.

\begin{table}
  \centering\footnotesize
  \begin{tabular}{lSSSSS}
    \toprule
    & $Z$ & {$\theta/\si{\degree}$} &
    {$\lambda/\si{\pico\metre}$}&
    {$E_\mathrm{K}/\si{\kilo\electronvolt}$}&
    {$\sigma_\mathrm{K}$}\\
    \midrule
    Ge (Germanium) & 32 & 16.4 & 113.73 & 10.9 & 3.94\\
    Nb (Niob) & 41 & 10 & 69.95 & 17.72 & 5.43 \\
    Rb (Rubidium) & 37 & 12.3 & 85.81 & 14.45 & 4.79\\
    Zn (Zink) & 30 & 19.3 & 133.13 & 9.31 & 4.04 \\
    Zr (Zirkonium) & 40 & 10.5 & 73.40 & 16.89 & 5.25\\
    \bottomrule    
  \end{tabular}
  \caption{Abgelesene Winkel, Wellenlängen und Energien der Absorptionsspektren}
  \label{tab:absorption}
\end{table}

Für Gold ergeben sich zwei Kanten, für die dann eine andere
Abschirmzahl~$\sigma_\mathrm{L}$ bestimmt werden kann. Dazu wird die
Differenz der Energien, die man aus den Wellenlängen der Kanten mithilfe
von Formel~\eqref{eq:photonenenergie} erhält, in Formel~\eqref{eq:sigma_l}
eingesetzt. Die Ergebnisse dazu finden sich in
Tabelle~\ref{tab:gold-absorption}.

\begin{table}
  \centering
  \begin{tabular}{lSS}
    \toprule
    $\theta/\si{\degree}$ &13.5 & 15.5 \\
    $\lambda/\si{\pico\metre}$ & 94 & 107.6\\
    $E/\si{\kilo\electronvolt}$ & 13.19 & 11.52\\
    \midrule
    $\sigma_L$ & \multicolumn{2}{S}{4.99} \\
    \bottomrule
  \end{tabular}
  \caption{Winkel der L-Kanten von Gold und die zugehörigen
    Wellenlängen sowie Energien. Die Abschirmzahl wird aus der
    Energiedifferenz und der Kernladungszahl bestimmt.}
  \label{tab:gold-absorption}
\end{table}

\begin{figure}
  \centering
  \includegraphics[width=0.7\textwidth]{au-absorption}
  \caption{Absorptionsspektrum Gold}
  \label{fig:au-absorption}
\end{figure}

\begin{figure}
  \centering
  \includegraphics[width=0.7\textwidth]{ge-absorption}
  \caption{Absorptionsspektrum Germanium}
\end{figure}

\begin{figure}
  \centering
  \includegraphics[width=0.7\textwidth]{nb-absorption}
  \caption{Absorptionsspektrum Niob}
\end{figure}

\begin{figure}
  \centering
  \includegraphics[width=0.7\textwidth]{rb-absorption}
  \caption{Absorptionsspektrum Rubidium}
\end{figure}

\begin{figure}
  \centering
  \includegraphics[width=0.7\textwidth]{zn-absorption}
  \caption{Absorptionsspektrum Zink}
\end{figure}

\begin{figure}
  \centering
  \includegraphics[width=0.7\textwidth]{zr-absorption}
  \caption{Absorptionsspektrum Zirkonium}
  \label{fig:zr-absorption}
\end{figure}

\subsection{\name{Moseley}sches Gesetz}

Die in Tabelle~\ref{tab:absorption} aufgelisteten Kernladungszahlen mit
den zugehörigen Energien können in ein
$\sqrt{E_\mathrm{K}}$-$Z$-Diagramm aufgetragen werden und mit einer
linearen Ausgleichsrechnnung\footnote{Die lineare Regression wird mit
  der Funktion \texttt{scipy.stats.linregress} aus der
  \texttt{scipy}-Bibliothek in der Version 0.11 durchgeführt.} wird eine
Gerade durch die Meßpunkte gezogen. Das Ergebnis der Regression kann in
Abbildung~\ref{fig:moseley} betrachtet werden. Die ermittelte
Geradengleichung lautet:
%
\begin{equation}
  \label{eq:moseley-gerade}
  \sqrt{E_\mathrm{K}} = \SI{3.29}{\electronvolt^{1/2}} \cdot Z
  - \SI{1.54}{\electronvolt^{1/2}}
\end{equation}

\begin{figure}
  \centering
  \includegraphics[width=0.7\textwidth]{moseley-law}
  \caption{Diagramm zum Moseleyschen Gesetz}
  \label{fig:moseley}
\end{figure}

Die Rydberg-Energie berechnet sich aus dem Quadrat der Steigung der
Geradengleichung dann zu
%
\begin{equation}
  h c R_\infty = \SI{10.82}{\electronvolt}
\end{equation}
%
ab.

% This work is licensed under the Creative Commons
% Attribution-NonCommercial 3.0 Unported License.  To view a copy of
% this license, visit http://creativecommons.org/licenses/by-nc/3.0/.

\section{Aufbau und Durchführung}
%
\subsection{Messapparatur}
%
\begin{figure}[h]
\centering
\includegraphics[width=0.8\textwidth]{apparatur}
\caption{Schematische Darstellung der verwendeten Messapparatur.
             Bei der Messwaage handelt es sich um eine Präzisionswaage.}
\label{fig:apparatur}
\end{figure}
%
In diesem Versuch werden ein Mischungskalorimeter, eine Heizplatte und 
ein Thermoelement verwendet. Außerdem wird eine Waage verwendet, um 
die Massen der zu untersuchenden Proben zu ermitteln.
In Abb.~\ref{fig:apparatur} ist ein 
schematischer Aufbau des Versuches zu sehen.
Die gemessene Thermospannung $U_\text{th}$ wird mithilfe von 
Formel~\eqref{eq:thermo} in die Temperaturdifferenz $T$ zwischen den 
beiden Enden umgerechnet. Dabei ist die Spannung in \si{\milli\volt} 
und die Temperatur in \si{\celsius} anzugeben.
%
\begin{equation}
\label{eq:thermo}
T = \num{25.157}\cdot U_\text{th} - \num{0.19}\cdot U_\text{th}^2
\end{equation}
%
Ein Ende des Thermoelements wird in Eiswasser gehalten, welches eine 
annähernd konstante Temperatur von \SI{0}{\celsius} besitzt, sodass die 
mit Formel~\eqref{eq:thermo} errechnete Temperaturdifferenz der Temperatur 
in \si{\celsius} entspricht.
%
\subsection{Messung der spezifischen Wärmekapazität verschiedener Elemente}
In diesem Versuch werden drei Proben, welche aus Blei, Kupfer und Zinn 
bestehen, verwendet.

Als erstes werden die Massen der Proben $m_\text{k}$ mit der oben 
genannten Waage ermittelt. 
Da jede Probe an einem Deckel hängt, mit welchem beim 
Einführen der Probe in das Mischkalorimeter eben jenes gegen die 
Umgebung isoliert wird, wird das Deckelgewicht von der ermittelten 
Gesamtmasse abgezogen.

Das Mischkalorimeter wird mit \SI{600}{\milli\litre} Wasser gefüllt.
Die Masse des Wassers im Kalorimeter wird mit $m_\text{w}$ bezeichnet,
 die spezifische Wärmekapazität mit $c_\text{w}$.
Die zu untersuchende Probe wird auf $\approx$\SI{100}{\celsius} erwärmt.
Das Erwärmen der Probe wird dadurch erreicht, dass diese in ein Messbecher 
mit Wasser gebracht wird, welcher durch eine Heizplatte so lange erwärmt 
wird, bis das Wasser kocht.
Besitzt die Probe die gewünschte Temperatur, so wird vor dem einführen 
der Probe in das Kalorimeter sowohl die Wassertemperatur $T_\text{w}$ des 
in dem Kalorimeter befindlichen Wassers, als auch die Probentemperatur 
$T_\text{k}$ gemessen.
Für letzteres sind spezielle Bohrlöcher in jeder Probe vorhanden.

Ungefährt zwei Minuten nachdem die Probe in das Mischkalorimeter 
eingelassen wurde, wird kaum noch eine weitere Temperaturänderung 
registiert. Um schneller das thermische Gleichgewicht zwischen Probe und 
Kalorimeterwasser zu erreichen, wird ein Rührfisch verwendet.
 Nun wird die Mischtemperatur $T_\text{m}$ mit dem 
Thermoelement gemessen.

Dieser Messvorgang wird für Blei dreimal durchgeführt.
Für die Kupfer- und Zinnproben aus Zeitmangel jeweils nur 
einmal.

Mit Formel~\eqref{eq:waermekapa} kann aus den gemessenen Daten und 
der Wärmekapazität des Mischkalorimeters die spezifische Wärmekapazität 
$c_\text{k}$ bestimmt werden, welche sich bei konstantem Druck ergibt.
%
\begin{equation}
\label{eq:waermekapa}
c_\text{k} = \frac{(c_\text{w}m_\text{w} + c_\text{g}m_\text{g})(T_\text{m} - T_\text{w})}{m_\text{k}(T_\text{k} - T_\text{m})}
\end{equation}
%
\subsection{Messung der spezifischen Wärmekapazität des Kalorimeters}
%
Die durchscnittlich von allen Proben verdrängte Wassermenge ergibt 
sich zu $\approx$\SI{40}{\milli\litre}.
Es werden zwei Messbecher mit jeweils \SI{320}{\milli\litre} Wasser 
gefüllt. Die Volumina werden mit $V_x$ und $V_y$ bezeichnet, 
die Massen entsprechend mit $m_x$ und $m_y$. 

Das Wasser des einen Messbechers wird in das Mischkalorimeter gegeben.
Das Wasser des anderen Messbechers wird auf der Heizplatte erhizt.
Sobald das Wasser anfängt zu kochen, wird dessen Temperatur $T_y$ 
gemessen. Die Temperatur des im Kalorimeter befindlichen Wassers $T_x$ 
wird ebenfalls gemessen. 

Anschließend wird das erwärmte Wasser auch in das Kalorimeter eingefüllt.
Nach kurzer Wartezeit wird die Mischtemperatur $T_m'$ gemessen.
Gäbe es keinen Wärmeverlust durch das Kalorimeter, so wäre die Mischtemperatur
\begin{equation}
\label{eq:mischtemp1}
T_\text{m} = \frac{T_x + T_y}{2},
\end{equation}
da $V_x = V_y$.
Die Messung ergibt aufgrund der nicht verschwindenden Wärmekapazität 
des Kalorimeters einen Mischtemperaturunterschied 
\begin{equation}
\label{eq:mischtemp2}
\Delta T_\text{w} = T_\text{m} - T_\text{m}'
\end{equation}
Also muss das Kalorimeter die Wärmenergie $\Delta Q$,welche aufgrund 
der Kenntniss der spezifischen Wärmekapazität von Wasser berechnet 
werden kann, aufgenommen haben.
Es gilt:
\begin{equation}
\label{eq:wasserkapa}
\Delta Q = (m_x + m_y)c_\text{w}\Delta T_\text{w}
\end{equation}
%
Die Temperaturänderung des Kalorimeters beträgt 
\begin{equation}
\label{eq:tempdiff}
\Delta T = T_\text{m}' - T_x
\end{equation}
Aus den Gleichungen~\eqref{eq:wasserkapa} und~\eqref{eq:tempdiff} 
kann nun leicht die Wärmekapazität $c_\text{g}m_\text{g}$des Kalorimeters 
 mit Formel~\eqref{eq:kalorikapa}bestimmt werden. 
%
\begin{equation}
\label{eq:kalorikapa}
c_\text{g}m_\text{g} =\frac{\Delta Q}{\Delta T} = \frac{(m_x +m_y)\cdot c_\text{w}\cdot\left(\frac{T_x - T_y}{2} - T_\text{m}'\right)}{T_\text{m}' - T_x}
\end{equation}
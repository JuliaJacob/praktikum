% This work is licensed under the Creative Commons
% Attribution-NonCommercial 3.0 Unported License. To view a copy of this
% license, visit http://creativecommons.org/licenses/by-nc/3.0/.

\section{Diskussion}
Zum Schluss soll eine kleine Diskussion zu den in diesem Versuch
erhaltenen Ergebnissen durchgeführt werden.

\paragraph{Zur Längenmessung}
Wie man Anhand von Tabelle~\ref{tab:diskussion_laenge} sieht, beträgt
die relative Abweichung von in diesem Versuch bestimmten Kabellängen zu
tatsächlichen Kabellängen maximal \SI{45}{\percent} für das schwarze
Kabel. Bei den anderen Kabeln ist der Fehler nicht so dramatisch.
%
\begin{table}[h]
  \centering
  \begin{tabular}{SSSS}
    \toprule
    {Kabel}&{Gemessene Länge /}\si{\metre}&
    {Angegebene Länge /}\si{\metre}&{relative Abweichung}\\
    \midrule
    {Rot}&1.8&2&0.1\\
    {Schwarz}&11.6&8&0.45\\
    {Grün}&8.5&10&0.15\\
    {Kabeltrommel}&82.1&100&0.18\\
    \bottomrule
  \end{tabular}
  \caption{Vergleich zwischen den ermittelten Kabellängen und 
    den auf den Kabeln angegebenen Kabellängen.}
  \label{tab:diskussion_laenge}
\end{table}
%
Die Längenmessung ist also zum Teil nicht sehr genau.  Hierbei sind Zwei
Probleme zu nennen. Zum einen musste in der Berechnung der Kabellängen
die Dielektrizität der Koaxialkabel als \num{2.25} angenommen werden,
ohne diesen Wert zu überprüfen.  Zum anderen ist es schwierig den
zeitlichen Abstand zwischen einlaufendem und reflektierem Signal zu
ermitteln. Die Hauptschwierigkeit besteht darin, die richtigen Stellen
im Oszillatorbild zu finden, an denen der zeitliche Abstand abgelesen
werden soll. Dies wird durch das schlechte Rechtecksignal, welches von
dem in diesem Versuch verwendeten Funktionengenerator erzeugt wird,
zusätzlich erschwert.

\paragraph{Zur Belagsmessung}
Die ermittelten Widerstands- und Kapazitätsbeläge sind im untersuchten
Frequenzbereich ziemlich konstant. Bei Frequenzen unterhalb von
\SI{10}{\hertz} hat sich das RLC-Messgerät als ungenau herausgestellt.
Bei \SI{100}{\kilo\hertz} sind leichte Abweichungen in den Widerstands-
und Kapazitätsbelägen von dem vorher konstanten Wert zu erkennen. Bei
den Induktivitäts- und Querleitbelägen treten diese
Frequenzabhängigkeiten bereits ab ca. \SI{1}{\kilo\hertz} auf.

Allgemein ist die Richtigkeit der Belagswerte nicht zu überschätzen, da
die Messvorrichtung aufgrund von verwendeten Klemmen sehr empfindlich
ist. Leichte Änderungen der Klemmenposition führt direkt zu anderen
Messwerten. Es wurde versucht, diese Störung durch geschickte
Platzierung der Klemmen zu unterdrücken.

\paragraph{Zur Dämpfungskonstanten}
Die Methode, die Dämpfung mittels Betrachtung der
Fast-Fourier-Transformierten eines Rechtecksignals zu berechnen, erweist
sich als nützlich. Dadurch können selbst kleine Dämpfungen festgestellt
werden.  Allerdings muss an dieser Stelle wieder auf das schlechte
Rechtecksignal des Funktionengenerators hingewiesen werden.  Da das
Rechtecksignal aber stets die gleiche Form hatte, sollte dies keine
großen Auswirkungen auf die Bestimmung der Dämpfung mittels oben
genannter Methode haben.

\paragraph{Zur Mehrfachreflexion}
Der Signalverlauf eines mehrfach reflektierten Signals kann gut
aufgenommen werden. Die Sprünge sind im Prinzip gut ausmessbar und
liefern gute Reflexionsfaktoren.  Auch an dieser Stelle erschwert aber
das schlechte Rechtecksignal die Bestimmung der Sprunghöhen.

\paragraph{Verschiedene Abschlüsse}
Die Cursorfunktion des Oszilloskopen hat sich in diesem Teil als äußerst
Hilfreich erwiesen. Die Messwerte konnten gut aufgenommen werden. Auch
die aus der Theorie berechneten Funktionen konnten gut durch die
aufgenommenen Messwerte gefittet werden und somit die Art des
Abschlusses bestimmt werden.  Wurden andere Funktionen zum fitten
verwendet, hat sich eine große Diskrepanz zwischen gefitteter Kurve und
den Messwerten ergeben, weswegen Sicherheit besteht, dass in diesem
Versuch die richtigen Abschlüsse festgestellt wurden.\\
Jedoch kann auf diese Weise nur die für den Abschluss spezifische 
Zeitkonstante bestimmt werden. Der Versuch, die Widerstände, 
Induktivitäten und Kapazitäten der Abschlüsse festzustellen, 
erweist sich als vergebens, da es für eine Zeitkonstante unendlich 
viele Kombinationen der genannten elektronischen Größen gibt, 
welche die gleiche Zeitkonstante für das System ergeben. Somit 
sind die in diesem Protokoll angegeben Werte der Abschlusswiderstände,
 -Induktivitäten und - Kapazitäten auf keinen Fall für die 
tatsächlichen Werte der im Abschluss verbauten Teile zu halten!

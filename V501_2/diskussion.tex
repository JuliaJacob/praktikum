% This work is licensed under the Creative Commons
% Attribution-NonCommercial 3.0 Unported License. To view a copy of this
% license, visit http://creativecommons.org/licenses/by-nc/3.0/.

\section{Diskussion}
Zunächst werden die Ergebnisse der Elektronenstrahlablenkung im
elektrischen Feld betrachtet. Es ist festzustellen, dass der aus der
Theorie erhaltene Zusammenhang zwischen Ablenkspannung und
Leuchtfleckverschiebung korrekt ist. Es zeigt sich jedoch ein
Unterschied zwischen der Steigung der Ausgleichsgeraden im Plot
\ref{fig:a} und dem Wert $\frac{pL}{2d}$ von
ca. \SI{0.05}{\metre}. Diese Abweichung von der Theorie ist zu erwarten,
da bei den theoretischen Überlegungen mit zwei auf ganzer Länge
parallele Ablenkplatten gearbeitet wird, die in diesem Versuch
verwendeten Ablenkplatten diese Voraussetzung jedoch nicht erfüllen. Die
Platten besitzen einen Knick und haben zum Leuchtschirm hin einen
größeren Abstand voneinander.

Bei der Realisierung des Kathodenstrahl-Oszillographen ist es gelungen,
für kurze Zeit ein stehendes Bild auf dem Leuchtschirm zu
erhalten. Allerdings hat sich die Frequenz der Sägezahnspannung in der
Zeit leicht verändert, weswegen das erhaltene Bild nie stabil geblieben
ist. Um ein stehendes Bild zu betrachten, ist diese Methode also nicht
effektiv genug, es sei denn, die Frequenz der Sägezahnspannung ist
stabiler und feiner einzustellen, als bei dem in diesem Versuch
verwendeten Gerät.

Abschließend wird in dieser Diskussion noch kurz auf die Ergebnisse der
Elektronenstrahlablenkung im magnetischen Feld eingegangen. Die
ermittelte spezifische Ladung beträgt
\SI{-3.45e11}{\coulomb\per\kilo\gram}. Im \textcite{demtroeder-1}
ist der Literaturwert der spezifischen Ladung des Elektrons zu
\SI{-1.76e11}{\coulomb\per\kilo\gram} angegeben. Der in diesem Versuch
ermittelte Wert weicht also um ca \SI{96}{\percent} ab, was ein
beträchtlicher Unterschied ist. Allerdings ist zu beachten, dass hierbei
kleine Änderungen bei den Messgrößen zu großen Änderungen bei der zu
ermittelnden Größe führen.

Die Stärke des lokalen Erdmagnetfeldes ist mit diesem Versuch schwierig
zu messen. Insbesondere die Bestimmung des Inklinationswinkels mittels
Deklinatorium-Inklinatorium erweist sich als sehr ungenau. Bei der
Messung der Abweichung des Elektronenstrahls durch das horizontale
Magnetfeld ergeben sich die gleichen Probleme wie bei der Bestimmung der
spezifischen Ladung des Elektrons. Da außerdem nur eine einzige Messung
bezüglich dieser Abweichung vorgenommen wird, ist insbesondere keine
Aussage über den Fehler des horizontalen Magnetfeldes möglich und somit
die Güte dieser Messung nicht zu überschätzen.

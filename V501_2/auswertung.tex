% This work is licensed under the Creative Commons
% Attribution-NonCommercial 3.0 Unported License. To view a copy of this
% license, visit http://creativecommons.org/licenses/by-nc/3.0/.

\section{Auswertung}
%
\subsection{Elektronenstrahl im E-Feld}
%
Die Elektronen werden in diesem Versuch in die y-Achse der
Kathodenstrahlröhre mithilfe von Ablenkplatten beschleunigt, weswegen
die Abmessungen dieser verwendet werden. Der Abstand L zwischen den
Ablenkplatten und dem Schirm beträgt \SI{0.143}{\metre}, die
Plattenlänge p = \SI{0.019}{\metre} und der Plattenabstand d ist zu
\SI{0.0038}{\metre} bemessen.
%
\subsubsection{Empfindlichkeit der Kathodenstrahlröhre}
%
Die Empfindlichkeit E einer Kathodenstrahlröhre bezeichnet den
Quotienten $\frac{D}{U_d}$, also die Ablenkung pro angelegter
Ablenkspannung. In Abb.~\ref{fig:empfindlichkeit} ist ein Plot der
aufgenommenen Ablenkungen des Elektronenstrahls gegen die angelegte
Ablenkspannung für die Beschleunigungsspannungen $U_B$ =
\SIlist{200;260;320;440;500}{\volt} abgebildet. Im selben Plot sind auch
die jeweiligen Ausgleichsgeraden zu finden.
%
Die Koeffizienten und deren Fehler der Ausgleichsgerade werden durch
eine lineare Ausgleichsrechnung bestimmt\footnote{Dazu wurde
  \texttt{ipython} in der Version 0.13 verwendet}.
%
\begin{figure}
\centering
\includegraphics{empfindlichkeit.pdf}
\caption{Ablenkung D gegen Ablenkspannung $U_d$}
\label{fig:empfindlichkeit}
\end{figure}
%
In Tabelle \ref{tab:empfindlichkeiten} sind die errechneten
Empfindlichkeiten E, sowie deren Fehler, zu den verschiedenen
Beschleunigungsspannungen $U_B$ zu finden.
%
\begin{table}
  \centering
  \begin{tabular}{S S S}
    \toprule
    $U_B${ in V}& {|E| in \si{\milli\metre\per\volt}} &
    $\Delta ${E in \si{\milli\metre\per\volt}}\\
    \midrule
     200 & 1.61& 0.03 \\
     260 & 1.19& 0.02 \\
     320 & 1.03& 0.02 \\
     440 & 0.76& 0.02 \\
     500 & 0.65& 0.01 \\
 \bottomrule
  \end{tabular}
  \caption{Ermittelte Empfindlichkeiten der Kathodenstrahlröhre}
  \label{tab:empfindlichkeiten}
\end{table}
%
Ein Plot dieser ermittelten Werte, sowie eine Ausgleichsgerade durch
diese ist in Abb. \ref{fig:a} zu finden.
%
\begin{figure}
\centering
\includegraphics{a.pdf}
\caption{Ermittelte Empfindlichkeiten E gegen $U_B^{-1}$}
\label{fig:a}
\end{figure}
%
Die Steigung a der Ausgleichsgeraden in Abb. \ref{fig:a} beträgt
\SI{-0.3109}{\metre} $\pm$ \SI{0.0116}{\metre}. Steckt man die
Anschlüsse des Multimeters um, so ergibt sich ein positiver Wert für
a. Aufgrund von theoretischen Überlegungen sollte dieser Wert mit der
Größe $\frac{pL}{2d}$ übereinstimmen. Es ergibt sich, dass
$\frac{pL}{2d}$ = \SI{0.3575}{\metre} ist.
%
\subsubsection{Kathodenstrahl-Oszillograph}
%
Nahezu unbewegte Bilder auf dem Leuchtschirm erhält man bei den in
Tabelle \ref{tab:freq} angegebenen Werten. Dort sind ebenfalls die
dazugehörigen Amplituden des Sinusbildes zu finden. Als Frequenz $\nu$
des Sinusgenerators errechnet sich als Mittelwert mit Standardabweichung
\begin{equation*}
\nu  = \SI{79.7}{\hertz} \pm \SI{0.1}{\hertz}.
\end{equation*}
Bei der Verwendung des im vorherigen Abschnitts Wertes für a, ergibt
sich als Scheitelwert $\hat{U}$ des Sinusgenerators zu
\begin{equation*}
\hat{U} = \SI{3.86}{\volt} \pm\SI{0.22}{\volt}.
\end{equation*}
Der Fehler hierbei wird mithilfe der in Formel \eqref{eq:gauss_e}
angegebenen Gau\ss schen Fehlerforpflanzung berechnet.
%
\begin{table}
  \centering
  \begin{tabular}{S S S}
    \toprule
    {Frequenzverhältnis n}& $\nu${ in \si{\hertz}} & $A_{FPP}${ in \si{\milli\metre}}\\
    \midrule
     0.5 &39.9& 31.75 \\
     1 & 79.5&31.75 \\
     2 & 159.5& 29.21 \\
     3 & 239.2& 29.21 \\
 \bottomrule
  \end{tabular}
  \caption{Gemessene Werte bei stehendem Bild}
  \label{tab:freq}
\end{table}
%
\begin{equation}
\label{eq:gauss_e}
\Delta \hat{U} = \sqrt{\left(\frac{-\bar{D} \cdot U_B}{\bar{a}^2} \cdot \Delta a\right)^2 + \left(\frac{U_B}{\bar{a}}  \cdot \Delta D\right)^2}
\end{equation}
%
\subsection{Elektronenstrahl im B-Feld}
%
Die in diesem Versuch verwendete Kathodenstrahlröhre gleicht in ihrem
Aufbau und Abmessungen der Röhre, die im vorherigen Versuch verwendet
wurde.
%
\subsubsection{Spezifische Ladung des Elektrons}
%
In Abb. \ref{fig:b} sind der Quotient $\frac{D}{L^2 + D^2}$ gegen die
verwendeten Magnetfeldstärken B bei zwei verschiedenen
Beschleunigungsspannungen, sowie eine Ausgleichsgerade durch die
Messwerte aufgetragen. Die Stärke des Magnetfeldes eines
Helmholtzspulenpaares mit N Windungen und dem Radius R erhält man aus
der verwendeten Stromstärke I nach Formel
\eqref{eq:mag-force-helmholtz}. Aus dem Proportionalitätsfaktor A ergibt
sich nach Formel \eqref{eq:spec-charge-measurement} die spezifische
Ladung $\frac{e_0}{m_0}$ durch die in Gleichung \eqref{eq:spez_Ladung}
wiedergegebene Rechenvorschrift.
%
\begin{equation}
\label{eq:spez_Ladung}
\frac{e_0}{m_0} = 8 \cdot U_B \cdot A^2
\end{equation}
%
\begin{figure}
\centering
\includegraphics{b.pdf}
\caption{Zur spezifischen Ladung: Quotient $\frac{D}{L^2 + D^2}$ in 1/m gegen B in T}
\label{fig:b}
\end{figure}
%
Als Steigung dieser Geraden errechnen sich die in Tabelle
\ref{tab:steigungen} wiedergegebenen Werte. Ebenfalls in dieser Tabelle
zu finden sind die daraus resultierenden Werte für die spezifische
Ladung des Elektrons, sowie die Fehler der einzelnen Ergebnisse. Der
Fehler der spezifischen Ladung wird mithilfe einer Gau\ss schen
Fehlerfortpflanzung von \eqref{eq:spez_Ladung}. Dabei ist A, also die
Steigung der Geraden in Abb. \ref{fig:b} die Fehlerbehaftete Größe.
%
\begin{table}
  \centering
  \begin{tabular}{S S S S}
    \toprule
    $U_B${ in V} & {A in \si{\metre\per\volt\second}} & $\frac{e_0}{m_0}${ in \si{\coulomb\per\kilo\gram}} & $\Delta \frac{e_0}{m_0}${ in \si{\coulomb\per\kilo\gram} }\\
    \midrule
     250 &-13030.4& 3.4 $\cdot 10^{11}$ & 6 $\cdot 10^{7}$ \\
     450 & -9872.4& 3.5 $\cdot 10^{11}$& 3 $\cdot 10^{7}$ \\
 \bottomrule
  \end{tabular}
  \caption{Proportionalitätsfaktoren und spez. Ladung}
  \label{tab:steigungen}
\end{table}
%
Somit ergibt sich in diesem Versuch der Wert der spezifischen Ladung zu
\begin{equation*}
\frac{e_0}{m_0} = \SI{3.45e11}{\coulomb\per\kilo\gram}
\end{equation*}
Es wird kein Fehler beim endgültig bestimmten Wert angegeben, da dieser
kleiner ist, als die niedrigste betrachtete Nachkommastelle.
\subsubsection{Lokale Erdmagnetfeldstärke}
%
Nach der Drehung des Kathodenstrahlrohres von Nord-Süd-Richtung in
Ost-West-Richtung wird eine Stromstärke von \SI{0.85}{\ampere} in den
verwendeten Helmholtzspulen benötigt, um mit diesen ein Magnetfeld zu
erzeugen, welches die durch das Erdmagnetfeld verursachte Ablenkung des
Strahls ausgleicht. Nach Formel \eqref{eq:mag-force-helmholtz} ergibt
sich somit für die horizontale Magnetfeldstärke $B_hor$ ein Wert von
\SI{50.4}{\micro\tesla}.

Um nun die Gesamtfeldstärke zu berechnen wird der Inklinationswinkel
$\phi$ des Magnetfeldes in Versuchsnähe benötigt. In Tabelle
\ref{tab:winkel} sind die in einer Messreihe bestimmten
Inklinationswinkel angegeben. Als Mittelwert ergibt sich $\phi$
$\approx$ \SI{70.5}{\degree} $\pm$ \SI{9.4}{\degree}. Daraus lässt sich
leicht mithilfe von Trigonometrie die gesamte lokale Erdmagnetfeldstärke
B bestimmen. Es ergibt sich, dass
\begin{equation*}
B = \SI{0.162}{\milli\tesla}.
\end{equation*}
Aufgrund der ungenauen Winkelmessung ergibt sich für diesen Wert der
Magnetfeldstärke ein Fehler, der größer ist, als der ermittelte Wert.
%
\begin{table}
  \centering
  \begin{tabular}{S S}
    \toprule
    {Messung Nr.}& {Inklinationswinkel in \si{\degree}}\\
    \midrule
     1 &70\\
     2 & 86 \\
     3 &40\\
     4 & 86 \\
 \bottomrule
  \end{tabular}
  \caption{Messwerte zur Bestimmung des Inklinationswinkels}
  \label{tab:winkel}
\end{table}

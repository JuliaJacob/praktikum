% This work is licensed under the Creative Commons
% Attribution-NonCommercial 3.0 Unported License. To view a copy of this
% license, visit http://creativecommons.org/licenses/by-nc/3.0/.

\section{Diskussion}
Die Magnetfeldmessung zeigt die erwarteten Ergebnisse. Das Feld ist in
der Nähe der Probe maximal und zeigt über einen sehr kleinen Bereich
Homogenität.  Weit außerhalb fällt das Feld stark ab.

Die Bestimmung der Faraday-Rotation ist nicht optimal gelungen.  Die
Messfehler sind ziemlich groß und wie man in \cref{fig:linregress}
erkennen kann, streuen die Punkte stark um die Regressionsgerade. Das
Verhältnis $m^*/m_\text{e}$ ist in \cite{ecee-colorado} für
Galliumarsenid mit \num{0.067} angegeben.  Unser Wert weicht mit
\num{0.063} um \SI{5.7}{\percent} ab.  Dies erscheint zunächst nicht
schlecht, allerdings kann wenig Vertrauen in diesen Wert gesetzt werden,
da die geschätzten Fehler in der Regressionsrechnung sehr groß waren.
Das lässt sich zum Teil auf die beschädigten Interferenzfilter
zurückführen, von denen einige Kratzer oder Löcher aufwiesen. Außerdem
befindet sich am Ort der Probe kein perfekt homogenes Magnetfeld, wie es
leicht in Abbildung~\ref{fig:magnetfeld} zu sehen ist und Anhand der
Messwerte in Tabelle~\ref{tab:magnetmess} festzustellen ist, dass das
Magnetfeld sich über die Probendicke von ca. \SI{1}{\milli\metre}
bereits um mindestens \SI{1}{\milli\tesla} ändert.

Es kann aber abschließend gesagt werden, dass der Faraday-Effekt 
eine Möglichkeit bietet, die effektive Masse von Elektronen in 
Halbleitern zu bestimmen. Für exakte Messungen sind aber 
bessere Geräte als die in diesem Versuch verwendeten Apparaturen 
notwendig. 

% This work is licensed under the Creative Commons
% Attribution-NonCommercial 3.0 Unported License. To view a copy of this
% license, visit http://creativecommons.org/licenses/by-nc/3.0/.

\section{Auswertung}

\subsection{Magnetfeldmessung}
Nach der Justage der Apparatur wird die Komponente des Magnetfeldes,
welche parallel zum Lichtstrahl verläuft, gemessen.  Dies geschieht mit
einer Hallsonde.  Die örtliche Abhängigkeit des Magnetfeldes ist in
\cref{fig:magnetfeld} zu sehen.  Die maximale magnetische Flußdichte
beträgt $B_\text{max} = \SI{448}{mT}$.

\begin{figure}
  \centering
  \includegraphics[width=0.8\textwidth]{abbildungen/magnetfeld.pdf}
  \caption{Die grapische Darstellung der Ergebnisse der
    Magnetfeldmessung.  Das Maximum ist deutlich zu erkennen.  Außerdem
    beschränkt sich das Magnetfeld auf die Zone, in der sich die Probe
    befindet.}
  \label{fig:magnetfeld}
\end{figure}

\subsection{Die effektiven Masse der Leitungselektronen in GaAs}

In \cref{fig:faraday-rot} ist die Faraday-Rotation der drei vermessenen
Proben gegen die Wellenlänge aufgetragen.  Die Faraday-Rotation wird aus
den erhaltenen Meßdaten gemäß \cref{eq:drehwinkel} berechnet. Das ist in
\cref{tab:faraday-rotation} dargestellt.

Zur Berechnung der effektiven Masse der freien Leitungselektronen
bestimmen wir die Differenz $\Delta\theta$ der Faraday-Rotationen von
jeweils einer dotierten und der hochreinen Probe.  Gemäß Formel
\eqref{eq:winkelfrei} kann ein lineares Regressionmodell für die
Meßdaten $y = (y_i) = (\Delta\theta_i)$ und $x = (x_i) = (\lambda_i^2)$
zugrundegelegt werden:
%
\begin{equation}
  \label{eq:linregress}
  y_i = \beta_0 x_i + \beta_1.
\end{equation}
Die Ausgleichsrechnung wird von der Bibliothek \texttt{scipy.stats}
übernommen, die in der Version 0.10 verwendet wird und eine Funktion
\texttt{scipy.stats.linregress} enthält, die die Parameter $\beta_0$ und
$\beta_1$ und die Standardabweichung $s$ berechnet.  Gemäß den Formeln
%
\begin{align}
  \label{eq:stat-formeln}
  s_{\beta_0} &= \frac{s}{\sqrt{n}} \sqrt{1 + \frac{n\bar{x}^2}
    {\operatorname{var}(x)}}\\
  s_{\beta_1} &= \frac{s}{\sqrt{n \operatorname{var}(x)}}
\end{align}
%
werden dann die Standardabweichungen der Parameter geschätzt.  Das
Ergebnis der Rechnung ist in \cref{tab:linregress} numerisch und in
\cref{fig:linregress} graphisch gegeben.  Aus Formel
\eqref{eq:winkelfrei} wird abgelesen, daß für den Zusammenhang zwischen
effektiver Masse und dem Parameter $\beta_0$ gilt:
%
\begin{equation}
m* = \sqrt{\frac{e_0^3\lambda^2}{8\pi^2\varepsilon_0c^3\beta_0} \frac{NBL}{n}}
\end{equation}
%
Der Brechungsindex $n = \num{3.455}$ von GaAs ist \cite{filmetrics}
entnommen. \Cref{tab:linregress} zeigt die Ergebnisse, die aus dieser
Formel erhalten werden.  Im nächsten Abschnitt werden diese Ergebnisse
kurz diskutiert.

\begin{table}\centering
  \begin{tabular}{SSSSSSSSSS}
    \toprule
    &
    \multicolumn{3}{c}{hochreine Probe} &
    \multicolumn{3}{c}{schwach dotierte Probe} &
    \multicolumn{3}{c}{stark dotierte Probe}
    \\
    \cmidrule(rl){2-4}
    \cmidrule(rl){5-7}
    \cmidrule(rl){8-10}
    {$\lambda/\si{\micro\meter}$} &
    {$\theta_1/\si{\degree}$} &
    {$\theta_2/\si{\degree}$} &
    {$\theta/\si{\degree}$} &
    {$\theta_1/\si{\degree}$} &
    {$\theta_2/\si{\degree}$} &
    {$\theta/\si{\degree}$} &
    {$\theta_1/\si{\degree}$} &
    {$\theta_2/\si{\degree}$} &
    {$\theta/\si{\degree}$}
    \\
    \midrule
    1.06 & 315. & 337.67 & 11.33 & 75. &  64.5 &  5.25 &
    320.67 & 333. & 6.17\\
    1.29 & 335. & 320. & 7.5 & 64.7 & 72.55 & 3.925 & 333. & 323. &
    5.\\
    1.45 & 320.33 & 333. & 6.33 & 74. & 62.42 & 5.79 &
    321.33 &  330.67 & 4.67\\
    1.72 & 332.57 & 324. & 4.28 & 61.6 & 71.5 & 4.95 & 333. & 324. &
    4.5\\
    1.96 & 330.75 & 337.25 & 3.25 & 336. & 329. & 3.5 & 328. & 339. &
    5.5\\
    2.16 & 336. & 329.58 & 3.21 & 59. & 66. & 3.5 & 340. & 327. & 6.5\\
    2.34 & 355.83 & 358.53 & 1.35 & 43. & 35. & 4. & 350.5 & 369. &
    9.25\\
    2.51 & 13. & 22. & 4.5 & 19.33 & 11.42 & 3.96 & 21. & 13. & 4. \\
    2.65 & 53.43 & 58.67 & 2.62 & 345. & 335.67 & 4.67 & 331.56 & 347. &
    7.71\\ 
    \bottomrule
   \end{tabular}
   \caption{Hier sind die gemessenen Winkel $\theta_1$ und $\theta_2$
     eingetragen.  Gemäß \cref{eq:drehwinkel} sind die Winkel $\theta$
     ausgerechnet worden.  Die schwach dotierte Probe hat eine
     Ladungsträgerdichte $N=\SI{1.2e18}{cm^{-3}}$ und ist \SI{1.296}{mm}
       lang, die stark dotierte Probe hat $n=\SI{2.8e18}{cm^{-3}}$ und
       ist \SI{1.36}{mm}.}
  \label{tab:faraday-rotation}
\end{table}

\begin{figure}
  \centering
  \includegraphics[width=0.8\textwidth]{abbildungen/faraday-rotation}
  \caption{Hier ist die Faraday-Rotation der drei verschiedenen Proben
    gegen die Wellenlänge aufgetragen.  Das oberste Diagramm zeigt die
    Faraday-Rotation der hochrei}
  \label{fig:faraday-rot}
\end{figure}

\begin{figure}
  \centering
  \includegraphics[width=0.8\textwidth]{abbildungen/effektive_masse}
  \caption{Hier sind die Differenzen der Faraday-Rotationen von
    dotierter und hochreiner Probe gegen das Wellenlängenquadrat
    aufgetragen.  Nach Formel \eqref{eq:winkelfrei} ist ein linearer
    Zusammenhang zu vermuten.  Die eingezeichneten Geraden zeigen die
    Ergebnisse einer linearen Regressionsrechnung.}
  \label{fig:linregress}
\end{figure}

\begin{table}
  \centering
  \begin{tabular}{lSSSS}
    \toprule
    {Dotierung} &
    {$\beta_0/\si{\degree\per\meter}$} &
    {$\beta_1/\si{\degree}$} &
    {$m^*/(10^{-31}\si{kg})$} &
    {$m^*/m_\text{e}$}
    \\
    \midrule
    schwach & 0.982(236) & -4.369(54) & 2.170 & 0.238 \\
    stark & 1.474(345) & -4.766(79) & 2.640 & 0.290 \\
    \midrule
    Mittelwerte & & & 2.405 & 0.264\\
    \bottomrule
  \end{tabular}
  \caption{Hier sind die Ergebnisse der linearen Ausgleichsrechnung und
    die Berechnung der effektiven Masse dargestellt.}
  \label{tab:linregress}
\end{table}

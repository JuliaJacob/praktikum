% This work is licensed under the Creative Commons
% Attribution-NonCommercial 3.0 Unported License. To view a copy of this
% license, visit http://creativecommons.org/licenses/by-nc/3.0/.

\section{Diskussion}

Insgesamt sind in der Auswertung relativ hohe Fehler beobachtet
worden. Die Gründe dafür lassen sich auf zwei Punkte zusammenfassen.
%
\begin{itemize}
\item Der ohmsche Widerstand der Spule ist nicht beachtet worden
\item Der Fall der schwachen Dämpfung ist nur annähernd gegeben, da
  \begin{equation*}
    \frac{R^2}{2 L^2} LC = \num{0.002},
  \end{equation*}
  und damit nicht allzu nah am Dämpfungsfall $R^2/(2L^2) \ll 1/(LC)$
  ist.
\end{itemize}

Die Abweichung von \SI{112}{\percent} bei der Bestimmung des effektiven
Dämpfungswiderstands sind auf den ersten Punkt zurückzuführen. Auch bei
der Bestimmung des aperiodischen Grenzfalls läßt sich der Fehler auf
diesen Punkt zurückführen, ebenso bei der Bestimmung der
Resonanzüberhöhung.

Die Auswertung der Breite der Resonanzkurve weißt den höchsten Fehler
von \SI{133}{\percent} auf. Hier kommt der zweite Punkt zum Tragen. Die
gemachte Näherung approximiert hier nur schlecht.

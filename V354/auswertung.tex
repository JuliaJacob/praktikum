% This work is licensed under the Creative Commons
% Attribution-NonCommercial 3.0 Unported License. To view a copy of this
% license, visit http://creativecommons.org/licenses/by-nc/3.0/.

\section{Auswertung}

\subsection{Bestimmung des  effektiven Dämpfungswiderstands}
Zur Bestimmung des effektiven Dämpfungswiderstands werden aus
Abbildung~\ref{fig:thermodruck} die Minima und Maxima abgelesen, die
sich in Tabelle~\ref{tab:minmax-thermo} finden. Zur Bestimmung des
Exponenten $m := -R/(2L)$ in Gleichung~\eqref{eq:loesung} werden die
Maxima und Minima einzeln behandelt. Aus diesem Exponent kann dann
mithilfe von
\begin{equation}
  R = -2mL
  \label{eq:exp-widerstand}
\end{equation}
der effektive Dämpfungswiderstand bestimmt werden.
Die Meßpunkt werden in ein Koordinatensystem eingetragen, dessen
Ordinate logarithmisch skaliert ist. Die Ausgleichsgerade
%
\begin{equation}
	y = \log U = mt + b
\end{equation}
wird mithilfe der Formeln
\begin{align*}
  m &= \frac{N \sum_i x_i y_i - \sum_i\sum_j x_i y_i} {N \sum_i x_i^2 -
    \sum_{i, j} x_i x_j} & b &= \frac{ \sum_{i,j} x_i y_j - \sum_{i, j}
    x_i x_j y_j} {N \sum_i x_i^2 - \sum_{i, j} x_i x_j}
\end{align*}
für $i,j, k\in\{1,\dotsc,n\}$ bestimmt. Die Ergebnisse für die beiden
Geraden sind:
\begin{align*}
  m_1 &= \SI{-4960}{\per\second} & m_2 &= \SI{-6454}{\per\second}\\
  b_1 &= \num{329}& b_2 &= \num{139}
\end{align*}
%
Der Mittelwert der Steigung errechnet sich zu
%
\begin{equation*}
  \overline{m} = \SI{-5707}{\per\second}.
\end{equation*}
Daraus wird der effektive Dämpfungswiderstand gemäß
Formel~\eqref{eq:exp-widerstand} berechnet. Es ergibt sich
\begin{equation*}
  R_\text{eff} = \SI{116}{\ohm}.
\end{equation*}
Die relative Abweichung vom erwarteten Wert~$R$, der in
Tabelle~\ref{tab:geraetedaten} zu finden ist, beträgt
\begin{equation*}
  \frac{|R - R_\text{eff}|}{R} = \SI{112}{\percent}
\end{equation*}

\begin{table}
  \centering\footnotesize
  \begin{tabular}{SSSSSSSS}
    \toprule 
    \multicolumn{2}{c}{Maxima} & \multicolumn{2}{c}{Minima} &
    \multicolumn{2}{c}{Maxima} & \multicolumn{2}{c}{Minima} \\
    {$t/\si{\micro\second}$} & {$U_C/\si{\volt}$} &
    {$t/\si{\micro\second}$} & {$U_C/\si{\volt}$} &
    {$t/\si{\micro\second}$} & {$U_C/\si{\volt}$} &
    {$t/\si{\micro\second}$} & {$U_C/\si{\volt}$} \\
    \cmidrule(r){1-2} \cmidrule(rl){3-4}
    \cmidrule(rl){5-6} \cmidrule(l){7-8}
      0.0 & 6.4 &  15.0 & 6.4 &  35.0 & 5.2 &  45.0 & 5.2 \\ 
     60.0 & 4.4 &  75.0 & 4.8 &  90.0 & 3.6 & 105.0 & 4.0 \\ 
    120.0 & 2.8 & 135.0 & 3.2 & 150.0 & 2.4 & 165.0 & 2.8 \\ 
    180.0 & 2.0 & 195.0 & 2.4 & 215.0 & 1.6 & 225.0 & 2.4 \\ 
    \bottomrule
  \end{tabular}
  \caption{Hier sind die aus der Abbildung~\ref{fig:thermodruck}
    entnommenen Punkte der Einhüllenden. Die Punkte sind nach Maxima und
    Minima getrennt. Die Wertepaare $(t, \log U)$ werde in eine lineare
    Ausgleichsrechnung gegeben. In Abbildung~\ref{fig:minmax-plot} sind
    die Meßwerte in ein Koordinatensystem eingetragen, das auf der
    Ordinate logarithmisch skaliert ist.}
  \label{tab:minmax-thermo}
\end{table}

\begin{figure}
  \centering \includegraphics{bild}
  \caption{Diese Graphik wurde mit dem Oszilloskop aufgenommen.  Es ist
    der zeitliche Verlauf der Kondensatorspannung aufgezeichnet. Der
    exponentielle Abfall der Amplitude ist zu erkennen.}
  \label{fig:thermodruck}
\end{figure}

\begin{figure}
  \centering \includegraphics[width=0.7\textwidth]{exp-plot}
  \caption{Hier sind die aus Abbildung~\ref{fig:thermodruck} abgelesenen
    Werte halblogarithmisch aufgetragen. Die zwei Geraden sind deutlich
    zu erkennen. Die obere Ausgleichsgerade verläuft durch die Minima
    der Amplitude, die untere durch die Maxima. Die beiden Steigungen
    sind dann gemittelt worden.}
  \label{fig:minmax-plot}
\end{figure}  

\subsection{Bestimmung des aperiodischen Grenzwerts}

In Tabelle~\ref{tab:geraetedaten} finden sich die Daten der verwendeten
Bauteile für den Versuch. Gemäß Formel~\eqref{eq:rap} ergibt sich für
den erwarteten Widerstand~$R_\text{ap}$ und den experimentell
ermittelten Widerstand~$R^\text{(exp)}_\text{ap}$ im aperiodischen
Grenzfall
\begin{align*}
  R_\text{ap} &= \SI{4407}{\ohm}, & 
  R^\text{(exp)}_\text{ap} &= \SI{3410}{\ohm}.
\end{align*}
Hieraus ergibt sich der relativer Fehler 
\begin{equation*}
  \frac{|R^\text{(exp)}_\text{ap} - R_\text{ap}|}{R_\text{ap}} =
  \SI{22.6}{\percent}. 
\end{equation*}

\begin{table}
  \centering
  \begin{tabular}{SSS}
    \toprule 
    \multicolumn{3}{c}{Gerätedaten}\\
    \midrule
    {$L/\si{\milli\henry}$} & 
    {$C/\si{\micro\farad}$} & 
    {$R/\si{\ohm}$}\\
    10.14 & 2.088 & 54.7\\
    \bottomrule
  \end{tabular}
  \caption{Für den Versuch ist Gerät~2 verwendet worden. Die
    entsprechenden Werte für die Kapazität, Induktivität und
    Dämpfungswiderstand können hier abgelesen werden.}
  \label{tab:geraetedaten}
\end{table}


\subsection{Bestimmung der Resonanzüberhöhung}

Die Meßwerte, die in diesem Teil des Versuches aufgenommen worden sind,
finden sich in Tabelle~\ref{tab:resonanz}. In
Abbildung~\ref{fig:resonanz} ist im oberen Plot das Verhältnis $U_C/U_0$
gegen die Frequenz $\nu$ halblogarithmisch aufgetragen und im unteren
Plot ist $U_C/U_0$ im Bereich um die Resonanzfrequenz herum linear gegen
$\nu$ aufgetragen. Dabei wurden alle Meßwerte beachtet, für die die
Kondensatorspannung $U_C > \SI{60}{\volt}$ ist. Die nach
Formel~\eqref{eq:} berechnete Resonanzüberhöhung beträgt
%
\begin{equation*}
  q = \num{40.29}.
\end{equation*}
%
Aus dem maximalen Wert von $U_C/U_0$ wird der experimentell bestimmte
Wert der Resonanzüberhöhung zu
%
\begin{equation*}
  q^\text{(exp)} = \num{14.1}
\end{equation*}
%
ermittelt. Die relative Abweichung beträgt
%
\begin{equation*}
  \frac{|q^\text{(exp)} - q|}{q} = \SI{65}{\percent}.
\end{equation*}
Mit Formel~\eqref{eq:} kann die Breite der Resonanzspitze aus den
Geräte-Daten berechnet werden. Es ergibt sich:
%
\begin{equation*}
  \nu_+ - \nu_- = 
\end{equation*}

\begin{table}
  \centering\footnotesize
  \begin{tabular}{SSSSSSSS}
    \toprule
    {$\nu / \si{\kilo\hertz}$} & {$ U_C/\si{\volt}$} &
    {$\nu / \si{\kilo\hertz}$} & {$ U_C/\si{\volt}$} &
    {$\nu / \si{\kilo\hertz}$} & {$ U_C/\si{\volt}$}\\
    \midrule
    10.0 & 16.0  &  11.0 & 46.0  &  12.0 & 20.5  &  13.0 & 11.5  \\
    14.0 & 10.0  &  15.0 & 10.0  &  16.0 & 10.5  &  17.0 & 11.5  \\
    18.0 & 12.0  &  19.0 & 12.5  &  20.0 & 13.5  &  21.0 & 14.5  \\
    22.0 & 15.5  &  23.0 & 16.5  &  24.0 & 18.0  &  25.0 & 19.5  \\
    26.0 & 22.0  &  27.0 & 25.0  &  28.0 & 29.0  &  29.0 & 34.0  \\
    30.0 & 42.0  &  30.5 & 49.0  &  31.0 & 57.0  &  31.5 & 68.0  \\
    32.0 & 84.0  &  32.5 & 110.0 &  33.0 & 140.0 &  33.5 & 155.0 \\
    34.0 & 140.0 &  34.5 & 110.0 &  35.0 & 86.0  &  35.5 & 79.0  \\
    36.0 & 56.0  &  36.5 & 48.0  &  37.0 & 41.0  &  37.5 & 36.0  \\
    38.0 & 31.0  &  38.5 & 28.5  &  39.0 & 26.0  &  39.5 & 23.5  \\
    40.0 & 21.5  &  41.0 & 18.5  &  42.0 & 16.0  &  43.0 & 14.0  \\
    44.0 & 12.5  &  45.0 & 11.5  &  46.0 & 10.0  &  47.0 & 9.0   \\
    48.0 & 8.5   &  49.0 & 8.0   &  50.0 & 7.5 \\
    \bottomrule
  \end{tabular}
  \caption{Hier finden sich die Werte der Kondensatorspannung in
    Abhängigkeit von der Frequenz gemessen.}
  \label{tab:resonanz}
\end{table}

\begin{figure}[h]
  \centering
  \includegraphics[width=0.7\textwidth]{resonanz}
  \caption{Das Verhältnis $U_C/U_0$ ist gegen $\nu$ einmal
    halblogarithmisch und einmal linear aufgetragen. Beim linearen Plot
    ist nur der Bereich um die Resonanzspitze beachtet worden.}
  \label{fig:resonanz}
\end{figure}

\begin{figure}[h]
  \centering
  \includegraphics[width=0.7\textwidth]{phasen-plot}
  \caption{Hier ist der Phasenunterschied $\phi$ gegen die Frequenz
    $\nu$ aufgetragen. Im oberen Plot sind alle Meßwerte
    halblogarithmisch und im unteren Plot ist der Bereich um
    \SI{90}{\degree} linear dargestellt.}
  \label{fig:phasen-plot}
\end{figure}

% This work is licensed under the Creative Commons
% Attribution-NonCommercial 3.0 Unported License. To view a copy of this
% license, visit http://creativecommons.org/licenses/by-nc/3.0/.

\section{Theorie}
%
\subsection{Fraunhofersche Beugung}
Bei der Fraunhoferschen Beugung wird Licht, welches von einer weit entfernten Lichtquelle stammt, an einem Spalt gebeugt. Die am Spalt ankommenden Lichtstrahlen sind aufgrund der großen Entfernung zur Lichtquelle parallel zueinander.
Die Beugung des Lichtes am Spalt kann mithilfe des Fresnel-Huygensschen-Prinzips erklärt werden. Dieses besagt, dass von jedem Punkt einer Wellenfront eine sphärische Elementarwelle ausgeht. Die Superposition aller Elementarwellen einer Wellenfront bildet wiederum eine neue Wellenfront.
Die gebeugten Lichtstrahlen verlaufen ebenfalls parallel zueinander, besitzen nun allerdings einen Phasengangunterschied, wodurch es zu Interferenzerscheinungen kommt. Bei Verwendung einer Sammellinse werden die parallelen, gebeugten Lichtstrahlen im Brennpunkt der Linse vereinigt. Dadurch entsteht in der Brennebene eine sogenannte Beugungsfigur. 
%
\subsection{Beugung am Einzelspalt}
Da bei der Fraunhoferschen Beugung parallele, ebene Wellen am Spalt ankommen, besitzt die Welle die in Formel \eqref{eq:ebene_Welle} wiedergegebene Form. Dabei breitet sich die Welle in z-richtung aus. Mit A wird die Amplitude am Ort z zur Zeit t bezeichnet.
%
\begin{equation}
A(z,t) = A_0 \cdot \exp(i(\omega t - 2 \pi z / \lambda))
\label{eq:ebene_Welle}
\end{equation}
%
Die Phasendifferenz $\delta$ der gebeugten Lichtstrahlen lässt sich aus Abb. \ref{fig:phasendifferenz} leicht zu dem in Formel \eqref{eq:phasendifferenz} angegebenen Ausdruck herleiten.
\begin{figure}
\centering
\includegraphics{phasendifferenz.pdf}
\caption{Skizze zur Bestimmung der Phasendifferenz der gebeugten Lichtstrahlen}
\label{fig:phasendifferenz}
\end{figure}
%
\begin{equation}
\delta  = \frac{2 \pi \cdot x \cdot \sin(\phi)}{\lambda}
\label{eq:phasendifferenz}
\end{equation}
%
Um einen Ausdruck für die Amplitude B zu einem Zeitpunkt t im Abstand z zum Spalt in Richtung $\phi$ zu erhalten, wird über alle Strahlen, die um den Winkel $\phi$ gebeugt werden unter Berücksichtigung der Phasendifferenz summiert. Im Limes entsteht dadurch ein Integral. Nach Ausführen des Integrals wird nur der Faktor betrachtet, welcher die Amplitude B($\phi$) beschreibt. Dieser Faktor ist in \eqref{eq:amplitude_einzelspalt} zu finden.
%
\begin{equation}
B(\phi) = A_0 \cdot b \cdot \sin\left(\frac{\pi b \sin(\phi)}{\lambda}\right) \cdot \frac{\lambda}{\pi b \sin(\phi)}
\label{eq:amplitude_einzelspalt}
\end{equation}
%
 Da aufgrund der hohen Frequenz des Lichtes nicht die Amplitude direkt vermessen werden kann, muss die Intensitätsverteilung des gebeugten Lichtes betrachtet werden, welche sich proportional zum Quadrat der Amlitude verhält. Dadurch ergibt sich die Formel \eqref{eq:intensität_einzelspalt} für die Intensitätsvereiltung I der am Einzelspalt gebeugten Lichtstrahlen.
%
\begin{equation}
I(\phi) \propto A_{0}^2 b^2 \cdot \left(\frac{\lambda}{\pi b \sin(\phi)}\right)^2 \cdot \sin^2\left(\frac{\pi b \sin(\phi)}{\lambda}\right)
\label{eq:intensität_einzelspalt}
\end{equation}
%
Es handelt sich bei der Intensitätsverteilung um die auszumessende Beugungsfigur. Diese besitzt unendlich viele Minima und Maxima , wobei die Amplitude dieser mit steigendem Winkel gegen Null geht.
%
\subsection{Beugung am Doppelspalt}
Die Beugung am Doppelspalt wird analog zur Beugung am Einzelspalt behandelt. Hierbei wird angenommen, dass beide Spaltbreiten gleich sind. Die Entfernung der beiden Spalte voneinander plus die Spaltbreite wird mit s bezeichnet. Dadurch ergibt sich für die Winkelabhängige Intensitätsverteilung bei der Beugung einer ebenen Welle am Doppelspalt die in Formel \eqref{eq:intensität_doppelspalt} beschriebene Beziehung.
%
\begin{equation}
I(\phi) \propto 4 \cdot \cos^2\left(\frac{\pi \cdot s \cdot sin(\phi)}{\lambda}\right) \cdot \left(\frac{\lambda}{\pi b \sin(\phi)}\right)^2 \cdot \sin^2\left(\frac{\pi b \sin(\phi)}{\lambda}\right) 
\label{eq:intensität_doppelspalt}
\end{equation}
%
Es sind also bei der entstehenden Beugungsfigur bei der Beugung am Doppelspalt zusätzlich zu den Minima der Einzelspalte noch weitere Minima einer $\cos^2$-Verteilung zu finden. 
%
\subsection{Beugungsfunktion durch Fouriertransformation}
%
Die Amplitudenverteilung g($\xi$) ist mit der in Formel \eqref{eq:fourier} Fourier-Transformation der Aperturfunktion f(x) zu erhalten.
\begin{equation}
g(\xi) := \int\limits_{- \infty}^{\infty} f(x) \cdot e^{ix\xi} \d x
\label{eq:fourier}
\end{equation}
%
Bei Betrachtung der Aperturfunktion eines Einzelspaltes mit Spaltbreite b, welche durch eine konstante Amplitude $A_0$ beschrieben wird, erhält man mit der angegebenen Fouriertransformation den Ausdruck aus Formel \eqref{eq:amplitude_fourier}. 
%
\begin{equation}
g(\xi) = \frac{2 A_0}{\xi} \cdot \exp\left(\frac{i \xi b}{2}\right) \cdot \sin\left(\frac{\xi b}{2}\right)
\label{eq:amplitude_fourier}
\end{equation}
%
Für $\xi = \frac{2 \pi \sin(\phi)}{\lambda}$ entsteht die für die Amplitudenverteilung bei Beugung am Einzelspalt gefundene Gleichung \eqref{eq:amplitude_einzelspalt}.
%  
\section{Auswertung}

\begin{figure}
  \centering
  \includegraphics{geometrie}
  \caption{Geometrie der Apparatur}
  \label{fig:apparatur-geometrie}
\end{figure}

\section{Einzelspalt}
Bei der Messung am Einzelspalt, kommt ein Einzelspalt mit einer Weite
von ca. \SI{0.08}{\milli\metre} zum Einsatz. Die gewonnenen Meßwerte
werden nun in eine nichtlineare Regression gegeben\footnote{hierzu wird
  die Funktion \texttt{scipy.optimize.curve\_fit} aus der
  \texttt{scipy}-Bibliothek in der Version 0.11 verwendet}. Aus
Formel~\eqref{eq:intensität_einzelspalt}, die die Intensität beschreibt,
und der Abbildung~\ref{fig:apparatur-geometrie} ergibt sich dieser
Zusammenhang zwischen der gemessenen Stromstärke $I$ und der Strecke $x$
%
\begin{equation}
  \label{eq:strom-einzelspalt}
  I(x) \propto A_0\frac{L\lambda}{\pi(x-c)} \sin\left(\frac{\pi b}{\lambda
        L} (x-c)\right)
\end{equation}
%
Die Werte für die Parameter führen auf folgenden Funktionsterm:
%
\begin{equation}
  \label{eq:einzelspalt_fit}
  I(x) = \SI[exponent-to-prefix = true]{8.828e-5}{\ampere\per\metre}
\end{equation}
%
In Abbildung~\ref{fig:einzelspalt} sind die Meßwerte und der Graph der
erhaltenen Funktion dargestell.

\section{Variabler Spalt}

\section{Doppelspalt}

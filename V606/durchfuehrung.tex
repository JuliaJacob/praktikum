% This work is licensed under the Creative Commons
% Attribution-NonCommercial 3.0 Unported License.  To view a copy of
% this license, visit http://creativecommons.org/licenses/by-nc/3.0/.

\section{Aufbau und Durchführung}

\subsection{Aufbau einer Apparatur zur Bestimmung der Suszeptibilität}
\label{sec:aufbau-apparatur}

Die Apparatur besteht im wesentlichen aus vier Teilen: Einem
Funktionengenerator als Signalgeber, einer Brückenschaltung, einem
Verstärker und einem Meßgerät.  Der Signalgeber speist die
Brückenschaltung mit einem Sinussignal, ein Verstärker leitet das Signal
verstärkt an das Meßgerät weiter.  Ein Schema der Versuchsanordnung ist
in \cref{fig:schema-aufbau} skizziert.  Das Prinzip, nach dem die
Suszeptibilität bestimmt wird, basiert auf einer Induktivitätsmessung
einer Spule, in welche die zu messende Probe eingeführt wird.  Der
Verstärker ist notwendig, da die auftretenden Spannungen sehr gering
sind und von Störspannungen, die in dieser Größenordnung immer
auftreten, überlagert sind.  Aus diesem Grunde ist zusätzlich ein Filter
in den Verstärker integriert, da das zu messende Signal monofrequent,
die Störsignale aber ein Frequenz-Spektrum haben können, das viele
verschiedene Frequenzen enthält.

\subsubsection{Beschreibung der Brückenschaltung}

Die Brücke zur Induktivitätsmessung ist in \cref{fig:bruecke}
skizziert.  In das Innere der Spule, die mit $L_\text{M}$ bezeichnet ist,
kann die zu untersuchende Probe eingeführt werden.  Durch das Einführen
einer Probe mit Querschnitt~$Q$ und magnetischer Suszeptibilität~$\chi$
in eine Spule der Länge~$l$ und Windungszahl~$n$ ist der Unterschied
zwischen der Induktivität~$L_\text{M}$ und $L$ um
\begin{equation}
  \Delta L = \mu_0 \chi Q \frac{n^2}{l}.
\end{equation}
Der Zusammenhang zwischen Brückenspannung~$U_\text{Br}$ und magnetischer
Suszeptibilität~$\chi$ ergibt sich aus den Abgleichbedingungen für die
Brücke.  Es gilt:
\begin{equation}
  \label{eq:chi-spannungen}
  \chi = \frac{U_\text{Br}}{U_\text{Sp}} \frac{4 l}{\omega \mu_0 n^2 Q}
  \sqrt{R^2 + \omega^2 \left(\mu_0 \frac{n^2}{l} F\right)^2},
\end{equation}
wobei hier $F$ die Querschnittsfläche der Spule und $U_\text{Sp}$ die
Speisespannung bezeichnet.  Wird nun in die ohne eingelegte Probe
abgeglichene Brücke die Probe eingeführt, dann ergibt sich ebenfalls aus
den Abgleichbedingungen der Brücke, daß der Widerstand~$R_3$ sich um
\begin{equation}
  \label{eq:delta-r}
  \Delta R = \chi \frac{R_3 Q}{2 F}.
\end{equation}

\subsubsection{Unterdrückung von Störspannungen}

Wie bereits erwähnt treten an den Ausgangsklemmen der Brückenschaltung
Störspannungen auf.  Da die zu messende Brückenspannung vermutlich
völlig von den Störungen verdeckt wird, muß ein Weg gefunden werden,
diese herauszufiltern.  Man wählt dazu als Speisespannung ein
Sinussignal und verwendet einen Selektivverstärker, der nur Spannungen
im Bereich der Frequenz dieses Signals herum passieren läßt.  Da die
Störspannungen viele verschiedene Frequenzen enthalten, kann ein
Großteil der Störungen herausgefiltert werden.  Die Güte
\begin{equation}
  \label{eq:guete}
  Q = \frac{\nu_0}{\nu_+ - \nu_-}
\end{equation}
ist ein Maß für die Schärfe des Bereichs, in dem die ungewollten
Frequenzen abgeschnitten werden.

\subsection{Messung der Güte des Selektivverstärkers}

Zu allererst wird eine Güte-Messung des verwendeten Selektivverstärkers
durchgeführt.  Dazu wird ein Sinussignal vom Funktionengenerator in den
Eingang des Verstärkers gegeben und am Ausgang von einem Meßgerät wieder
abgenommen.  Jetzt wird die Frequenz des Sinussignals verändert und so
eine Meßreihe Spannung gegen Frequenz durchgeführt.  Ziel ist es,
herauszufinden, wie gut die Filtereigenschaften des Selektivverstärkers
sind.  Die Meßwerte werden um den Bereich der Resonanzspitze dichter
gewählt, um eine bessere Auflösung gewährleisten zu können.

\subsection{Messung der Suszeptibilität der Proben}

Wie bereits in \cref{sec:aufbau-apparatur} angerissen, wird die
Suszeptibilität der Proben über eine Induktivitätsmessung realisiert.
Die Induktivitäten werden mithilfe einer Brückenschaltung und der
Nullmethode bestimmt, um eine gute Genauigkeit zu erreichen.  Da die
auftretenden Signale allerdings in der Größenordnung der Störungen
liegt, ist ein sogenannter Selektivverstärker notwendig, der das Signal
verstärkt und nur eine Frequenz durchläßt.

Es gibt zwei Möglichkeiten, mit der oben beschriebenen Apparatur die
Suszeptibilität zu bestimmen.  Wird bei ausgeglichener Brücke,
d.\,h. die Brückenspannung verschwindet, die Probe in die Spule
geschoben, so läßt sich aus der Änderung der Brückenspannung die
Änderung der Induktivität berechnen.  Die zweite Möglichkeit ist nun,
die Brücke mit eingelegter Probe auszugleichen und daraus die Änderung
der Induktivität abzulesen.  Hier werden beide Möglichkeiten
nacheinander ausgeführt, damit je zwei Meßwerte pro Messung erhalten
werden.

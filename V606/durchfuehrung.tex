% This work is licensed under the Creative Commons
% Attribution-NonCommercial 3.0 Unported License. To view a copy of this
% license, visit http://creativecommons.org/licenses/by-nc/3.0/.

\section{Aufbau und Durchführung}

\subsection{Aufbau einer Apparatur zur Bestimmung der Suszeptibilität}
\label{sec:aufbau-apparatur}

Die Apparatur besteht im wesentlichen aus vier Teilen: Einem
Funktionengenerator als Signalgeber, einer Brückenschaltung, einem
Verstärker und einem Meßgerät. Der Signalgeber speist die
Brückenschaltung mit einem Sinussignal, ein Verstärker leitet das Signal
verstärkt an das Meßgerät weiter. Ein Schema der Versuchsanordnung ist
in \cref{fig:schema-aufbau} skizziert. Das Prinzip, nach dem die
Suszeptibilität bestimmt wird, basiert auf einer Induktivitätsmessung
einer Spule, in welche die zu messende Probe eingeführt wird. Der
Verstärker ist notwendig, da die auftretenden Spannungen sehr gering
sind und von Störspannungen, die in dieser Größenordnung immer
auftreten, überlagert sind. Aus diesem Grunde ist zusätzlich ein Filter
in den Verstärker integriert, da das zu messende Signal monofrequent,
die Störsignale aber ein Frequenz-Spektrum haben können, das viele
verschiedene Frequenzen enthält.

\subsection{Messung der Güte des Selektivverstärkers}

Zu allererst wird eine Güte-Messung des verwendeten Selektivverstärkers
durchgeführt. Dazu wird ein Sinussignal vom Funktionengenerator in den
Eingang des Verstärkers gegeben und am Ausgang von einem Meßgerät wieder
abgenommen. Jetzt wird die Frequenz des Sinussignals verändert und so
eine Meßreihe Spannung gegen Frequenz durchgeführt. Ziel ist es,
herauszufinden, wie gut die Filtereigenschaften des Selektivverstärkers
sind. Die Meßwerte werden um den Bereich der Resonanzspitze dichter
gewählt, um eine bessere Auflösung gewährleisten zu können.

\subsection{Messung der Suszeptibilität der Proben}

Wie bereits in \cref{sec:aufbau-apparatur} angerissen, wird die
Suszeptibilität der Proben über eine Induktivitätsmessung
realisiert. Die Induktivitäten werden mithilfe einer Brückenschaltung
und der Nullmethode bestimmt, um eine gute Genauigkeit zu erreichen. Da
die auftretenden Signale allerdings in der Größenordnung der Störungen
liegt, ist ein sogenannter Selektivverstärker notwendig, der das Signal
verstärkt und nur eine Frequenz durchläßt.

Es gibt zwei Möglichkeiten, mit der oben beschriebenen Apparatur die
Suszeptibilität zu bestimmen. Wird bei ausgeglichener Brücke, d.\,h. die
Brückenspannung verschwindet, die Probe in die Spule geschoben, so läßt
sich aus der Änderung der Brückenspannung die Änderung der Induktivität
berechnen. Die zweite Möglichkeit ist nun, die Brücke mit eingelegter
Probe auszugleichen und daraus die Änderung der Induktivität abzulesen.
Hier werden beide Möglichkeiten nacheinander ausgeführt, damit je zwei
Meßwerte pro Messung erhalten werden.

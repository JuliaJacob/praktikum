% This work is licensed under the Creative Commons
% Attribution-NonCommercial 3.0 Unported License.  To view a copy of
% this license, visit http://creativecommons.org/licenses/by-nc/3.0/.

\section{Theorie}

\subsection{Magnetische Suszeptibilität und Paramagnetismus}

Zur Beschreibung des magnetischen Feldes im Vakuum wird die magnetische
Flußdichte~$\vec{B}$ verwendet.  Sie erfüllt die
\name{Maxwell}-Gleichungen für das Vakuum.  Werden nun Felder in Materie
studiert, ist es zweckmäßig eine neue Größe, die magnetische
Feldstärke~$\vec{H}$, so einzuführen:
\begin{equation}
  \vec{H} = \frac{1}{\mu_0}(\vec{B} - \vec{M}).
\end{equation}
Hierbei ist $\vec{M}$ die sogenannte Magnetisierung, die von der Materie
abhängig ist.  Im wesentlichen ist sie ein Mittelwert der im Material
vorkommenden magnetischen Momente~$\mu$.  Durch Einführung der
Größe~$\vec{H}$ haben die \name{Maxwell}-Gleichungen in Materie wieder
dieselbe Form wie im Vakuum.  Es gilt:
\begin{equation}
  \vec{M} = \mu_0 \chi \vec{H}.
\end{equation}
Die Größe~$\chi$ heißt Suszeptibilität und ist keine Konstante, sondern
von Temperatur~$T$ und magnetischer Feldstärke~$\vec{H}$ abhängig.

Ab einer magnetischen Suszeptibilität~$\chi>0$ spricht man von
Paramagnetismus.  Dieser tritt nur bei Materie auf, deren Bausteine (wie
z.\,B. Atome, Moleküle) einen Drehimpuls aufweisen.  Damit ist eine
Ausrichtung der mit dem Drehimpuls verknüpften magnetischen Momente in
einem äußeren Magnetfeld verbunden.  Diese Momente sind gegeben durch
\begin{align}
  \vec\mu_L &= -\frac{\mu_B}{\hbar} \vec{L} &
  \vec\mu_S &= -g_S\frac{y\mu_B}{\hbar} \vec{S},
\end{align}
wobei hier $\mu_B$ das sogenannte \name{Bohr}sche Magneton (das
magnetische Moment zum Drehimpuls $\hbar$) und $g_S$ das gyromagnetische
Verhältnis des freien Elektrons bezeichnen.  Die Ausrichtung dieser
Momente wird durch thermische Bewegungen gestört, so daß der
Paramagnetismus ein temperaturabhängiger Effekt ist.  Insbesondere für
hohe Temperaturen ist $\chi$ antiproportional zu $T$.  Dies nennt man
\name{Curie}sches Gesetz.

\subsection{Suszeptibilität der Verbindungen seltener Erden}

Die Verbindungen von Elementen, die Seltene Erden genannt werden, weisen
einen starken Paramagnetismus auf.  Daraus ergibt sich, daß die
Elektronen in diesen Atomen große Drehimpulse aufweisen müssen, der von
den inneren Elektronen ausgeht.  Die Elektronenkonfiguration und der
Gesamtdrehimpuls des Atoms kann mithilfe der
\name{Hund}schen\footnote{Friedrich Hund (1896--1997) deutscher
  Physiker. (nach \textcite{wikipedia:friedrich-hund}} Regeln bestimmt
werden. Diese lauten:

\begin{enumerate}
\item Die Anordnung der Spins ist durch das Pauli-Prinzip vorgegeben
\item Die Bahndrehimpulse ordnen sich so an, daß Regel~1 und das
  \name{Pauli}-Prinzip beachtet wird
\item Der Gesamtdrehimpuls ist gegeben durch $\vec{J} = \vec{L} -
  \vec{S}$, falls die Schale weniger als halb besetzt ist, und durch
  $\vec{J} = \vec{L} + \vec{S}$, falls die Schale mehr als halb besetzt
  ist.
\end{enumerate}

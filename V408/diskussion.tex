% This work is licensed under the Creative Commons
% Attribution-NonCommercial 3.0 Unported License. To view a copy of this
% license, visit http://creativecommons.org/licenses/by-nc/3.0/.

\section{Diskussion}
Nun sollen noch einmal die erhaltenen Ergebnisse dieses Versuches
diskutiert werden.  Die Bestimmung der Brennweiten der Linsen~1 und 2
durch Messung der Gegenstands-und Bildweiten liefert gute und genaue
Ergebnisse. In diesem Versuch wurde die Brennweite von Linse~1 zu
ca. $f_1 = \SI{48}{\milli\metre}$ bestimmt. Die Herstellerangabe für
diese Brennweite beträgt \SI{50}{\milli\metre}. Also gibt es eine
relative Abweichung von \SI{4}{\percent} zwischen in diesem Versuch
errechneten Wert für die Brennweite von Linse 1 und der entsprechenden
Herstellerangabe. In Abb.~\ref{fig:linsen1und2} erkennt man ebenfalls,
dass sich die Schnittpunkte aller Geraden ungefährt in dem Bereich
schneiden, in dem die Brennweite berechnet wurde.

Die Brennweitenbestimmung nach \name{Bessel} liefert ebenfalls nur
geringe statistische Fehler. Die Brennweiten der verschiedenen Farben
lassen sich im Rahmen dieser Messung nicht unterscheiden. Dies bedeutet,
dass die verwendete Linse so gut wie keine chromatische Aberration
zeigt. Der hier errechnete Wert für die Brennweite der verwendeten Linse
beträgt ca. $f = \SI{16.5}{\centi\metre}$. Der Hersteller der Linse hat
einen Wert von \SI{15}{\centi\metre} für die Brennweite angegeben. Also
ergibt sich hierbei eine relative Abweichung von \SI{10}{\percent}.

Bei der Bestimmung der Brennweite und Lage der Hauptebenen des
Linsensystems, welches als dicke Linse betrachtet wird, ergeben sich
riesige Fehler. der Fehler von h' liegt sogar bei über
\SI{100}{\percent}. Die Ausgleichsrechnungen ergeben ebenfalls für die
Brennweite des Linsensystems zwei deutlich unterschiedliche Werte.

Der Grund für den großen Fehler ist aber vermutlich nicht bei der
Bestimmungsmethode von \name{Abbe} zu finden, sondern eher darin, dass
das Linsensystem als dicke Linse genähert betrachtet wird.  Nach Formel
\eqref{eq:brechkraft} müsste die Brechkraft des Systems~$f = 0$
betragen, da eine Streu- und eine Sammellinse hintereinander gestellt
werden, welche dem Betrag nach gleiche Brennweiten besitzen. Die
Brennweite dieses Linsensystems müsste nach der Formel also gegen
unendlich gehen, das verwendete Linsensystem kann in diesem Sinne also
garnicht als optisches Abbildungselement verwendet werden.  Dass sich
trotzdem eine endliche Brennweite einstellt liegt daran, dass Formel
\eqref{eq:brechkraft} nur gilt, wenn der Abstand der beiden verwendeten
Linsen gegen Null geht. Da dies hier aber nicht der Fall ist, sondern
vielmehr der Abstand zwischen den beiden Linsen so vergrößert wurde,
dass man ein Bild auf dem Schirm erhält, wird die tatsächliche
Brennweite des Linsensystem endlich groß.  Der große Fehler in diesem
Versuchsteil rührt nun im wesentlichen daher, dass es große
Schwierigkeiten bereitet, ein scharfes Abbild des verwendeten
Gegenstandes auf dem Schirm zu erhalten. Vielmehr scheint das Bild nie
wirklich scharf zu werden, sodass versucht wurde, das Bild immer so
scharf wie möglich zu stellen.

% This work is licensed under the Creative Commons
% Attribution-NonCommercial 3.0 Unported License. To view a copy of this
% license, visit http://creativecommons.org/licenses/by-nc/3.0/.

\section{Auswertung}
\subsection{Brennweitenbestimmung über Gegenstands- und Bildweite}
Die Messreihen zur Bestimmung der Brennweiten einer Linse mit bekannter Brennweite (Linse 1) und einer Linse mit unbekannter Brennweite (Linse 2) liefern die in Tabelle \ref{tab:linsen1} eingetragenen Messwerte. Hierbei bezeichnet g die Gegenstandsweite und b die Bildweite.

Aus den erhaltenen Messwerten werden die jeweiligen Brennweiten mit der Linsengleichung, welche in Formel \eqref{eq:linsengleichung} notiert ist, bestimmt.
Es ergibt sich, dass 
\begin{equation}
f_1 = \SI{48.4(1)}{\milli\metre}
\label{eq:f1}
\end{equation}
die Brennweite der ersten Linse ist, und dass
\begin{equation}
f_2 = \SI{102.3(1)}{\milli\metre}
\label{eq:f2}
\end{equation}
die Brennweite der zweiten Linse ist. 

Die Angegebenen Fehler sind die statistischen Fehler der Mittelwerte der Messwerte.
%
\begin{table}[h]
  \centering
  \begin{tabular}{SSSS|SSSS}
     \toprule
\multicolumn{4}{c|}{Linse 1} & \multicolumn{4}{c}{Linse 2}\\
{g/}\si{\centi\metre}&{b/}\si{\centi\metre}&{g/}\si{\centi\metre}&{b/}\si{\centi\metre}&{g/}\si{\centi\metre}&{b/}\si{\centi\metre}&{g/}\si{\centi\metre}&{b/}\si{\centi\metre}\\
\toprule
15.1&7.1&16.6&6.9&13.0&44.4&25.7&17.3\\
22.0&6.3&13.0&8.0&15.1&30.5&28.4&15.9\\
8.2&13.1&22.4&6.1&18.0&23.2&31.1&15.5\\
18.7&6.5&30.7&5.9&19.8&20.6&35.0&14.7\\
13.3&7.3&37.9&5.1&22.2&19.1&38.1&14.4\\
\bottomrule
\end{tabular}
\caption{Gemessene Gegenstands- und Bildweiten der beiden verwendeten Linsen}
\label{tab:linsen1}
\end{table}
%
Um die Messgenauigkeit neben der angabe des statistischen Fehlers noch weiter zu untersuchen, werden die gemessenen Gegenstandsweiten auf die x-Achse und die Bildweiten auf die y-Achse eines kartesischen Koordinatensystems aufgetragen und die zugehörigen Messwerte ($g_i$/$b_i$) mit einer Geraden verbunden. Je genauer die Messung, desto mehr konzentrieren sich alle Schnittpunkte der eingezeichneten Geraden in einem Punkt.

In Abb. \ref{fig:linsen1und2} ist das Ergebnis dieses Vorgehens zu sehen. Die grünen Punkte markieren die aus \eqref{eq:f1} und \eqref{eq:f2} errechneten Brennweiten.
%
\begin{figure}[h]
\centering
\includegraphics[width=0.8\textwidth]{linsen1und2}
\caption{Zur Überprüfung der Messgenauigkeit}
\label{fig:linsen1und2}
\end{figure}
%
\subsection{Brennweitenbestimmung nach \name{Bessel}}
Das Messverfahren zur Bestimmung der Brennweite einer Linse nach \name{Bessel} ergibt für weißes, rotes und blaues Licht die in Tabelle \ref{tab:bessel} aufgeführten Messwerte. Mit e wird der Abstand zwischen dem Gegenstand und dem Bild bezeichnet. $g_1$ und $b_1$ geben die Gegenstands- und Bildweite an, bei der sich die Linse näher am Gegenstand befindet. $g_2$ und $b_2$ bezeichnen entsprechend den Fall, in dem sich die Linse näher am Schirm befindet.
%
\begin{table}[h!]
  \centering
  \begin{tabular}{SSSSS|SSSSS}
     \toprule
\multicolumn{10}{c}{weißes Licht}\\
{e/}\si{\centi\metre}&{$g_1$/}\si{\centi\metre}&{$b_1$/}\si{\centi\metre}&{$g_2$/}\si{\centi\metre}&{$b_2$/}\si{\centi\metre}&
{e/}\si{\centi\metre}&{$g_1$/}\si{\centi\metre}&{$b_1$/}\si{\centi\metre}&{$g_2$/}\si{\centi\metre}&{$b_2$/}\si{\centi\metre}\\
\toprule
70.0&26.2&43.8&44.2&25.8&85.0&22.1&62.9&63.1&21.9\\
73.0&24.6&48.4&48.6&24.4&88.0&21.9&66.1&65.2&22.8\\
76.0&23.8&52.2&52.2&23.8&91.0&23.5&69.5&70.0&21.0\\
79.0&23.2&55.8&56.1&22.9&94.0&21.3&72.7&73.1&20.9\\
82.0&22.6&59.4&59.6&22.4&97.0&21.1&75.9&76.3&20.7\\
\midrule
\multicolumn{5}{c}{rotes Licht}&\multicolumn{5}{c}{blaues Licht}\\
{e/}\si{\centi\metre}&{$g_1$/}\si{\centi\metre}&{$b_1$/}\si{\centi\metre}&{$g_2$/}\si{\centi\metre}&{$b_2$/}\si{\centi\metre}&
{e/}\si{\centi\metre}&{$g_1$/}\si{\centi\metre}&{$b_1$/}\si{\centi\metre}&{$g_2$/}\si{\centi\metre}&{$b_2$/}\si{\centi\metre}\\
\toprule
70.0&26.1&43.9&43.6&26.4&70.0&25.8&44.2&44.4&25.6\\
73.0&25.1&47.9&48.6&24.4&73.0&24.8&48.2&48.5&24.5\\
76.0&24.1&51.9&52.1&23.9&76.0&24.0&52.0&52.5&23.5\\
79.0&23.5&55.5&55.7&23.3&79.0&23.2&55.8&56.2&22.8\\
82.0&22.7&59.3&59.6&22.4&82.0&23.0&59.0&59.6&22.4\\
\bottomrule
\end{tabular}
\caption{Gemessene Gegenstands- und Bildweiten einer Linse für verschiedene Farben mit der Methode nach \name{Bessel}}
\label{tab:bessel}
\end{table}
%

Aus der Linsengleichung \eqref{eq:linsengleichung} ergibt sich durch Einsetzen der Abstände e = $g_1 + b_1 = g_2 + b_2$ und d = $|g_1 - b_1| = g_2 - b_2$ der in Formel \eqref{eq:bessel} widergegebene Zusammenhang zwischen Brennweite f der Linse und den Abständen e und d.
%
\begin{equation}
f = \frac{e^2 - d^2}{4e}
\label{eq:bessel}
\end{equation}
%
Mit Formel \eqref{eq:bessel} werden nun die Brennweiten $f_w$ für weißes, $f_r$ für rotes und $f_b$ für blaues Licht errechnet.
Es ergibt sich, dass
%
\begin{equation*}
f _w = \SI{16.5(1)}{\centi\metre} \text{ ; } f _r = \SI{16.4(1)}{\centi\metre} \text{ ; } f _b = \SI{16.4(1)}{\centi\metre}.
\end{equation*}
%
Die angebenenen Fehler sind wieder die statistischen Fehler der Mittelwerte.
%
\subsection{Brennweitenbestimmung nach \name{Abbe}}
Die Messreihe zur Bestimmung der Brennweite und Lage der Hauptebenen des Linsensystems, welches aus einer Zerstreuungslinse mit f = \SI{-100}{\centi\metre} und einer Sammellinse mit f = \SI{100}{\centi\metre} besteht, ergibt die in Tabelle \ref{tab:abbe} aufgeführten Messwerte. Außerdem ist in dieser Tabelle das Abbildungsverhältnis V angegeben.
g' und b' sind hierbei die Gegenstands- und Bildweiten bezüglich des gewählten Referenzpunktes A. B gibt die gemessene Bildgröße an. Die Gegenstandsgröße bemisst sich zu G = \SI{3.8}{\centi\metre}.
%
\begin{table}[h]
  \centering
  \begin{tabular}{SSSS|SSSS}
     \toprule
{g'/}\si{\centi\metre}&{b'/}\si{\centi\metre}&{B/}\si{\centi\metre}&{V}&{g'/}\si{\centi\metre}&{b'/}\si{\centi\metre}&{B/}\si{\centi\metre}&{V}\\
\toprule
54.5&55.9&3&0.79&61.3&50.0&2.0&0.53\\
50.8&57.4&2.8&0.74&47.4&61.5&3.1&0.82\\
43.7&65.4&3.7&0.97&62.5&49.5&1.9&0.5\\
50.8&55.7&2.7&0.71&57.5&52.3&2.2&0.58\\
52.7&59.0&2.2&0.58&54.4&55.5&2.5&0.66\\
\bottomrule
\end{tabular}
\caption{Gemessene Gegenstands- und Bildweiten bezüglich des gewählten Referenzpunktes A}
\label{tab:abbe}
\end{table}
%

Hierbei gelten die Formeln \eqref{eq:gs} und \eqref{eq:bs}. Mit h wird der Abstand zwischen Referenzpunkt A und gegenstandsseitiger Hauptebene H bezeichnet, mit h' entsprechend der Abstand zwischen dem Referenzpunkt und der bildseitigen Hauptebene H'.
\begin{equation}
g' = g+h = f \cdot \left(1 + \frac{1}{V}\right) + h
\label{eq:gs}
\end{equation}
\begin{equation}
b' = b+h' = f \cdot \left(1 +V\right) + h'
\label{eq:bs}
\end{equation}
Es werden die Gegenstandsweiten g' gegen $\left(1 + \frac{1}{V}\right)$, sowie die Bildweiten b' gegen $1+ V$ aufgetragen. Eine lineare Ausgleichsrechnung\footnote{Dazu wird \texttt{ipython} in der Version 0.13 verwendet} ergibt somit die Brennweite des Linsensystems, sowie die Lage der Hauptebenen. 
In Abb. \ref{fig:abbe} ist ein Plot der Messwerte gegen die entsprechenden Werte zu finden. Ebenfalls in diesem Plot sind die beiden Ausgleichsgeraden zu sehen.
%
\begin{figure}[h!!]
\centering
\includegraphics[width=0.8\textwidth]{abbe}
\caption{Zur Überprüfung der Messgenauigkeit}
\label{fig:abbe}
\end{figure}
%

Als Ergebnis der linearen Ausgleichsrechnung ergeben sich zwei verschiedene Brennweiten $f_g$ und $f_b$; eine aus der Messung der Gegenstandsweite und eine aus der Messung der Bildweite. 
Die gesuchten Werte mit den entsprechenden Fehlern werden bestimmt zu
\begin{equation*}
f_g = \SI{16.9(27)}{\centi\metre} \text{ ; } f_b = \SI{29.0(59)}{\centi\metre} \text{ ; } h = \SI{11.1(70)}{\centi\metre} \text{ ; } h' = \SI{7.3(100)}{\centi\metre}.
\end{equation*}
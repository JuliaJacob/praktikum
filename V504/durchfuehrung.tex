% This work is licensed under the Creative Commons 
% Attribution-NonCommercial 3.0 Unported License. To view a copy of this 
% license, visit http://creativecommons.org/licenses/by-nc/3.0/.

\section{Durchführung}
%
Als erstes wird der Verlauf der Kennlinie für positive Beschleunigungsspannungen bei fünf verschiedenen Kathodentemperaturen untersucht. Dafür wird der in Abb.~\ref{fig:apparaur} zu sehende Versuchsaufbau verwendet. Die in diesem Versuch verwendete Diode trägt die Aufschrift "Diode 1". Die verwendete Spannungsquelle besitzt ein internes Amperemeter.

Die Heizstromstärken werden von \SI{2.1}{\ampere} bis \SI{2.5}{\ampere} eingestellt. Pro eingestellter Temperatur werden zwischen 20 und 27 Stromstärkemessungen durchgeführt, wobei die Beschleunigungsspannung jedesmal in einem Bereich von \SI{10}{\volt} bis \SI{260}{\volt} liegen.

Im zweiten Teil dieses Versuches wird die Kennlinie der verwendeten Hochvakuumdiode für negative Spannungen, also für Gegenspannungen, untersucht.
Hierbei wird der Versuchsaufbau nur geringfügig verändert. Anstelle der Spannungsquelle mit integriertem Amperemeter wird ein exakteres Konstantspannungsgerät verwendet, welches Spannungen zwischen 0 und \SI{1}{\volt} liefert. Außerdem wird dazu in Reihe ein Nanoamperemeter geschaltet, um die nun deutlich geringere Stromstärke zu messen.

Bei der maximalen Heizleistung, welche bei \SI{2.5}{\ampere} Heizstrom liegt, werden für 10 Bremsspannungen zwischen 0 und \SI{1}{\volt} die Stromstärken mit dem Nanoamperemeter gemessen.
\begin{figure}
\centering
\includegraphics[width=0.6\textwidth]{apparatur}
\caption{Verwendeter Versuchsaufbau zur Untersuchung der Kennlinie einer Hochvakuumdiode. Der Schaltplan aus \textcite{v504} wurde leicht verändert, damit dieser mit der tatsächlich verwendeten Apparatur übereinstimmt.}
\label{fig:apparatur}
\end{figure}
%
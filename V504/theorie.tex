% This work is licensed under the Creative Commons
% Attribution-NonCommercial 3.0 Unported License. To view a copy of this
% license, visit http://creativecommons.org/licenses/by-nc/3.0/.

\section{Theorie}
%
\subsection{Leitungselektronen im Metall}
%
Metalle bestehen aus Atomen, von denen fast alle ionisert sind, und
welche ein Kristallgitter bilden. Die von den Atomen freigegebenen
Elektronen bewegen sich zwischen den Ionenrümpfen.  Das Gitterpotential,
welches durch diese Ionenrümpfe erzeugt wird, kann in einer schlichten
Näherung als konstant angesehen werden, sodass sich das Modell des
Potentialtopfes für das Atomgitter ergibt. Die Elektronen müssen in
diesem Modell eine Energie besitzen, die größer ist als die Tiefe des
Potentialtopfes multipliziert mit der Elementarladung.

Die Quantenmechanik liefert Aussagen über die Energie der
Leitungselektronen.  Elektronen sind Fermionen, besitzen also einen
halbzahligen Spin. Daher greift für Elektronen das \name{Pauli}-Prinzip,
welches besagt, dass es keine zwei Elektronen im Metallverband gibt, die
sich im selben Zustand befinden. Da es zwei mögliche Spins,
$\frac{1}{2}$ und $-\frac{1}{2}$, für Elektronen gibt bedeutet dies,
dass es pro Energie nur zwei Elektronen gibt, die diese Energie
besitzen.

Des weiteren folgt aus der Quantenmechanik bei der Betrachtung von
Elektronen im Potentialtopf, dass diese Elektronen nur diskrete Energien
besitzen können. Die größtmögliche Energie, die ein Elektron im Metall
am absoluten Nullpunkt haben kann, wird als \name{Fermi}sche
Grenzenergie $\zeta$ bezeichnet.

Die Wahrscheinlichkeit, ein Elektron mit der Energie $E$ im Metall mit
der Temperatur $T$ im thermischen GLeichgewicht zu finden, wird durch
die \name{Fermi}-\name{Dirac}-Verteilung beschrieben, welche in
Formel~\eqref{eq:fermi_dirac} wiedergegeben ist.
\begin{equation}
f(E) = \frac{1}{\exp{\left(\frac{E - \zeta}{kT}\right)}+1}
\label{eq:fermi_dirac}
\end{equation}
%
\subsection{Kennlinie einer Hochvakuum-Diode}
%
In diesem Versuch wird eine Hochvakuum-Diode verwendet, um die
Austrittsarbeit von Wolfram zu bestimmen. In Abb.~\ref{fig:diode} ist
ein Schaltbild zu sehen, in welcher der grundsätzliche Aufbau dieses
Versuchs skizziert ist.  Das zu untersuchende Metall dient als Kathode
und wird erhitzt. Die aus dem Metall gelösten Elektronen werden durch
eine Beschleunigungsspannung zur Anode hin beschleunigt. Der fließende
Strom wird gemessen. Die Beschleunigungsspannung aufgetragen gegen die
Stromsärke wird als Kennlinie der Hochvakuumdiode bezeichnet.  In
Abb.~\ref{fig:diode} ist eine solche Kennlinie zu sehen.  Das
Zustandekommen der verschiedenen Gebiete wird in den nachfolgenden
Sektionen erklärt.
%
\begin{figure}[]
\centering
\includegraphics[height = 3 cm]{diode}
\hspace{7 mm}
\includegraphics[height = 4 cm]{kennlinie}
\caption{Links zu sehen ist das Schaltbild einer Hochvakuumdiode. Rechts
  befindet sich der theoretische Verlauf der Kennlinie einer
  Hochvakuumdiode. Entnommen aus \textcite{v504}}
\label{fig:diode}
\end{figure}
%

\subsubsection{Das Anlaufstromgebiet}
%
Das Anlaufstromgebiet befindet sich bei der Kennlinie bei negativer
Beschleunigungsspannung. Dies bedeutet also, dass auch bei einer
angelegten Gegenspannung Elektronen bis zur Anode gelangen.

Dies ist damit zu erklären, dass es nach der
\name{Fermi}-\name{Dirac}-Verteilung in~\eqref{eq:fermi_dirac}
Elektronen im Metall gibt, die bei einer Temperatur $T > 0$ eine
Energiebesitzen, mit der sie das Metall verlassen können und noch eine
kinetische Energie außerhalb der Metalls besitzen, mit welcher die
Elektronen gegen die Gegenspannung anlaufen können. Daher rührt auch der
Name des Anlaufgebietes.

Außerdem besitzt das Anodenmaterial eine größere Austrittsarbeit, sodass
sich durch die elektrische Verbindung der Kathode mit der Anode
außerhalb der Diode bereits ein Gegenfeld bei nicht eingestellter
Spannung bildet. Für große Energien verhält sich die
Verteilung~\eqref{eq:fermi_dirac} exponentiel fallend, sodass sich die
Kennlinie im Bereich des Anlaufstromgebietes exponentiell steigend
verhalten muss.

Die Stromdichte im Anlaufstromgebiet ergibt sich zu dem in
Formel~\eqref{eq:anlaufstromdichte} wiedergegeben Ausdruck.
\begin{equation}
\text{const}\exp{\left(-\frac{e_0V}{kT}\right)}
\label{eq:anlaufstromdichte}
\end{equation}
%
\subsubsection{Das Raumladungsgebiet}
%
Da die Elektronen im elektrischen Feld der Diode gleichmäßig
beschleunigt werden und innerhalb der Diode nicht abgebremst werden,
gilt hierbei nicht das \name{Ohm}sche Gesetz, sodass sich auch kein
linearer Zusammenhang zwischen Spannung und Stromstärke einstellt.

Um den Verlauf der Kennlinie im Raumladungsgebiet, also für positive
Spannungen, welche noch nicht in den Sättigungsbereich gehen, zu
erhalten, werden die Kontinuitätsgleichung und die Poissongleichung
verwendet.

Das Ergebnis dieser Rechnung wird als
\name{Langmuir}-\name{Schottky}sche Raumladungsgleichung bezeichnet und
ist in Formel~\eqref{eq:raumladung} wiedergegeben. Dabei bezeichnet $j$
die Stromdichte, $V$ die angelegte Spannung und $a$ den Abstand zwischen
Kathode und Anode.
\begin{equation}
j = \frac{4}{9} \epsilon_0 \sqrt{2e_0/m_0} \frac{V^{\frac{3}{2}}}{a^2}
\label{eq:raumladung}
\end{equation}
%
\subsubsection{Das Sättigungsstromgebiet}
%
Das Sättigungsstromgebiet beginnt ab der Spannung, ab der ein Wendepunkt
in der Kennlinie auftritt. Dieses Gebiet kommt dadurch zustande, dass
bei einer festen Temperatur $T$ der Kathode pro Zeit und Fläche eine
endliche Menge an Elektronen aus der Metalloberfläche
austreten. Folglich kann also pro Zeit und Fläche nur maximal diese
Anzahl von Elektronen an der Anode ankommen.

Um eine Formel für die Anzahl der austretenden Elektronen pro Zeit und
Fläche aus der Kathode, also derSättigungsstromdichte $j_S$, zu
erhalten, wird die
\name{Fermi}-\name{Dirac}-Verteilung~\eqref{eq:fermi_dirac} für große
Energien der Elektronen betrachten.  Anschließend werden alle Elektronen
pro Zeit und Fläche gezählt, die eine eine Geschwindigkeitskomponente
senkrecht zur Metalloberfläche besitzen, welche ausreicht, um das Metall
zu verlassen. Multipliziert man die erhaltene Anzahl mit der
Elementarladung $e_0$, ergibt sich die gesuchte Formel, welche
\name{Richardson}-Gleichung genannt wird und in
Formel~\eqref{eq:richard} angegeben ist. $T$ ist die Temperatur, $k$ die
\name{Boltzmann}-Konstante und $\phi$ das Austrittspotential.
%
\begin{equation}
j_S(T) = 4\pi\frac{e_0m_0k^2}{h^3}T^3\exp{\left(\frac{-e_0\phi}{kT}\right)}
\label{eq:richard}
\end{equation}
%

% This work is licensed under the Creative Commons
% Attribution-NonCommercial 3.0 Unported License. To view a copy of this
% license, visit http://creativecommons.org/licenses/by-nc/3.0/.

\section{Versuchsaufbau}

Da verschiedene Eigenschaften der Mikrowellen und des Klystron
untersucht werden, ist der Aufbau dieses Versuchs flexibel.  Je nach
Versuchsteil müssen verschiedene Elemente aufgebaut werden.  Der
grundlegende Aufbau bleibt aber derselbe und ist in \cref{fig:aufbau}
dargestellt.

Links steht das Klystron, das für die Erzeugung der Mikrowellen
zuständig ist, dicht gefolgt von einem Isolator, dem Frequenzzähler,
einem Dämpfungsglied und einem Detektor am Ende. Je nach Versuchsteil
befinden sich zwischen Detektor und Dämpfungsglied verschiedene andere
Bauteile oder der Detektor entfällt ganz und wird durch verschiedene
Abschlüsse ersetzt.  Zum Darstellen des Signals, das vom Detektor oder
verschiedenen Sonden aufgenommen wird, werden zwei Meßgeräte verwendet:
Ein sogenanntes \emph{Standing Wave Ratio (SWR)} Meter und ein
Oszilloskop.

\section{Durchführung}

\subsection{Erster Versuchsteil: Inbetriebnahme und Modenmessung}

In diesem Teil des Versuchs geht es darum, das Klystron in Betrieb zu
nehmen und verschiedene Moden zu untersuchen.  Hinter das Dämpfungsglied
wird die Detektorvorrichtung montiert und an das SWR-Meter
angeschlossen.  Zur Messung der Ausgangsleistung mit einem SWR-Meter muß
das Klystron amplitudenmoduliert werden.  Wir verwenden dazu eine
Rechteckmodulation.  Für zwei verschiedene Reflektorspannungen wird der
Ausschlag am Meßgerät betrachtet.  Danach wird mit dem Frequenzmesser
die Frequenz gemessen.  Dazu stimmt man den Frequenzmesser ab, bis man
im Auschlag des SWR-Meters einen Rückgang sieht.  Die Stellung des
Frequenzmessers im Minimum dieses Rückgangs gibt die Schwingungsfrequenz
des Klystrons an.  Am Ende wird das Klystron auf eine bestimmte Frequenz
abgestimmt.  Dazu betätigt man den Abstimmknopf am Klystron und folgt
mit Reflektorspannung und Frequenzmesser.

Jetzt wird das SWR-Meßgerät durch ein Oszilloskop ersetzt.  Die
Horizontalablenkung des Oszilloskops wird jetzt durch ein Sinus-Signal,
das vom Klystron-Netzgerät kommt gespeist.  Die Vertikalablenkung wird
gleichspannungsgekoppelt mit dem Detektor verbunden.  Jetzt müssen die
Feinheit der Vertikalablenkung und die Reflektorspannung so eingestellt
werden, daß eine Modenkurve zu sehen ist.  Nachdem das Klystron auf
\SI{9}{GHz} abgestimmt worden ist und das Maximum der Modenkurve auf
Mittellinie des Oszilloskops gebracht und die Vertikalempfindlichkeit
und/oder Dämpfung so eingestellt worden ist, daß man volle vertikale
Auslenkung erhält, wird mit der Vermessung von drei Moden begonnen.  Das
Vorgehen ist dabei wie folgt:
%
\begin{enumerate}
\item Ablesen der Reflektorspannung $V_0$, Amplitude $A_0$ des Modus und
  Frequenz $f_0$ der Modenspitze.
\item Ändern der Reflektorspannung bis rechte Flanke des Signals links
  von der Mittellinie liegt.  Reflektorspannung $V_1$ ablesen.
\item Ändern der Reflektorspannung bis linke Flanke des Signals rechts
  von der Mittellinie liegt.  Reflektorspannung $V_2$ ablesen.
\item Nächste Mode einstellen
\end{enumerate}

Im letzten Teil dieses Abschnitts wird noch auf die elektronische
Abstimmung eingegangen.  Dazu wird das Klystron auf \SI{9000}{MHz}
eingestimmt und der höchste Modus ausgewählt.  Mit dem Frequenzmesser
wird jetzt wieder eine Delle in der Modenkurve erzeugt.  Wenn sich diese
genau in der Mitte der Modenkurve befinded, wird Frequenz und
Reflektorspannung abgelesen.  Dann wird die Modenkurve mit Hilfe der
Reflektorspannung so verschoben wie in 1. bei der Modenvermessung und
die Delle in den Usprung des Koordinatenkreuzes auf dem Schirm des
Oszilloskops verschoben.  Wieder wird die Frequenz und die
Reflektorspannung abgelesen.  Ebenso wird wie in 2. verfahren, die Delle
entsprechend verschoben und Frequenz und Reflektorspannung abgelesen.

\subsection{%
  Zweiter Versuchsteil: Messung von Frequenz, Wellenlänge und Dämpfung}

Das Ziel in diesem Versuchsteil ist es, die Frequenz des Klystrons über
die Wellenlänge einer Welle in einer rechteckigen Hohlleitung zu
bestimmen und die Dämpfung zu messen.  Der Aufbau variert je nach dem,
was gemessen werden soll.  Er ist in \cref{fig:fwd-aufbau} dargestellt.

Zunächst wird die Frequenz zum späteren Vergleich direkt mit dem
Frequenzmesser bestimmt.  Dazu wird so vorgegangen wie in Teil 1 des
Versuchs.  Der Aufbau ist dieses Mal jedoch ein wenig anders: Nach dem
Dämpfungsglied befindet sich nun ein Stehwellendetektor, an den das
SWR-Meßgerät angeschlossen ist.  Dahinter befindet sich ein Abschluß zur
Leitungsanpassung.  Der Ausschlag im SWR-Meßgerät wird mithilfe des
Frequenzmessers auf ein Minimum eingeregelt.  Dann wird die Frequenz
abgelesen.

Für die Wellenlängenmessung muß der Abschluß durch den einstellbaren
Kurzschluß ersetzt werden.  Falls der Frequenzmesser aus der vorherigen
Messung noch gestimmt ist, muß er nun verstimmt werden.  Der Kurzschluß
ist erforderlich, damit sich im Hohlleiter eine stehende Welle
ausbildet, dessen Wellenlänge mit dem Stehwellendetektor ausgemessen
werden kann.  Dazu werden mit der Sonde entlang der Meßleitung zwei Minima
auf dem SWR-Meter gesucht.  Der doppelte Abstand dieser Minima
entspricht der Hohlleitungswellenlänge.  Nachdem die Innenabmessung der
Längsseite des Hohlleiters bestimmt worden ist, kann mithilfe
\cref{eq:freq-hohl} die Frequenz ausgerechnet werden.

Zum Schluß soll noch eine Dämpfungsmessung durchgeführt werden.  Diese
wird mit der sogennanten Methode der Leistungsverhältnisse
durchgeführt. Der einstellbare Kurzschluß wird wieder durch den Abschluß
ersetzt.  Dann wird das Klystron auf 9.0 GHz abgestimmt.  Dann muß durch
Einstellen der Verstärkung am SWR-Meter und/oder der Dämpfung ein
Vollausschlag im \SI{30}{dB}-Meßbereich erreicht werden.  Die
Einstellung der Mikrometerschraube und der Leistungspegel am SWR-Meter
werden notiert.  Daraufhin wird die Dämpfung nach und nach in 2 dB
Schritten gesteigert und die Mikrometereinstellung am Dämpfungsglied
notiert.  In der Auswertung werden die Ergebnisse mit der Eichkurve auf
dem Dämpfungsglied verglichen.

\subsection{%
  Dritter Versuchsteil: Stehwellen-Messungen}

Im letzten Versuchsteil geht es darum, sich mit den grundlegenden
Techniken der Welligkeitsmessung mit der Meßleitung und dem Gebrauch des
SWR-Meters vertraut zu machen.

Ein Signal in einer Übertragungsleitung läßt sich als Superposition von
einfallenden und rücklaufenden Welle schreiben.  Die rücklaufende Welle
entsteht durch Reflexion an Unstetigkeiten der Leitung oder an einer
nicht angepaßten Lastimpedanz. Der Spannungsreflexionskoeffizient $\rho$
ist durch das Verhältnis zwischen den elektrischen Feldstärken der
reflektierten und einfallenden Welle gegeben. Er ist daher eine komplexe
Zahl, d.\,h. von der Phasenlage und vom Ort entlang der
Übertragungsleitung abhängig.  Eine ideale Anpassung liegt vor, wenn
$|\rho = 0|$.  Das Stehwellenverhältnis (SWR) oder auch Welligkeit
genannt ist definiert als das Verhältnis zwischen maximaler und
minimaler Feldstärke auf der Leitung.  Bei idealer Anpassung gilt
$\text{SWR} = 1$.  Eine vollständige Fehlanpassung bedeutet, daß $|\rho|
= 1$, bzw. $\text{SWR} = \infty$.  Entsprechend gelten die Formeln
%
\begin{gather}
  \rho = \frac{E_1}{E_0},\\
  S = \frac{E_\text{max}}{E_\text{min}} = \frac{|E_0| + |E_1|}{|E_0| -
    |E_1|},\\
  |\rho| = \frac{S-1}{S+1},
\end{gather}
%
wobei $S$ die Welligkeit und $E_0,E_1$ die elektrische Feldstärke der
einlaufenden bzw. rücklaufenden Welle bezeichnet.

Der Versuchsaufbau ist ähnlich wie zuvor.  An das Dämpfungsglied wird
wieder der Stehwellendetektor angeschlossen, mit dem auch wieder das
SWR-Meßgerät verbunden wird.  Nach dem Stehwellendetektor wird dann ein
Gleitschraubentransformator angeschlossen, der mit einem Abschluß
beendet wird. Im Stehwellendetektor wird kleiner Teil des Feldes über
Antenne ausgekoppelt und detektiert.  Wir werden das SWR auf drei
verschiedene Arten messen: mit der direkten Methode, mit der
\SI{3}{dB}-Methode und mit der Abschwächer-Methode.

Bei der direkten Methode kann das SWR direkt am SWR-Meter abgelesen
werden. Das funktioniert aber nur, solange die Sonde noch nicht tief
genug eindringt, also Feld nicht merklich gestört wird.  Außerdem muß
die Gleichrichterdiode im Detektor im quadratischen Bereich arbeiten,
d.\,h. die Ausgangsspannung muß proportional zur Eingangsspannung
sein. Bei großen SWR muß die Sondentiefe allerdings erhöht werden und es
kommt zu Feldverzerrungen und/oder hohe Leistung am Detektor, so daß der
quadratische Bereich verlassen wird.

Eine Möglichkeit, die Wirkung der Sondenüberlastung zu überwinden,
besteht in der \SI{3}{dB}-Methode.  Hier wird der Abstand zwischen den
Punkten gemessen, an welchen die Detektor-Ausgangsspannung den doppelten
Wert des Minimums erreicht.  Das SWR wird dann durch
%
\begin{equation}
  S = \sqrt{ 1 + \frac{1}{\sin^2 \frac{\pi(d_1-d_2)}{\lambda_g}}}
  \approx = \frac{\lambda_g}{\pi(d_1-d_2)}
\end{equation}
bestimmt. Hier wird also der quadratische Bereich nicht verlassen.  Um
die Wirkung der Abweichungen vom quadratischen Verhalten des Detektors
einzudämmen, kann die Abschwächermethode verwendet werden: Mit dem
Dämpfungsglied wird das Signal des Maximums dem des Minimums
gleichgemacht.  Die Differenz der Einstellungen des Dämpfungsglied
ergibt das SWR gemäß
%
\begin{equation}
  A_2 - A_1 = 20 \log S.
\end{equation}

Bei der direkten Messung wird die Sondentiefe fest eingestellt.  Dann
wird die Sonde der Meßleitung in ein Maximum bewegt, die Verstärkung so
eingestellt, daß \num{1.0} angezeigt wird, und danach die Sonde in ein
Minimum bewegt.  Dann kann man das SWR auf der oberen Skala
ablesen. Dies wird für drei verschiedene Sondentiefen durchgeführt.

Bei der \SI{3}{dB}-Methode wird die Sonde der Meßleitung in ein Minimum
gefahren und die Verstärkung des SWR so eingestellt, daß \SI{3}{dB}
angezeigt wird.  Jetzt wird die Sonde nach links verschoben bis sich ein
Vollausschlag ergibt.  Die Sondenstellung $d_1$ ablesen.  Das gleiche
wird noch einmal gemacht, diesmal wird aber die Sonde nach rechts
verschoben.  Dann wird die Stellung $d_2$ abgelesen.

Auch bei der Abschwächermethode wird die Sonde in ein Minimum gefahren.
Dann wird das Dämpfungsglied auf $A_1 = \SI{20}{dB}$ und die Verstärkung
des SWR-Meters so eingestellt, daß sich ein Ausschlag von \SI{3}{dB}
ergibt.  Jetzt wird mit der Sonde entlang der Meßleitung gefahren und
mit der Einstellung des Dämpfungsgliedes gefolgt, so daß der Ausschlag
des SWR-Meters im Skalenbereich bleibt.  Auf diese Weise wird ein relatives
Minimum gesucht.  Die Dämpfungseinstellung $A_2$ wird notiert.

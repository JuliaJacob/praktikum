% This work is licensed under the Creative Commons
% Attribution-NonCommercial 3.0 Unported License.  To view a copy of
% this license, visit http://creativecommons.org/licenses/by-nc/3.0/.

\section{Theorie}

Eine Spannungsquelle ist ein elektrisches Gerät, das eine über die Zeit
konstante Leistung liefert.  Beispiele sind der Dynamo am Fahrrad, eine
Batterie oder ein Kraftwerk.  Kenngrößen einer Spannungsquelle sind
hierbei die Leerlaufspannung~$U_0$, die gemessen werden kann, wenn die
Spannungsquelle nicht durch einen Verbraucher belastet wird, und der
Innenwiderstand, der eingeführt werden muß, um das Verhalten der
Spannungsquelle bei Belastung, nämlich das Absinken der
Klemmenspannung~$U_K$ zu erklären.  Um eine Beziehung zwischen den
Größen Belastungswiderstand/-strom, Innenwiderstand, Leerlaufspannung
und Klemmenspannung herstellen zu können, wird das Induktionsgesetz
\eqref{eq:inductionsgesetz} aus den \name{Maxwell}-Gleichungen benötigt.
\begin{equation}
  \label{eq:inductionsgesetz}
  \oint_{\partial A} \vec{E} \: \d\vec{s} = \iint_A \frac{\partial
    \vec{B}}{\partial t} \: \d\vec{A}
\end{equation}
Bei Abwesenheit von magnetischen Feldänderungen kann dieses Gesetz in
eine einfachere Form, die sogenannte Maschenregel von \name{Kirchhoff}
gebracht werden.  Sie lautet dann:
\begin{equation}
  \label{eq:maschenregel}
  \sum_j U_0^{(j)} = \sum_k R_k I_k
\end{equation}

\begin{figure}
  \centering
  \includegraphics{spannungsquelle}
  \caption{Hier ist das Ersatzschaltbild einer realen Spannungsquelle
    mit einem Belastungswiderstand in Reihe geschaltet.  Der
    gestrichelte Kasten kennzeichnet das Ersatzschaltbild einer realen
    Spannungsquelle. \cite{v301}}
\end{figure}

In \cref{fig:ersatzschaltung} ist ein Ersatzschaltbild einer realen
Spannungsquelle zu sehen.  Die Maschenregel ergibt in diesem
Spezialfall:
\begin{align}
  U_0 = R_\text{i} I + R_\text{a} I\\
\intertext{oder}
  U_K = R_\text{a}I = U_0 - R_\text{i} I. \label{eq:klemme}
\end{align}
Hieraus ist unmittelbar ersichtlich, warum die Klemmenspannung mit
zunehmendem Belastungsstrom sinkt.  Zur Messung der Leerlaufspannung
wird ein hochohmiges Voltmeter verwendet, da somit der Term mit dem
Innenwiderstand vernachlässigt werden kann und 
\begin{equation}
  U_K \to U_0 \text{ für } R_\text{i} \to 0
\end{equation}
gilt.  Aus der Existenz eines Innenwiderstandes ergibt sich auch die
Notwendigkeit einer sogenannten Leistungsanpassung.  Der Spannungsquelle
kann nämlich nicht nur eine beschränkte Leistung entnommen werden,
sondern diese ist abhängig vom gewählten Belastungswiderstand~$R_a$.

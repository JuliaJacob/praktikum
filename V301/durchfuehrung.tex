% This work is licensed under the Creative Commons
% Attribution-NonCommercial 3.0 Unported License.  To view a copy of
% this license, visit http://creativecommons.org/licenses/by-nc/3.0/.

\section{Durchführung}

\subsection{Aufbau}

In diesem Versuch werden drei verschiedene Spannungsquellen untersucht.
Eine Monozelle, eine Rechteckspannung und eine Sinusspannung, die beide
von einem Funktionengenerator geliefert werden.  Dazu wird ein
hochohmiges Voltmeter und ein Amperemeter zur Messung der Spannungen
sowie Belastungsströme verwendet.  Der Aufbau der Schaltungen ist in
\cref{fig:schaltungsbild} zu erkennen.

\begin{figure}
  \centering
  \includegraphics{schaltungsbild}
  \caption{Hier ist der Aufbau der verwendeten Schaltung skizziert.
    Einmal wird die Monozelle ohne Gegenspannung und einmal mit
    Gegenspannung vermessen. \cite{v301}}
  \label{fig:schaltungsbild}
\end{figure}

\subsection{Messung}

Es werden drei verschiedene Spannungsquellen mit verschiedenen Methoden
untersucht.  Zunächst wird die Leerlaufspannung mit einem hochohmigen
Voltmeter direkt gemessen, wobei der Eingangswiderstand dieses
Spannungsmeßgeräts notiert wird.  Danach wird die Klemmenspannung als
Funktion des Belastungsstroms aufgenommen, wobei der
Belastungswiderstand, mit dem der Strom gesteuert wird, im Bereich von
\SIrange{0}{50}{\ohm} variert.  Dies wird für die Sinus- und
Rechteckspannung ebenfalls durchgeführt.  Hierbei wird der
Belastungswiderstand für die Sinusspannung im Bereich
\SIrange{20}{250}{\ohm} und für die Rechteckspannung im Bereich
\SIrange{0.1}{5}{\kilo\ohm} variert.

Am Ende wird noch eine Gegenspannung (siehe \cref{fig:schaltungsbild})
an die Monozelle angelegt, die ca. \SI{2}{\volt} größer ist. Dadurch
fließt det Belastungsstrom in die andere Richtung und in der
Formel~\eqref{eq:klemme} wechselt das Vorzeichen, so daß sich
\begin{equation}
  \label{eq:klemme-gegenspannung}
  U_K = U_0 + I R_\text{i}
\end{equation}
ergibt.

% This work is licensed under the Creative Commons
% Attribution-NonCommercial 3.0 Unported License.  To view a copy of
% this license, visit http://creativecommons.org/licenses/by-nc/3.0/.

\section{Diskussion}
Zum Schluss sollen die Ergebnisse dieses Protokolls diskutiert werden.

Anhand der Abb.~\ref{fig:linregress} und den errechneten Fehlern der
Innenwiderstände und Leerlaufspannungen der drei untersuchten
Spannungsquellen ist zu erkennen, dass zum einen die Theorie stimmt
und zum anderen die verwendete Messmethode präzise ist.

Die weitere Untersuchung der Monozelle mit angelegtem Gegenfeld liefert 
ebenfalls gute und vor allem ähnliche Ergebnisse. 
Auch die direkte Messung der Leerlaufspannung ist zu empfehlen, da 
sich aufgrund des hohen Innenwiederstandes des Messgerätes ein
sehr geringer relativer Fehler von $\approx$ \SI{e-4}{\percent} 
ergibt.

Bei der Betrachtung der am Belastungswiderstand abfallenden Leistung 
ist beim Vergleich in Abb.~\ref{fig:leistung} zwischen Theoriekurve und 
experimenteller Kurve kaum ein Unterschied festzustellen. Die Theorie 
stimmt also gut mit der experimentellen Beobachtung überein. Lediglich 
im Grenzfall für $R_\text{a}$ gegen Null unterscheiden sich Theorie und 
experimentelle Beobachtung. Dieser Unterschied ist dadurch zu erklären, 
dass ein realer Widerstand niemals komplett auf Null fallen kann (es sei 
denn im Falle der Supraleitung, welche in diesem Versuch allerdings nicht 
realisiert wurde).

Ebenfalls gut zu erkennen ist, dass die abfallende Leistung bei $\approx$ 
\SI{15}{\ohm} maximal ist. Da dies ebenfalls ungefähr dem Innenwiderstand 
der Spannungsquelle entspricht, stimmt die theoretische Vorhersage, dass 
Leistungsanpassung bei Gleichstrom dann erreicht ist, wenn $R_\text{i}$ 
und $R_\text{a}$ übereinstimmen.

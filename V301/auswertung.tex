% This work is licensed under the Creative Commons
% Attribution-NonCommercial 3.0 Unported License.  To view a copy of
% this license, visit http://creativecommons.org/licenses/by-nc/3.0/.

\section{Auswertung}
\subsection{Leerlaufspannung und Innenwiderstand dreier Spannungsquellen}
Die zu den verschiedenen Belastungswiderständen $R_\text{a}$ gemessenen 
Klemmenspannungen $U_\text{K}$ und Stromstärken $I$ sind in 
Tabelle~\ref{tab:rechtsin} sowohl für eine Rechteck-, als auch für eine 
Sinusspannungsquelle eingetragen.
%
\begin{table}[h]
  \centering
  \begin{tabular}{SSS|SSS|SSS}
    \toprule
\multicolumn{3}{c|}{Monozelle}&\multicolumn{3}{c|}{Rechteckspannung}&
\multicolumn{3}{c}{Sinusspannung}\\
\midrule
$\frac{R_\text{A}}{\SI{50}{\ohm}}${/}\si{\percent}&{I/}\si{\ampere}&
$U_\text{K}${/}\si{\volt}&
$\frac{R_\text{A}}{\SI{250}{\ohm}}${/}\si{\percent}&{I/}\si{\milli\ampere}&
$U_\text{K}${/}\si{\milli\volt}&
$\frac{R_\text{A}}{\SI{5}{\kilo\ohm}}${/}\si{\percent}&{I/}\si{\milli\ampere}&
$U_\text{K}${/}\si{\milli\volt}\\
\midrule
0&0.084&0.22&0&7.10&195&0&1.7&480\\
10&0.065&0.37&10&6.10&250&10&1.5&610\\
20&0.053&0.59&20&4.50&320&20&0.95&950\\
30&0.043&0.74	&30&3.50&380&30&0.72&1100\\
50&0.032&0.95&50&2.60&460&50&0.48&1250\\
60&0.029&1.00&60&2.30&480&60&0.40&1300\\
70&0.026&1.05&70&2.05&490&70&0.33&1350\\
80&0.023&1.10&80&1.75&510&80&0.30&1400\\
90&0.021&1.15&90&1.70&520&90&0.26&1450\\
100&0.020&1.20&100&1.60&520&100&0.24&1450\\
\bottomrule
  \end{tabular}
  \caption{Gemessene Spannungen und Stromstärken für verschiedene 
Belastungswiderstände. Die Messung wurde für eine Monozelle, eine 
Rechteckspannung und eine Sinusspannung durchgeführt.}
  \label{tab:rechtsin}
\end{table}
%

Nun wird pro Spannungsquelle eine lineare Ausgleichrechnung\footnote{Dazu 
wurde \texttt{ipython} in der Version 0.13  verwendet} 
durchgeführt, 
wobei die gemessenen Klemmenspannungen gegen die 
Stromstärken betrachtet werden. In Abb.~\ref{fig:leerlauf} 
ist ein Plot der gemessenen Klemmenspannungen gegen die Stromstärken, 
sowie die errechneten Ausgleichsgeraden zu sehen.

\begin{figure}[]
\centering
\includegraphics[width=1.0\textwidth]{leerlauf.pdf}
\caption{Plot der errechneten Regressionsgeraden durch die gemessenen 
Klemmenspannungen gegen die Stromstärken. Im linken Subplot ist der Graph 
der Gleichspannung zu sehen. Im rechten Subplot befinden sich die Ausgleichsgeraden 
und Messwerte der Rechteck- und Sinusspannung.}
\label{fig:leerlauf}
\end{figure}

Die errechneten Geradengleichungen lauten 
\begin{align*}
U_\text{K} &= \SI{-15.7}{\ohm}\cdot I + \SI{1.46}{\volt}\text{ für die Monozelle,}\\
U_\text{K} &= \SI{-60.5}{\ohm}\cdot I + \SI{0.61}{\volt}\text{ für die Rechteckspannung,}\\
U_\text{K} &= \SI{-654.6}{\ohm}\cdot I + \SI{1.58}{\volt}\text{ für die Sinusspannung.}
\end{align*}
%
Aus diesen lassen sich nun mithilfe von Formel~\eqref{eq:klemme} die 
Innenwiderstände $R_\text{i}$ und Leerlaufspannungen $U_0$
der Spannungsquellen leicht ablesen. In Tabelle~\ref{tab:leerlaufergebnis} 
sind diese aus den Geradengleichungen extrahierten Werte, sowie dessen 
Fehler angegeben. Die Fehler sind die aus der 
linearen Ausgleichsrechnung bestimmten Fehler.
%
\begin{table}[]
  \centering
  \sisetup {
	table-number-alignment = center,
	table-figures-integer = 2,
	table-figures-decimal = 2
  }
  \begin{tabular}{S[table-figures-decimal = 1,table-figures-uncertainty = 1]
				    S[table-figures-decimal = 2,table-figures-uncertainty = 2]|
				    S[table-figures-decimal = 1,table-figures-uncertainty = 1]
				    S[table-figures-decimal = 2,table-figures-uncertainty = 2]|
				    S[table-figures-decimal = 1,table-figures-uncertainty = 1]
				    S[table-figures-decimal = 2,table-figures-uncertainty = 2]}
    \toprule
\multicolumn{2}{c|}{Monozelle}&\multicolumn{2}{c|}{Rechteckspannung}&
\multicolumn{2}{c}{Sinusspannung}\\
\midrule
$R_\text{i}${/}\si{\ohm}&$U_{0}${/}\si{\volt}&
$R_\text{i}${/}\si{\ohm}&$U_{0}${/}\si{\volt}&
$R_\text{i}${/}\si{\ohm}&$U_{0}${/}\si{\volt}\\
\midrule
15.7(7)&1.46(3)&60.5(22)&0.61(1)&654.6(130)&1.58(1)\\
\bottomrule
  \end{tabular}
  \caption{Gemessene Spannungen und Stromstärken für verschiedene 
Belastungswiderstände. Die Messung wurde für eine Monozelle, eine 
Rechteckspannung und eine Sinusspannung durchgeführt.}
  \label{tab:leerlaufergebnis}
\end{table}
%
\FloatBarrier
%
\subsection{Weitere Analyse der Monozelle}
%
Die einmalige direkte Leerlaufspannungsmessung mit einem hochohmigen 
Voltmeter mit einem Innenwiderstand von $R_\text{v} \ge$ \SI{10}{\mega\ohm} 
ergibt einen Wert von
\begin{equation}
U_{0} = \SI{1.5}{\volt}
\end{equation}
%
\subsubsection{Leerlaufspannung und Innenwiderstand mittels Gegenspannung}
Der bei angelegter Gegenspannung von \SI{3.5}{\volt} am Belastungswiderstand 
für verschiedene Widerstände $R_\text{a}$ gemessene Spannungsabfall 
$U_\text{K}$ und die gemessenen Stromstärken $I$ sind in 
Tabelle~\ref{tab:gegenspannung} aufgeführt.
%
\begin{table}[h]
  \centering
  \begin{tabular}{SSS|SSS}
    \toprule
$\frac{R_\text{A}}{\SI{50}{\ohm}}${/}\si{\percent}&{I/}\si{\milli\ampere}&
$U_\text{K}${/}\si{\volt}&
$\frac{R_\text{A}}{\SI{50}{\ohm}}${/}\si{\percent}&{I/}\si{\milli\ampere}&
$U_\text{K}${/}\si{\volt}\\
\midrule
0&110&3.5&60&40&2.2\\
10&93&3.0&70&36&2.1\\
20&76&2.7&80&32&2.1\\
30&61&2.5&90&29&2.0\\
40&53&2.4&100&27&1.95\\
50&46&2.3&&\\
\bottomrule
  \end{tabular}
  \caption{Gemessene Spannungen und Stromstärken für verschiedene 
Belastungswiderstände. Die Messung wurde mit einer Monozelle als 
Spannungsquelle durchgeführt, wobei eine Gegenspannung von 
\SI{3.5}{\volt} angelegt wurde.}
  \label{tab:gegenspannung}
\end{table}
%

Mit den in dieser Tabelle eingetragenen Messwerten wird eine lineare 
Ausgleichsrechnung durchgeführt. Es wird hierbei die Klemmenspannung 
gegen die Stromstärke betrachtet. Ein Plot der Messwerte und der 
Ausgleichsgeraden ist in Abb.~\ref{fig:gegenspannung} zu sehen.

\begin{figure}[]
\centering
\includegraphics[width=0.8\textwidth]{gegenspannung.pdf}
\caption{Plot der errechneten Regressionsgeraden durch die gemessenen 
Klemmenspannungen gegen die Stromstärken bei angelegter Gegenspannung 
von \SI{3.5}{\volt} an einer Monozelle.}
\label{fig:gegenspannung}
\end{figure}

Die lineare Ausgleichsrechnung ergibt die Geradengleichung
\begin{equation}
U_{K} = \SI{17.1}{\ohm}\cdot I + \SI{1.50}{\volt}
\end{equation}
Ein Vergleich mit Formel~\eqref{eq:klemme-gegenspannung} ergibt folgende 
Werte für die Leerlaufspannung $U_{0}$ und den 
Innenwiderstand $R_{i}$ der Monozelle:
\begin{align*}
R_{i}&=\SI{17.1(7)}{\ohm}\\
U_{0}&=\SI{1.50(4)}{\volt}
\end{align*}
Die hier angegebenen Fehler sind wieder die Fehler der linearen 
Ausgleichsrechnung.

Wird das arithmetische Mittel zwischen diesen 
Werten für die Leerlaufspannung und dem Innenwiderstand 
und denen aus Tabelle~\ref{tab:leerlaufergebnis} gebildet, 
ergibt sich 
\begin{align*}
R_{imon}&=\SI{16.4(7)}{\ohm}\\
U_{0mon}&=\SI{1.48(4)}{\volt}
\end{align*}
Die Fehler sind durch eine Gaußsche Fehlerfortpflanzung des arithmetischen 
Mittels errechnet worden, was einfach dem arithmetischen Mittel der 
Einzelfehler entspricht.


Der systematische Fehler, der bei der direkten Leerlaufspannung 
gemacht wird, beträgt in Abhängigkeit von $R_{i}$
\begin{equation*}
\Delta U_{0} = \frac{R_{i}\cdot U_{K}}{R_{v}}
\end{equation*}
Als Mittelwert der einzelnen systematischen Fehler ergibt sich ein 
Wert von 
\begin{equation*}
\Delta \overline{U_{0}} = \SI{1.4e-6}{\volt}.
\end{equation*}
%
\subsubsection{Leistungsabfall am Belastungswiderstand}
Die in Tabelle~\ref{tab:gegenspannung} eingetragenen Werte werden nun 
verwendet, um die am Belastungswiderstand verrichtete Leistung 
$N = U\cdot I$ zu berechnen. In Abb.~\ref{fig:leistung} ist diese Leistung 
gegen den Belastungswiederstand $R_{a}$ aufgetragen. 
Im selben Diagramm ist der nach Formel~\eqref{eq:leistungsformel} 
berechnete theoretische Leistungsabfallverlauf in Abhängigkeit von 
$R_\text{a}$ eingezeichnet.
\begin{equation}
\label{eq:leistungsformel}
N(R_\text{a})=\frac{(U_{0}-\frac{R_{i}U_{0}}{R_{i}+R_{a}})^2}{R_{a}}
\end{equation}
Es werden hierbei die in diesem Protokoll errechneten Werte für den 
Innenwiderstand $R_{i}$ und der Leerlaufspannung $U_{0}$ 
verwendet.

\begin{figure}[]
\centering
\includegraphics[width=0.8\textwidth]{leistung.pdf}
\caption{Hier zu sehen ist der theoretische Verlauf des Leistungsabfall 
am Belastungswiderstand, sowie der gemessene. Die Kurven ähneln sich 
stark}
\label{fig:leistung}
\end{figure}
\FloatBarrier
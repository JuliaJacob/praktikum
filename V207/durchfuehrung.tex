% This work is licensed under the Creative Commons
% Attribution-NonCommercial 3.0 Unported License. To view a copy of this
% license, visit http://creativecommons.org/licenses/by-nc/3.0/.

\section{Durchführung}
Als erstes wird die Ansprechzeit der Thermosäule überprüft. Dazu wird der \name{Leslie}-Würfel etwa zur Hälfte mit kochendem Wasser gefüllt, sodass dieser sich ebenfalls erwärmt. Zum Zeitpunkt t=0 wird der Eingang der Thermosäule nicht mehr durch schwarze Pappe verdeckt, sodass nun die abgestrahlte Wärmestrahlung des \name{Leslie}-Würfels von der Thermosäule detektiert werden kann.

Es wird eine Messreihe aufgenommen, in der alle Zehn Sekunden die Thermospannung abgelesen wird.
Anschließend wird eine weitere Messreihe gestartet, in der alle Drei Sekunden ein Messwert aufgenommen wird.


Um das \name{Stefan}-\name{Boltzmann} Gesetz zu überprüfen wird der Strahlungswürfel nach \name{Leslie} erneut mit kochendem Wasser gefüllt, diesmal wird der Würfel fast vollständig gefüllt.

Es wird pro Seite ein Messwert vom Voltmeter abgelesen. Ist der Würfel um einige Grade abgekühlt, wird das Vorgehen wiederholt bis zu einer Temperatur von ca. \SI{50}{\celsius}.

Die Analyse des Abstandsverhalten der Wärmeleistung wird ebenfalls mithilfe des \name{Leslie}-Würfels und der Thermosäule durchgeführt. Dazu wird zunächst bei ca. \SI{50}{\celsius} die Thermospannung für verschiedene Abstände der Thermosäule vom Würfel gemessen.

Diese Messreihe wird bei einer Temperatur des Würfels von ca. \SI{93}{\celsius} wiederholt.

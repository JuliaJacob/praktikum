\section{Auswertung}

\subsection{Bestimmung des Emissionsvermögens}

Aus den gemessenen Werten wird mithilfe einer linearen
Regressionsrechnung das jeweilige Emissionsvermögen bestimmt. Die schwarz
lackierte Oberfläche wird dabei als ein Schwarzer Strahler bestrachet,
d.\,h. es wird ein Emissionsvermögen $\epsilon = 1$ angenommen. Für die
anderen Oberflächen gilt
\begin{equation}
  P(T) = f(T^4) = \epsilon \sigma T^4
\end{equation}

Aus der Steigung der berechneten Geradengleichungen kann also das
Emissionsvermögen bestimmt werden.
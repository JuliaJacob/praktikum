% This work is licensed under the Creative Commons
% Attribution-NonCommercial 3.0 Unported License. To view a copy of this
% license, visit http://creativecommons.org/licenses/by-nc/3.0/.

% ==========================================================================
%                     Festlegung der Dokumentenklasse
% ==========================================================================

% Dokumentklasse für Aufsätze, Berichte, etc.
\documentclass[paper=a4, german, titlepage]{scrartcl}

% Behebt ein paar Fehler in Latex
\usepackage{fixltx2e}

% ==========================================================================
%                            Detailtypographie
% ==========================================================================

\usepackage{microtype}

% ==========================================================================
%                             Zeichenkodierung
% ==========================================================================

% UTF-8 als Eingabe-Kodierung und T1 als Fontkodierung
\usepackage[utf8]{inputenc}
\usepackage[T1]{fontenc}

% ==========================================================================
%                               Schriftarten
% ==========================================================================

\usepackage{lmodern}

% ==========================================================================
%                           Spracheinstellungen
% ==========================================================================

% Deutsche Zeichenketten und deutsche Namen für die Referenzobjeke
\usepackage[german]{babel, varioref}

% ==========================================================================
%                  Aufzählungen, Referenzen und Links
% ==========================================================================

\usepackage{enumitem}

% Verlinkungen innerhalb und außerhalb des PDF-Dokuments
\usepackage{hyperref}

% Formattiert URLs, so dass sie sich z.B. besser vom Text abheben
\usepackage{url}

% TrueType-Schrift für URLs		
% \urlstyle{tt}		

% ==========================================================================
%                        Bibliograhphie und Anhang
% ==========================================================================

\newcommand{\theappendix}{
  \clearpage
  \appendix
}

% Deutsche Guillemets mit \enquote{}
\usepackage[german=guillemets]{csquotes}

\usepackage[style=numeric-comp, backend=biber]{biblatex}
\bibliography{../literatur.bib}

% Nachnamen in Kapitälchen
\renewcommand*{\mkbibnamelast}[1]{\textsc{#1}}

% ==========================================================================
%                    Grafiken, Abbildungen und Tabellen
% ==========================================================================

% Verwenden von Farben und Grafiken
\usepackage{graphicx}
\usepackage{xcolor}

% Einbinden von ganzen PDF-Seiten
\usepackage{pdfpages}

% kleine Schrift für Bildunterschriften, Fettgedruckte Bildunterschriften
\usepackage[font=small,	labelfont=bf, format=plain]{caption}

% Für mehrere Objekte nebeneinander mit eigenen Bildunterschriften
\usepackage{subcaption}

% Text umläuft Fließobjekte
%\usepackage{wrapfig}

% Tabellensatz
% \usepackage{tabularx}
\usepackage{booktabs}
\usepackage{longtable}

% Zum Verdrehen von Objekten. Nur mäßig verwenden.
% \usepackage{rotating}

% Setzen des Pfades für eingebundene Bilder
% \graphicspath{{figs/}{bilder/}}

% ==========================================================================
%                    Mathematikumgebungen und Einheiten
% ==========================================================================

% Paket für mathematische Umgebungen und Funktionen
\usepackage[intlimits]{amsmath}

% Zusätzliche Mathematische Schriftarten
\usepackage{amsfonts}

% Zusätzliche Mathematische Symbole
\usepackage{amssymb}

% Zum Setzen Kommutativer Diagramme
% \usepackage{amscd}

% Textsatz in der Matheumgebung
\usepackage{amstext}

% Aufrechte griechische Buchstaben
%\usepackage{upgreek}


% Diagramme mit tikz und Gnuplot zeichnen
% \usepackage{tikz}
% \usepackage{tikz-qtree}
% \usepackage{gnuplot-lua-tikz}

% ==========================================================================
%               automatischer Satz von Einheiten mit SIUnitX
% ==========================================================================

\usepackage[
% Stellt den Fehler separat dar: Siehe SIUnitX-Manual
  separate-uncertainty = true,
]{siunitx}

% Babel stellt SIUnitX auf deutsch ein, wenn german gewählt wird
\addto\extrasgerman{\sisetup{locale = DE}}

% Kürzen von Einheiten in SIUnitX ermöglichen
% \usepackage{cancel}


% ==========================================================================
%                            Textsatzparameter
% ==========================================================================

% Vermeidung von "Schusterjungen" Höchstwert 10000, dann dürfen
% theoretisch keine Schusterjungen mehr auftreten.
\clubpenalty = 3000
% Vermeidkung von "Hurenkindern" Höchstwert 10000, dann dürfen
% theoretisch keine Hurenkinder mehr auftreten.  Es werden beide
% Einstellungen benötigt.
\widowpenalty = 3000
\displaywidowpenalty = 3000


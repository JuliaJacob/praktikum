% This work is licensed under the Creative Commons
% Attribution-NonCommercial 3.0 Unported License. To view a copy of this
% license, visit http://creativecommons.org/licenses/by-nc/3.0/.

\section{Diskussion}
Zum Abschluss soll das Ergebis dieses Protokolls bewertet werden.
Die Bestimmung der Durchmesser der Kugeln mithilfe der Schieblehre
erweist sich als äußerst ungenau, da sich während des Messvorganges
nicht feststellen lässt, ob der Abstand zwischen zwei diametrisch auf
der Kugeloberfläche liegenden Punkte vermessen wird.

Das Verwenden von Winkeln ist frei von dieser Störanfälligkeit. Das
Aufstellen einer Messreihe gibt ebenfalls mehr Sicherheit bei der
Bestimmung des Durchmessers.  Alle ermittelten Werte für die Viskosität
des verwendeten Wassers sind um ca. \SI{0.2}{\milli\pascal\second}
größer als die Literaturwerte, welche aus \textcite{demtroeder-1}
entnommen werden. Also ist entweder die Viskosität des untersuchten
Wassers größer als in der Literaturangabe, oder aber es handelt sich um
einen systematischen Fehler. Ebenfalls zu beachten ist, dass sich ab
einer Temperatur von \SI{60}{\celsius} Blasen innerhalb des Wassers
gebildet haben, welche die Strömungseigenschaften beeinflusst haben.

Da der Wert der Reynoldszahl bei Zimmertemperatur zwischen 29 und 96
liegt, ist destilliertes Wasser also als laminare Flüssigkeit zu
betrachten.

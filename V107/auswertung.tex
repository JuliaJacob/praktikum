% This work is licensed under the Creative Commons
% Attribution-NonCommercial 3.0 Unported License. To view a copy of this
% license, visit http://creativecommons.org/licenses/by-nc/3.0/.

\section{Auswertung}
\subsection{Dichtebestimmung der verwendeten Kugeln}
Die Gewichtsmessung ergibt für die kleine Kugel eine Masse von
$m_\text{kl}$ = \SI{4.45}{\gram} und für die große Kugel eine Masse von
$m_\text{gr}$ = \SI{4.95}{\gram}.

Die einmalige Vermessung der großen Kugel mit einer Schieblehre ergibt
einen Durchmesser von $D_\text{gr}$ = \SI{15.8}{\milli\metre}. Für die kleine
Kugel wird eine Messreihe mit Winkeln durchgeführt. Die Ergebnisse sind
in Tabelle \ref{tab:durchmesser_kugel_kl} zu finden.

%
\begin{table}[h]
  \centering
  \begin{tabular}{SS|SS}
    \toprule
    {Nr.}  & {D/}\si{\milli\metre} & {Nr.}  & {D/}\si{\milli\metre}\\
    \midrule
1&15.22 & 6 & 15.22\\
2&15.28 & 7 & 15.26\\
3&15.08 & 8 & 15.16\\
4&15.22 & 9 & 15.12\\
5&15.24 & 10 & 15.22\\
    \bottomrule
{Mittelwert}& \multicolumn{3}{S}{\SI{15.20}{\milli\metre}} \\
{Standardabweichung}&\multicolumn{3}{S}{\SI{0.19}{\milli\metre}}\\
\bottomrule
  \end{tabular}
  \caption{Messwerte für den Durchmesser der kleinen Kugel}
  \label{tab:durchmesser_kugel_kl}
\end{table}
%
Aus den gemessenen Größen lässt sich die Dichte $\rho$ der Kugeln mit Formel \eqref{eq:dichte} berechnen. Dabei bezeichnet m die Masse der Kugeln.
\begin{equation}
\label{eq:dichte}
\rho = \frac{6m}{ \pi D^3}
\end{equation}
 In Tabelle \ref{tab:dichten} sind die Ergebnisse der Dichtebestimmung der beiden Kugeln zu finden. Bei der kleinen Kugel kann aufgrund der durchgeführten Messreihe der absolute Fehler durch eine Gausssche Fehlerfortpflanzung von Formel \eqref{eq:dichte} angegeben werden, wobei der Durchmesser mit einem statistischen Fehler behaftet ist.
%
\begin{table}[h]
  \centering
  \begin{tabular}{SSS}
    \toprule
   {Kugel} & $\rho${ /(}\si{\gram\per\centi\metre^3}{)} &$\Delta\rho${ /(}\si{\gram\per\centi\metre^3}{)}\\
    \midrule
{klein}&2.42&3e-5  \\
{groß}&2.40&{-}  \\
    \bottomrule
  \end{tabular}
  \caption{Ermittelte Dichten der beiden Kugeln}
  \label{tab:dichten}
\end{table}
%
\subsection{Apparaturkonstante der großen Kugel}
%
Um die Apparaturkonstante K mithilfe von Formel \eqref{eq:empirie}
bestimmen zu können, wird zunächst die Viskosität des destillierten
Wassers bei Zimmertemperatur (\SI{20}{\celsius}) bestimmt. In Tabelle
\ref{tab:zeiten_const} sind die gemessenen Fallzeiten der kleinen und
großen Kugel einzusehen. Die Apparaturkonstante der kleinen Kugel
besitzt laut \textcite{v107} einen Wert von 
%
\begin{equation*}
K_{kl} =\SI{0.07640}{\milli\pascal\centi\metre^3\per\gram}. 
\end{equation*}
%
Der Wert für die Dichte des destillierten Wassers beträgt laut
\textcite{wissenschaft-technik-ethik}
%
\begin{equation*}
 \rho_w = \SI{0.99820}{\gram\per\centi\metre^3} \text{ (bei einer Temperatur von \SI{20}{\celsius})}.
\end{equation*} 
%
\begin{table}[h]
  \centering
  \begin{tabular}{S|S|S|S}
    \toprule
   \multicolumn{2}{c|}{kleine Kugel} & \multicolumn{2}{c}{große Kugel}\\
    \midrule
{Zeit t/s}&{Zeit t/s}&{Zeit t/s}&{Zeit t/s}\\
\midrule
11.92&12.06&79.1&78.95\\
11.93&12.06&78.61&78.95\\
12.12&11.96&78.32&78.10\\
12.32&12.16&79.03&79.06\\
12.13&11.95&78.16&78.10\\
11.84&11.95&78.40&78.83\\
12.07&11.92&78.16&78.00\\
12.21&12.16&78.89&78.84\\
12.20&12.04&77.64&78.20\\
12.10&12.26&78.40&78.72\\
    \bottomrule
  \end{tabular}
  \caption{Gemessene Fallzeiten der Kugeln bei Zimmertemperatur}
  \label{tab:zeiten_const}
\end{table}
%

Durch Verwenden von Formel \eqref{eq:empirie} kann mithilfe der
Messwerte für die Fallzeiten der kleinen Kugel die Viskosität des
destillierten Wassers bei Zimmertemperatur bestimmt werden. Als Ergebnis erhält man, dass 
\begin{equation*}
\eta(\SI{20}{\celsius}) = \SI{1.310(3)}{\milli\pascal\second}.
\end{equation*}
Der Fehler wird mithilfe einer Gaussschen Fehlerforpflanzung der
verwendeten Formel errechnet. Die gemessenen Zeiten sind hierbei durch
einen statistischen Fehler behaftet.
%

Aus den gewonnenen Daten wird nun die Apparaturkonstante $K_{gr}$ für
die große Kugel bestimmt.  Dazu wird ebenfalls Formel \eqref{eq:empirie}
verwendet. Für die Viskosität wird der zuvor bestimmte Wert für die
Viskosität des Wassers bei Zimmertemperatur verwendet.  Die Rechnung
ergibt einen Wert von
\begin{equation}
\label{eq:apparatur}
K_{gr} = \SI{0,011(4)}{\milli\pascal\centi\metre^3\per\gram}.
\end{equation}
Bei der Fehlerfortpflanzung sind diesmal sowohl die Fallzeit, als auch die Viskosität mit einem Fehler behaftet.
%
\subsection{Temperaturabhängigkeit der dynamischen Viskosität}
%
Die gemessenen Fallzeiten der großen Kugel bei 10 verschiedenen Temperaturen ist in Tabelle \ref{tab:zeiten_var} angegeben.

Mit Formel \eqref{eq:empirie} wird die Viskosität des Wassers bei den
verschiedenen Temperaturen bestimmt. Es wird in der Formel die oben
bestimmte Apparaturkonstante, welche unter \eqref{eq:apparatur} zu finden ist, für die große Kugel verwendet, sodass bei der Fehlerfortpflanzung sowohl die Apparaturkonstante, als auch die
Fallzeit fehlerbehaftet ist. Pro Temperatur wird der statistische Fehler
der Fallzeit aus den vier Messwerten bei dieser Temperatur bestimmt.

Die Ergebnisse dieser Rechnung sind in Tabelle \ref{tab:zeiten_var} nachzulesen. Die dort angegebenen Dichten von Wasser $\rho_\text{fl}$ bei den betrachteten Temperaturen sind aus \textcite{wissenschaft-technik-ethik} entnommen.
%
\begin{table}[h]
  \centering
  \sisetup {
	table-number-alignment = center,
	table-figures-integer = 2,
	table-figures-decimal = 2
  }
  \begin{tabular}
	{S|SSSS|
	S[table-figures-decimal = 5] |
	S[table-figures-decimal = 3, table-figures-uncertainty = 1]|S}
    \toprule
{T /}\si{\celsius}&\multicolumn{4}{S|}{{t/s}}&$\rho_\text{fl}${ /(}\si{\gram\per\centi\metre^3}{)}&$\eta${ /}\si{\milli\pascal\second}&{Re}\\
\midrule
25	&71.66&71.01&71.27&71.18&0.99704&1.190(2)&18.6\\
32	&65.12&63.00&64.45&64.47&0.99502&1.075(7)&22.8\\
35	&58.75&58.10&58.70&58.80&0.99403&0.980(3)&27.4\\
40	&54.04&54.13&54.33&52.61&0.99221&0.901(6)&32.5\\
45	&49.72&49.58&49.83&49.83&0.99021&0.835(1)&37.9\\
50	&46.00&45.55&45.69&45.78&0.98803&0.769(2)&44.7\\
55	&42.15&42.03&41.96&42.49&0.98569&0.710(2)&52.6\\
60	&39.55&39.69&39.52&39.38&0.98319&0.667(1)&59.6\\
65	&37.66&38.33&37.18&37.60&0.98055&0.637(4)&65.4\\
70	&36.10&36.24&35.44&35.94&0.97776&0.608(3)&71.9\\
    \bottomrule
  \end{tabular}
  \caption{Gemessene Fallzeiten der großen Kugel und errechnete Werte}
  \label{tab:zeiten_var}
\end{table}
%

Ein graphischer Plot der Ergebnisse ist in Abb. \ref{fig:viskoseplot} zu
sehen. Dabei wird $\log{(\eta)}$ gegen $\frac{1}{T}$ aufgetragen. Im
selben Plot ist ebenfalls die Ausgleichsgerade durch die Messwerte zu
finden.

Mit einer linearen Ausgleichsrechnung\footnote{Dazu wurde \texttt{ipython} in der Version 0.13  verwendet} werden die Koeffizienten in der Andradeschen Gleichung \eqref{eq:andrade} bestimmt. 
%
\begin{figure}[h!!]
\centering
\includegraphics[width=0.8\textwidth]{viskoseplot}
\caption{Plot der errechneten Regressionsgerade durch die aufgenommenen Messwerte}
\label{fig:viskoseplot}
\end{figure}
%

Die Ausgleichsrechnung ergibt,
\begin{equation*}
A = \SI{0.43(1)}{\milli\pascal\second}
\end{equation*}
\begin{equation*}
B = \SI{27.5(10)}{\celsius},
\end{equation*}
sodass sich die gefundene Temperaturabhängigkeit der Viskosität (in \SI{}{\celsius}) des untersuchten Wassers beschreiben lässt durch
\begin{equation*}
\eta(T) = 0,43 \cdot \exp{\left(\frac{\SI{27.5}{\celsius}}{T}\right)} \text{ }\si{\milli\pascal\second}.
\end{equation*}
%
Zuletzt wird noch die Reynoldsche Zahl Re des analysierten Wassers für verschiedene Temperaturen berechnet. Diese ist festgelegt als
\begin{equation*}
Re = \frac{v_m \cdot D \cdot \rho_{w}}{\eta},
\end{equation*}
mit $v_m$ $\hat{=}$ mittlere Geschwindigkeit der Kugel, D $\hat{=}$
Durchmesser der Kugel und $\rho_{w}$ $\hat{=}$ Dichte der durchströmten
Flüssigkeit. Bei Zimmertemperatur ergibt sich als Reynoldszahl für die
kleine Kugel ein Wert von $\text{Re}_{kl}$ = \SI{96} und für die große
Kugel ergibt sich, dass $\text{Re}_{gr}$ = \SI{29}.
Die berechnete Reynoldszahl für die große Kugel bei den in diesem Versuch verwendeten Temperaturen des Wassers ist in Tabelle \ref{tab:zeiten_var} eingetragen.
%
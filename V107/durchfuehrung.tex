
\section{Durchführung}

\subsection{Vorbereitung}

Zunächst werden die Dichten der Glaskugeln aus ihrer Masse und ihrem
Volumen bestimmt. Dann folgt die Dichte der zu untersuchenden
Flüssigkeit, d.\,h. des destillierten Wassers. Dies wird mit zwei
Meßmethoden vorgenommen: 1. mithilfe eines Aräometers und 2. mithilfe
einer \name{Mohr}-\name{Westphal}schen Waage. Die erhaltenen
Meßergebnisse werden verglichen.

Dann wird die Aparatur vorbereitet. Erst wird mit der Libelle, die sich
am Viskosimeter befindet, kontrolliert ob dieses gerade steht. Danach
wird das destillierte Wasser eingeführt, während darauf geachtet wird,
daß sich keine Luftblasen an der Rohrwand sammeln. Auch beim Einlegen
der Kugeln müssen Luftblasen vermieden werden. Nachdem die Aparatur mit
einer Schraube verschlossen worden ist, kann die Messung durchgeführt
werden.

\subsection{Messung}

Für die große und die kleine Kugel werden jeweils zehn Messungen bei
Raumtemperatur durchgeführt. Gemessen wird die Zeit, die die Kugel
benötigt um von der oberen Marke zur unteren zu gelangen. Das
Viskosimeter wird umgedreht, wenn die Kugel die untere Marke
überschritten hat, und die Messung wiederholt. Aus den gewonnenen
Meßwerten kann die Apparaturkonstante bestimmt werden.

Jetzt wird das Wasserbad langsam auf \SI{70}{\degreeCelsius} aufgeheizt
und für zehn verschiedene Temperaturen jeweils zweimal die Zeit, die die
große Kugel von Marke zu Marke benötigt, gemessen.

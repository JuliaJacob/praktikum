\section{Theorie}
\let\name\relax

Eine Kugel, die durch eine Flüssigkeit fällt, erfährt Reibung. Die Größe
der Reibung hängt von der Querschnittsfläche der Kugel und ihrer
Geschwindigkeit, aber auch von der dynamischen Viskosität der
Flüssigkeit ab. Ist die Bewegung der Kugel so gegeben, daß sich keine
Wirbel in der Flüssigkeit ausbilden, die Strömung also laminar ist, kann
man die Reibungskraft nach dem \name{Stokes}schen Gesetz angeben:
\begin{equation}
  \label{eq:stokes}
  \vec{F} = -6 \pi \eta r \vec{v}.
\end{equation}
Hierbei ist $v$ die Fallgeschwindigkeit der Kugel und $\eta$ bezeichnet
die dynamische Viskosität der Flüssigkeit.

Während des Falls wirken drei verschiedene Kräfte auf die Kugel:
Reibung, Auftrieb und Schwerkraft. Die beiden ersten sind der
Schwerkraft entgegengerichtet. Nach hinreichend langer Zeit stellt sich
ein Gleichgewicht ein und die Kugel bewegt sich mit konstanter
Geschwindigkeit.

Bei der Bestimmung der dynamischen Viskosität nach der Methode von
Herrn \name{Höppler} gleitet eine Kugel in einem Rohr, dessen
Durchmesser nur wenig größer als derjenige der Kugel ist, in einer
Flüssigkeit hinab. Um Verwirbelungen zu vermeiden, ist das Rohr leicht
geneigt, so daß die Kugel nicht zwischen den Wänden hin und her stößt.
Aus der Dichte von Kugel $\rho_\text{K}$ und Flüssigkeit
$\rho_\text{Fl}$ und der Fallzeit $t$ läßt sich nun mithilfe einer
empirisch gewonnenen Formel die Viskosität $\eta$ berechnen.
\begin{equation}
  \label{eq:empirie}
  \eta = K(\rho_\text{K} - \rho_\text{Fl})t
\end{equation}
Die Konstante $K$ wird auch als Apparaturkonstante bezeichnet und hängt
von der Fallhöhe und der Kugelgeometrie ab.

Die eben bestimmte dynamische Viskosität ist bei vielen Flüssigkeiten
temperaturabhängig. In vielen Fällen läßt sich das Verhalten durch die
sogenannte \name{Andrade}sche Gleichung beschreiben:
\begin{equation}
  \label{eq:andrade}
  \eta(T) = A \exp \left( \frac{B}{T} \right).
\end{equation}
Hier sind $A$ und $B$ Konstanten, die bestimmt werden müssen, und $T$
bezeichnet die Temperatur.

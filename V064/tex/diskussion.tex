% This work is licensed under the Creative Commons
% Attribution-NonCommercial 3.0 Unported License. To view a copy of this
% license, visit http://creativecommons.org/licenses/by-nc/3.0/.

\section{Diskussion}

\subsection{Bestimmung des Kontrasts}

Die Kontrastmessung liefert gute Ergebnisse.  Vor allem die Abhängigkeit
des Kontrasts vom Polarisationswinkel kann gut experimentell
nachgewiesen werden, wie an \cref{fig:contrast} schön zu erkennen ist.
Merkwürdig ist nur, daß die erhaltenen Kontrastwerte zum Teil negativ
sind, was allerdings an der Vorgehensweise zur Erstellung der Meßreihe
liegt, auf die bereits in der Durchführung eingegangen worden ist.  Die
bestimmten Werte zeigen dennoch gute Übereinstimmung mit der Theorie.

\subsection{Bestimmung des Brechungsindex' von Glas}

Die Bestimmung des Brechungsindex' von Glas erweist sich als
außerordentlich präzise.  Eine Meßreihe von 30 Werten weicht nur in
wenigen Fällen von einer Anzahl von 48 Minima bzw. Maxima ab, so daß die
geschätzte Standardabweichung des Mittelwerts sehr gering ausfällt.
Vergleicht man den gemessenen Wert mit einer Auswahl von Literaturwerten
(siehe \cref{tab:lit_ref_index_glass}), so zeigt sich, daß der bestimmte
Wert realistisch ist.

\begin{table}
  \centering
  \begin{tabular}{lS}
    \toprule
    Material & {Brechungsindex (bei \SI{589}{nm})} \\
    \midrule
    FK3 & 1.464 \\
    BK7 & 1.516 \\
    Quarzglas & 1.458 \\
    \bottomrule
  \end{tabular}
  \caption{Einige Literaturwerte zu den Brechungsindices verschiedener
    Gläser und durchsichtiger Stoffe entnommen aus
    \textcite[][223]{demtroeder-2}.}
  \label{tab:lit_ref_index_glass}
\end{table}

\subsection{Bestimmung der Brechungsindices von Luft und Kohlendioxid}

Die Messung der Brechungsindices der beiden Gase Luft und Kohlendioxid
zeigt ebenfalls gute Ergebnisse.  Die geschätzten Standardabweichungen
der Parameter fallen sehr gering aus (siehe \cref{tab:linreg}).  In
\cref{tab:lit_ref_index_gas} sind einige Werte für den Brechungsindex
von Luft angegeben.  Der von uns ermittelte Wert fügt sich gut ein.

Für den Wert von Kohlendioxid findet sich auf Wikipedia $n =
\num{1.0004493}$ bei \SI{0}{\degreeCelsius} und \SI{101.325}{kPa}.  Der
von uns bestimmte Wert liegt bei \num{1.000332}.  Die Abweichung läßt
sich dadurch erklären, daß nicht bei \SI{0}{\degreeCelsius} gemessen
wurde.

\begin{table}
  \centering
  \begin{tabular}{SS}
    \toprule
    {Wellenlänge in \si{nm}} & {Brechungsindex} \\
    \midrule
    300 & 1.0002915 \\
    400 & 1.0002825 \\
    500 & 1.0002790 \\
    600 & 1.0002770 \\
    700 & 1.0002758 \\
    800 & 1.0002750 \\
    900 & 1.0002745 \\
    \bottomrule
  \end{tabular}
  \caption{Brechungsindices von trockener Luft bei
    \SI{20}{\degreeCelsius} und \SI{1000}{\milli\bar}. Die Werte
    stammen aus \cite[][222]{demtroeder-2}.}
  \label{tab:lit_ref_index_gas}
\end{table}

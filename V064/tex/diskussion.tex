% This work is licensed under the Creative Commons
% Attribution-NonCommercial 3.0 Unported License. To view a copy of this
% license, visit http://creativecommons.org/licenses/by-nc/3.0/.

\section{Diskussion}

\subsection{Bestimmung des Kontrasts}

Die Kontrastmessung liefert gute Ergebnisse.  Vor allem die Abhängigkeit
des Kontrasts vom Polarisationswinkel kann gut experimentell
nachgewiesen werden, wie an \cref{fig:contrast} schön zu erkennen ist.
Merkwürdig ist nur, daß die erhaltenen Kontrastwerte zum Teil negativ
sind, was allerdings an der Vorgehensweise zur Erstellung der Meßreihe
liegt, auf die bereits in der Durchführung eingegangen worden ist.  Die
bestimmten Werte zeigen dennoch gute Übereinstimmung mit der Theorie.

\subsection{Bestimmung des Brechungsindex' von Glas}

Die Bestimmung des Brechungsindex' von Glas erweist sich als
außerordentlich präzise.  Eine Meßreihe von 30 Werten weicht nur in
wenigen Fällen von einer Anzahl von 48 Minima bzw. Maxima ab, so daß
die geschätzte Standardabweichung des Mittelwerts sehr gering ausfällt.
Vergleicht man den gemessenen Wert mit einer Auswahl von Literaturwerten

\begin{table}
  \centering
  \begin{tabular}{lS}
    \toprule
    Material & Brechungsindex (bei \SI{589}{nm}) \\
    \midrule
    FK3 & 1.464 \\
    BK7 & 1.516 \\
    Quarzglas & 1.458 \\
    \bottomrule
  \end{tabular}
  \caption{Einige Literaturwerte zu den Brechungsindices verschiedener
    Gläser und durchsichtiger Stoffe entnommen aus
    \textcite[][223]{demtröder-2}.}
  \label{tab:lit_ref_index_glass}
\end{table}

\subsection{Bestimmung der Brechungsindices von Luft und Kohlendioxid}

% This work is licensed under the Creative Commons
% Attribution-NonCommercial 3.0 Unported License. To view a copy of this
% license, visit http://creativecommons.org/licenses/by-nc/3.0/.

%\FloatBarrier
\section{Meßdaten}
%
\begin{table}
  \centering
  
  \begin{tabular}{SSSSSS}
    \toprule
    {$\phi/\si{\degree}$} & {$V_\text{max}/\si{\volt}$} &
    {$V_\text{min}/\si{\volt}$} &
    {$\phi/\si{\degree}$} & {$V_\text{max}/\si{\volt}$} &
    {$V_\text{min}/\si{\volt}$}\\
    \midrule
      0 & 2.22 & 1.97 &  95 & 0.04 & 0.04  \\
      5 & 2.32 & 1.50 & 100 & 0.04 & 0.08  \\
     10 & 2.84 & 1.25 & 105 & 0.05 & 0.13  \\
     15 & 3.20 & 0.98 & 110 & 0.08 & 0.28  \\
     20 & 3.14 & 0.74 & 115 & 0.09 & 0.37  \\
     25 & 2.90 & 0.46 & 120 & 0.08 & 0.74  \\
     30 & 2.66 & 0.29 & 125 & 0.07 & 1.07  \\
     35 & 2.60 & 0.16 & 130 & 0.06 & 1.37  \\
     40 & 2.48 & 0.10 & 135 & 0.06 & 1.64  \\
     45 & 2.40 & 0.06 & 140 & 0.09 & 1.80  \\
     50 & 1.76 & 0.05 & 145 & 0.15 & 2.30  \\
     55 & 1.32 & 0.06 & 150 & 0.28 & 2.20  \\
     60 & 1.13 & 0.08 & 155 & 0.46 & 2.50  \\
     65 & 0.96 & 0.10 & 160 & 0.67 & 2.39  \\
     70 & 0.55 & 0.11 & 165 & 0.97 & 2.40  \\
     75 & 0.40 & 0.13 & 170 & 1.27 & 2.24  \\
     80 & 0.19 & 0.09 & 175 & 1.65 & 2.12  \\
     85 & 0.10 & 0.06 & 180 & 2.05 & 1.78  \\
     90 & 0.05 & 0.04 \\
     \bottomrule
  \end{tabular}
  
  \caption{Meßwerte der Kontrastmessung.  Die Größe $\phi$ ist der am
    Polarisator eingestellte Winkel, die gemessenen Spannungen sind
    zur Intensität des Strahls proportional.}
  \label{tab:constrast_messwerte}
\end{table}

\begin{table}
  \centering
  \begin{tabular}{SS}
    \toprule
    {Anzahl der Max/Min} & {Brechungsindex} \\
    \midrule
    48 & 1.4979 \\
    48 & 1.4979 \\
    48 & 1.4979 \\
    49 & 1.5136 \\
    48 & 1.4979 \\
    48 & 1.4979 \\
    48 & 1.4979 \\
    48 & 1.4979 \\
    48 & 1.4979 \\
    48 & 1.4979 \\
    48 & 1.4979 \\
    48 & 1.4979 \\
    47 & 1.4825 \\
    48 & 1.4979 \\
    49 & 1.5136 \\
    48 & 1.4979 \\
    48 & 1.4979 \\
    47 & 1.4825 \\
    48 & 1.4979 \\
    48 & 1.4979 \\
    48 & 1.4979 \\
    48 & 1.4979 \\
    48 & 1.4979 \\
    48 & 1.4979 \\
    48 & 1.4979 \\
    48 & 1.4979 \\
    48 & 1.4979 \\
    48 & 1.4979 \\
    48 & 1.4979 \\
    48 & 1.4979 \\
    \midrule
    {Mittelwert} & 1.4979(58) \\
    \bottomrule
  \end{tabular}
  \caption{Meßwerte zur Bestimmung des Brechungsindex' von Glas.  Die
    Brechungsindices wurden gemäß Formel~\eqref{eq:ref_index_glass}
    ausgerechnet.}
  \label{tab:ref_index_glass}
\end{table}

\begin{table}\centering
    \begin{subtable}{0.3\textwidth}
    \begin{tabular}{SS}
      \toprule
      {$p/\si{\milli\bar}$} & {Anzahl}\\
      \midrule
      36 & 0   \\
      136 & 4  \\
      316 & 1  \\
      416 & 16 \\
      516 & 20 \\
      616 & 24 \\
      770 & 31 \\
      807 & 33 \\
      882 & 36 \\
      912 & 37 \\
      996 & 41 \\
      \bottomrule
    \end{tabular}
    \caption{Luft, 1. Meßreihe}
  \end{subtable}
  \quad
  \begin{subtable}{0.3\textwidth}
    \begin{tabular}{SS}
      \toprule
      {$p/\si{\milli\bar}$} & {Anzahl}\\
      \midrule
      36 & 0   \\
      136 & 4  \\
      236 & 8  \\
      336 & 12 \\
      436 & 16 \\
      536 & 21 \\
      650 & 25 \\
      736 & 29 \\
      911 & 36 \\
      996 & 40 \\
      \\
      \bottomrule
    \end{tabular}
    \caption{Luft, 2. Meßreihe}
  \end{subtable}
  \quad
  \begin{subtable}{0.3\textwidth}
    \begin{tabular}{SS}
      \toprule
      {$p/\si{\milli\bar}$} & {Anzahl}\\
      \midrule
      38 & 0 \\
      101 & 3 \\
      201 & 8 \\
      301 & 14 \\
      401 & 20 \\
      506 & 26 \\
      600 & 30 \\
      704 & 36 \\
      840 & 45 \\
      923 & 47 \\
      1444 & 77 \\
      \bottomrule
    \end{tabular}
    \caption{Kohlendioxid}
  \end{subtable}
  \caption{Die gemessenen Daten zur Bestimmung des Brechungsindex von Gasen.}
\label{tab:ref-index-gas}
\end{table}

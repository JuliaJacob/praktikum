% This work is licensed under the Creative Commons
% Attribution-NonCommercial 3.0 Unported License. To view a copy of this
% license, visit http://creativecommons.org/licenses/by-nc/3.0/.

\section{Auswertung}

\subsection{Messung des Kontrasts}

Zur Auswertung der Meßdaten, die sich bei Kontrastmessung ergeben haben,
benutzen wir eine nichtlineare Ausgleichsrechnung.%
\footnote{Die numerische Berechnung wird von der Funktion
  \texttt{scipy.optimize.curve\_fit} aus der Python-Bibliothek
  \texttt{scipy} in der Version 0.10.1 durchgeführt} Die zu messenden
Intensitäten wurden über entsprechend proportionale Spannungen bestimmt.
Wir nehmen einen Zusammenhang $U = f(\phi)$ zwischen $\phi$ und den
gemessenen Spannungen an, der durch die Funktion $f\colon [0,\pi] \to
\R$ mit
%
\begin{equation}
  \label{eq:contrast_reg}
  f(\phi) = A \pm A\sin(2 \phi + B) + C
\end{equation}
%
gegeben ist.  Im Fall der maximalen Intensität wird das positive
Vorzeichen gewählt, im anderen Fall das negative.  Diese Funktion ergibt
sich wegen \eqref{eq:intens_contrast_max} und
\eqref{eq:intens_contrast_min} aus dem Theorieteil.  Die Ergebnisse der
Ausgleichsrechnung sind in \cref{tab:contrast} und \cref{fig:contrast}
dargestellt.

\begin{figure}
  \centering
  \includegraphics[width=0.75\textwidth]{figures/contrast}
  \caption{Dargestellt sind die Meßwerte und der geschätzte funktionale
    Zusammenhang gemäß \cref{eq:contrast_reg}.  Im oberen Plot sind die
    Meßwerte der maximalen Intensität und im unteren Plot diejenigen der
    minimalen Intensität}
  \label{fig:contrast}
\end{figure}

\begin{table}
  \centering
  \begin{tabular}{lSSSSSS}
    \toprule
    & 
    {$A/\si{\volt}$} & {$\frac{\sigma_A}{A}$} &
    {$B$} & {$\frac{\sigma_B}{B}$} &
    {$C/\si{\volt}$} & {$\frac{\sigma_C}{C}$} \\
    \midrule
    max. Intens. &
    1.4694 & 0.056 & 0.7559 & 0.0748 & -0.37826 & 0.2725 \\
    min. Intens. &
    1.1867 & 0.0588 & -0.8667 & 0.0681 & -0.3009 & 0.2879\\
    \bottomrule
  \end{tabular}
  \caption{Ergebnisse der Ausgleichsrechnung.  Die Größen $\sigma_i$
    bezeichnen die Standardabweichungen der geschätzten Parameter.  Die
    Phase $B$ ist im Bogenmaß angegeben.}
  \label{tab:contrast}
\end{table}

\subsection{Messung der Brechzahl von Glas}

Die ermittelte Anzahl der Minima und Maxima werden gemäß
\cref{eq:ref_index_glass} in die entsprechenden Brechindices
umgerechnet.  Hierbei werden für die Größen $T, \lambda_\text{vac},
\theta_{1, 2}$ die Werte aus \cref{tab:conf_vals} eingesetzt.  Da diese
Messung mehrfach durchgeführt worden ist, werden danach das
arithmetische Mittel und die Standardabweichung vom Mittelwert der
enhaltenen Brechindices errechnet.  Es ergibt sich:
%
\begin{equation}
  n = \num{1.4979(0058)}
\end{equation}
%
\begin{table}
  \centering
  \begin{tabular}{SSSS}
    \toprule
    {$T/\si{mm}$} & {$\lambda_\text{vac}/\si{nm}$} & {$\theta_1$} &
    {$\theta_2$}\\
    \midrule
    \\
    \bottomrule
  \end{tabular}
  \caption{Hier sind die Kenngrößen der Apparatur, die in
    \cref{eq:ref_index_glass} eingehen, gelistet.  Die Winkel $\theta_1,
    \theta_2$ sind im Bogenmaß angegeben.}
  \label{tab:conf_vals}
\end{table}
\subsection{Messung der Brechzahl von Gasen}


% This work is licensed under the Creative Commons
% Attribution-NonCommercial 3.0 Unported License.  To view a copy of
% this license, visit http://creativecommons.org/licenses/by-nc/3.0/.

\section{Auswertung}

\subsection{Charakteristik des \name{Geiger}-\name{Müller}-Zählrohrs}

Bei der Aufnahme der Charakteristik des \name{Geiger}-\name{Müller}-
Zählrohrs ist die Probe nach dem achten Meßwert losgeschraubt und damit
in ihrem Abstand vom Zählrohr verstellt worden. Dies ist in
\cref{fig:Charakteristik} zu erkennen. Die blaue Kurve zeigt den
eigentlichen Verlauf der Meßpunkte, die rote Kurve zeigt einen Verlauf,
der um 300 Impulse nach unten verschoben ist. Mit den sich so ergebenen
Meßwerten wird eine lineare Ausgleichsrechnung im Bereich des Plateaus
durchgeführt. Die durch Verschiebung korrigierten Werte sind in
\cref{tab:charakteristik} gelistet. Die Breite des Plateaus beträgt 



\subsection{Bestimmung der Totzeit des Zählrohrs mit der
  Zwei-Quellen-Methode}

\subsection{Bestimmung der Totzeit mit dem Oszilloskop}


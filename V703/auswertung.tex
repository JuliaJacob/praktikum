% This work is licensed under the Creative Commons
% Attribution-NonCommercial 3.0 Unported License.  To view a copy of
% this license, visit http://creativecommons.org/licenses/by-nc/3.0/.

\section{Auswertung}

\subsection{Charakteristik des \name{Geiger}-\name{Müller}-Zählrohrs}

Bei der Aufnahme der Charakteristik des \name{Geiger}-\name{Müller}-
Zählrohrs ist die Probe nach dem achten Meßwert losgeschraubt und damit
in ihrem Abstand vom Zählrohr verstellt worden. Dies ist in
\cref{fig:charakteristik} zu erkennen. Die blaue Kurve zeigt den
eigentlichen Verlauf der Meßpunkte, die rote Kurve zeigt einen Verlauf,
der um 300 Impulse nach unten verschoben ist. Mit den sich so ergebenen
Meßwerten wird eine lineare Ausgleichsrechnung im Bereich des Plateaus
durchgeführt. Die durch Verschiebung korrigierten Werte sind in
\cref{tab:charakteristik} gelistet. Die Breite des Plateaus beträgt
%
\begin{equation}
  d_\text{Plateau} = \SI{270}{\volt}.
\end{equation}
%
Innerhalb dieser Umgebung wird mit den rot markierten Meßwerten eine
lineare Ausgleichsrechnung durchgeführt. Die Steigung des Plateaus ist
dann ein Maß für die Güte des Zählrohrs. Da in diesem Fall die Meßwerte
jedoch durch eine Störung des Versuchsaufbaus beeinträchtigt worden
sind, ist die Steigung nicht aussagekräftig. Der ermittelte Wert beträgt
%
\begin{equation}
  m_\text{Plateau} = \SI{74.5}{\percent\per\volt}.
\end{equation}

\begin{figure}
  \centering
  \includegraphics[width=0.7\textwidth]{charakteristik}
  \caption{Die Anzahl der Impulse, die in der Zeit $\Delta t =
    \SI{120}{\second}$ gemessen worden sind, ist gegen die
    Zählrohrspannung aufgetragen. Die blaue Kurve stellt die tatsächlich
    gemessenen Werte dar. Es ist ab dem 8. Meßpunkt eine Verschiebung
    der Anzahl der Impulse nach oben zu erkennen. In der roten Kurve ist
    versucht worden, diesen systematischen Fehler, der durch das
    versehentliche Verändert des Abstands Präparat--Zählrohr entstanden
    ist, herauszurechnen. In dem Diagramm rechts ist im Plateau-Bereich
    eine lineare Ausgleichsgerade bestimmt worden. Hierfür sind die
    roten Meßpunkte verwendet worden.}
  \label{fig:charakteristik}
\end{figure}

\begin{table}
  \centering
  \begin{tabular}{SS}
    \toprule
    {Zählrohrspannung $U/\si{\volt}$} & {Impulszahl $Z$ in
      \SI{120}{\second}}\\
    \midrule
    320 & 11045\\
    350 & 11450\\
    380 & 11557\\
    410 & 11630\\
    440 & 11723\\
    470 & 11700\\
    500 & 11765\\
    530 & 11659\\
    560 & 11790\\
    590 & 11719\\
    620 & 11770\\
    650 & 11842\\
    680 & 12001\\
    700 & 12546\\
    \bottomrule
  \end{tabular}
  \caption{Hier finden sich die Meßwerte mit den korrigierten
    Impulszahlen. Der systematische Fehler ist auf 300 Impulse geschätzt
    worden und ab dem 8. Meßwert subtrahiert worden. In
    \cref{fig:charakteristik} kann ein Eindruck von der Verschiebung
    gewonnen werden. }
  \label{tab:charakteristik}
\end{table}

\subsection{Bestimmung der Totzeit des Zählrohrs mit der
  Zwei-Quellen-Methode}

Die Meßwerte, die sich aus der Zwei-Quellen-Methode ergeben haben, sind
in \cref{tab:zwei-quellen} zu finden. Mit
\cref{eq:zwei_quellen_naeherung} ergibt sich der Wert der Totzeit zu
\begin{equation}
  T_\text{Zwei Quellen} = \SI{502.8}{\micro\second}.
\end{equation}

\subsection{Bestimmung der Totzeit und Erholungszeit mit dem Oszilloskop}

Mit dem Oszilloskop wird die Totzeit zu
\begin{equation}
  T_\text{Oszilloskop} = \SI{125(13)}{\micro\second}
\end{equation}
abgelesen. Der angegebene Fehler ist der Ablesefehler von
\SI{12.5}{\micro\second}. Es kann hier ebenfalls die Summe aus Totzeit
und Erholungszeit~$E$ des Zählrohrs bestimmt werden. Es gilt:
\begin{equation}
  T_\text{Oszilloskop} + E = \SI{800(13)}{\micro\second}.
\end{equation}
Nach \name{Gauß}scher Fehlerfortpflanzung der fehlerbehafteten Größen
\SI{800(13)}{\micro\second} und $T_\text{Oszilloskop}$ ergibt sich
\begin{equation}
    E = \SI{675(18)}{\micro\second}.
\end{equation}

\subsection{Bestimmung des Abstands zwischen Primär- und
  Nachentladungen}

Der Abstand~$\Delta t$ zwischen Primär- und Nachentladungsimpulsen ist mit dem
Oszilloskop zu
%
\begin{equation}
  \Delta t = \SI{225(13)}{\micro\second}
\end{equation}
%
bestimmt worden.

\subsection{Freigesetzte Ladung pro Teilchen in Abhängigkeit von der
  Zählrohrspannung}

In \cref{fig:freigesetzte_ladung} ist die pro Teilchen freigesetzte
Ladung in Einheiten der Elementarladung gegen die Spannung aufgetragen.
Die Werte sind aus den gemessenen Zählrohrströmen gemäß
\cref{eq:ladungsmenge} bestimmt worden. Aus den gemessenen Impulsraten
und Strömen ergibt sich die freigesetzte Ladung. In
\cref{tab:frei-ladung} sind die Werte noch einmal gegenübergestellt.

\begin{figure}
  \centering
  \includegraphics[width=0.7\textwidth]{freigesetzte_ladung}
    \caption{Die pro Teilchen freigesetzte Ladung ist in Einheiten der
      Elementarladung gegen die Spannung aufgetragen.}
  \label{fig:freigesetzte_ladung}
\end{figure}

\begin{table}
  \centering
  \begin{tabular}{SSS}
    \toprule
    {Spannung $U/\si{\volt}$} &
    {Strom $I/\si{\nano\ampere}$}&
    {Ladung $Q/(10^6\mathrm{e})$} \\
    \midrule
    320 & 200 & 13562(46) \\
    350 & 200 & 13082(47) \\
    380 & 200 & 12961(48) \\
    410 & 300 & 19320(21) \\
    440 & 400 & 25555(12) \\
    470 & 400 & 25606(12) \\
    500 & 500 & 31830(7)  \\
    530 & 600 & 38544(5)  \\
    560 & 600 & 38116(5)  \\
    590 & 600 & 38347(5)  \\
    620 & 800 & 50907(3)  \\
    650 & 800 & 50598(3)  \\
    680 & 1000 & 62409(2) \\
    700 & 1000 & 59698(2) \\
    \bottomrule
  \end{tabular}
  \caption{Die gemessenen Zählrohrströme zu den Zählrohrspannungen. In
    der rechten Spalte sind die gemäß \cref{eq:ladungsmenge} bestimmten
    Ladungen, die pro Teilchen freigesetzt werden, berechnet. Da die
    Ströme mit einem Ablesefehler von \SI{0.02}{\micro\ampere} behaftet
    sind, wird eine \name{Gauß}sche Fehlerfortpflanzung durchgeführt.}
  \label{tab:frei-ladung}
\end{table}

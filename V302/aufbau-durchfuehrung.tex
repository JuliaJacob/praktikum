% This work is licensed under the Creative Commons
% Attribution-NonCommercial 3.0 Unported License. To view a copy of this
% license, visit http://creativecommons.org/licenses/by-nc/3.0/.

\section{Aufbau und Durchführung}

Der Versuchsaufbau besteht im wesentlichen aus drei Teilen: Einem
Oszilloskop, einem Funktionsgenerator und der zu untersuchenden Brücke,
die aus den bereit liegenden Bauteilen zusammengesetzt werden muß. Zum
Aufbau und zur Funktionsweise der Brückenschaltungen ist in
Abschnitt~\ref{sec:theorie} bereits genügend gesagt worden.

\subsection{\name{Wheatstone}-Messbrücke}

In diesem Versuchsteil wird die Brücke einem Funktionsgenerator mit
einer Sinus-Spannung von ca.~\SI{5e3}{\hertz} gespeist. Die
Brückenspannung wird mit einem Oszilloskop beobachtet. Hier ist für den
Widerstand $R_2$ aus Abbildung~\vref{fig:allg-bruecke} ein variabler
Widerstand eingebaut worden, um einen zusätzlichen Stellgrad zu haben,
damit eine kleine Messreihe gemacht durchgeführt werden kann.

\subsection{Kapazitätsmessbrücke}

In der Kapazitätsmessbrücke werden zwei als ideal betrachtete
Kondensatoren und zwei reale Kondensatoren untersucht. Bei den idealen
Kondensatoren werden $C_2$ und das Verhältnis von $R_3$ und $R_4$
variert, um wieder ein zusätzliches Stellglied für eine Messreihe zu
haben. Bei dem realen Kondensator wird aus gleichen Gründen zusätzlich
noch $R_2$ variert. Dann wird analog zur \name{Wheatstone}-Messbrücke
vorgegangen. 

\subsection{Induktivitätsmessbrücke}

Mit der Induktivitätsmessbrücke wird eine Spule vermessen. Dazu werden
$L_2, R_2$ und das Verhältnis $R_3/R_4$ als Stellglieder benutzt. Hier
werden zwei Messungen gemacht.

\subsection{\name{Maxwell}-Brücke}

In der \name{Maxwell}-Brücke werden $R_2, R_3, R_4$ und $C_4$ als
Stellglieder verwendet. Damit werden wieder zwei Messungen durchfeführt.

\subsection{\name{Wien}-\name{Robinson}-Brücke}

Die Untersuchung dieser Brücke unterscheidet sich von den anderen
dadurch, daß sie frequenzabhängig ist. Hier wird als einziges die
Frequenz der Speisespannung verändert und dann die Brückenspannung in
deren Abhängigkeit gemessen.

\FloatBarrier

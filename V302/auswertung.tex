% This work is licensed under the Creative Commons
% Attribution-NonCommercial 3.0 Unported License. To view a copy of this
% license, visit http://creativecommons.org/licenses/by-nc/3.0/.

\section{Auswertung}
\subsection{\name{Wheatstone}-Brücke}
%
Die erste Messreihe wird mit dem unbekannten Widerstand Wert 11 aufgenommen. Anschließend wird eine weiter Messreihe gestartet, diesmal mit dem unbekannten Widerstand Wert 12. 

Die Ergebnisse dieser Messreihen sind in Tabelle \ref{tab:wheatstone} zu finden.
Ebenfalls in dieser Tabelle sind die errechneten Werte für die unbekannten Widerstände zu finden. Diese werden mit Formel \eqref{eq:abgleich-ohmsch}
%
\begin{table}[]
  \centering
  \begin{tabular}{SSS|SSS}
     \toprule
   \multicolumn{3}{c|}{Wert 11} & \multicolumn{3}{c}{Wert 12}\\
    \midrule
{$R_2$/}\si{\ohm}&{$R_3$/}\si{\ohm}&{Wert 11/}\si{\ohm}&{$R_2$/}\si{\ohm}&{$R_3$/}\si{\ohm}&{Wert 12/}\si{\ohm}\\
\midrule
50	&912&518&50&888&396\\
150&768&497&150&721&388\\
250	&665&496&250&612&394\\
350	&586&495&350&531&396\\
450	&524&495&450&468&396\\
550	&474&496&550&418&395\\
650	&432&494&650&378&395\\
750	&397&494&750&345&395\\
850	&362&482&850&312&385\\
950&342&494&950&293&394\\
    \bottomrule
  \end{tabular}
  \caption{Gemessene und errechnete Widerstände mit der \name{Wheatstone}schen Brücke}
  \label{tab:wheatstone}
\end{table}
%
Als Ergebnis der beiden Messreihen ergibt sich also, dass
%
\begin{equation*}
\text{Wert 11} = \SI{496}{\ohm} \pm \SI{3}{\ohm}
\end{equation*}
%
\begin{equation*}
\text{Wert 12} = \SI{393}{\ohm} \pm \SI{1}{\ohm}.
\end{equation*}
%
Die Fehler sind die Standardabweichungen der Mittelwerte, welche aus den Werten der Messreihe errechnet sind.
%
\subsection{Kapazitätsmessbrücke}
%
Es werden Zwei annähernd ideale Kapazitäten Wert 1 und Wert 3, sowie eine reale Kapazität Wert 8 mit zu berücksichtigendem \name{Ohm}schen Widerstand mithilfe einer Kapazitätsmessbrücke untersucht.

In Tabelle \ref{tab:kapazitaet} sind die gemessenen Werte zu finden.
%
\begin{table}[]
  \centering
  \begin{tabular}{SS|SS|SSS}
     \toprule
   \multicolumn{2}{c|}{Wert 1} & \multicolumn{2}{c|}{Wert 3}&\multicolumn{3}{c}{Wert 8}\\
    \midrule
{$C_2$/}\si{\nano\farad}&{$R_3$/}\si{\ohm}&{$C_2$/}\si{\nano\farad}&{$R_3$/}\si{\ohm}&{$C_2$/}\si{\nano\farad}&{$R_2$/}\si{\ohm}&{$R_3$/}\si{\ohm}\\
\midrule
450	&404&992&701&450&373&604\\
597	&475&597&588&597&280&670\\
992	&599&450&516&992&170&770\\
    \bottomrule
  \end{tabular}
  \caption{Gemessene Werte mit der Kapazitätsmessbrücke}
  \label{tab:kapazitaet}
\end{table}
%

Mit Formel \eqref{eq:abgleich-komplex} errechnen sich die unbekannten Kapazitäten und Widerstände zu
%
\begin{equation*}
C_{x1} = \SI{663}{\nano\farad} \pm \SI{1}{\nano\farad} \text{ für Wert 1},
\end{equation*}
%
\begin{equation*}
C_{x3} = \SI{421}{\nano\farad} \pm \SI{1}{\nano\farad} \text{ für Wert 3 },
\end{equation*}
%
\begin{equation*}
C_{x8} = \SI{295}{\nano\farad} \pm \SI{0.5}{\nano\farad} ; R_x8 = \SI{569}{\ohm} \pm \SI{0.2}{\ohm} \text{ für Wert 8}.
\end{equation*}
%
Die Fehler ergeben sich wieder als statistischer Fehler aus den gemessenen Werten.
%
\subsection{Induktivitätsmessbrücke und \name{Maxwell}-Brücke}
In diesem Versuch wird die reale Induktivität Wert 18 analysiert. Die Messwerte, welche mithilfe einer Induktivitätsmessbrücke gewonnen werden, sind in Tabelle \ref{tab:induktivitaet} aufgelistet.

Die Messwerte, welche sich beim Untersuchen von Wert 18 mithilfe einer \name{Maxwell}-Brücke ergeben, sind ebenfalls in dieser Tabelle zu finden.
%
\begin{table}[]
  \centering
  \begin{tabular}{SSS|SSSS}
     \toprule
   \multicolumn{3}{c|}{Induktivitätsbrücke} & \multicolumn{4}{c|}{\name{Maxwell}-Brücke}\\
    \midrule
{$L_2$/}\si{\milli\henry}&{$R_2$/}\si{\ohm}&{$R_3$/}\si{\ohm}&{$R_2$/}\si{\ohm}&{$R_4$/}\si{\ohm}&{$R_3$/}\si{\ohm}&{$C_4$/}\si{\nano\farad}\\
\midrule
14.6&108&772&1000&311&110&450\\
20.1&143&712&322&141&151&992\\
    \bottomrule
  \end{tabular}
  \caption{Aufgenommene Werte mit der Induktivitäts- und der \name{Maxwell}-Brücke}
  \label{tab:induktivitaet}
\end{table}
%

Für die Induktivitätsmessbrücke werden die unbekannten Werte des \name{Ohm}schen Widerstandes und der Induktivität von Wert 18 mithilfe von Formel \eqref{eq:abgleich-ind-bruecke} errechnet und ergeben mit den aufgenommenen Messwerten, dass
%
\begin{equation*}
R_{x18} = \SI{360}{\ohm} \pm \SI{4}{\ohm}
\end{equation*}
%
\begin{equation*}
L_{x18} = \SI{49.6}{\milli\henry} \pm \SI{0.1}{\milli\henry}
\end{equation*}
%
Für die selbe Induktivität ergibt die Rechnung mit Formel \eqref{eq:abgleich-ind-maxwell} für Wert 18, dass
%
\begin{equation*}
R_{x18} = \SI{349}{\ohm} \pm \SI{3}{\ohm}
\end{equation*}
%
\begin{equation*}
L_{x18} = \SI{48.9}{\milli\henry} \pm \SI{0.4}{\milli\henry}
\end{equation*}
%
Die bei diesen Werten angebenen Fehler sind wieder die statistischen Fehler der Messreihe.
%
\subsection{\name{Wien}-\name{Robinson}-Brücke}
Die gemessene Brückenspannung $U_\text{Br}$ in Abhängigkeit von der Frequenz $\nu$ der Sinusspannung der Spannungsquelle ist in Tabelle \ref{tab:wien_robinson} eingetragen. 

Ein halblogarithmitischer Plot des Quotienten aus Brücken- und Eingangsspannung $\frac{U_\text{Br}}{U_\text{S}}$ gegen den Quotienten aus Kreisfrequenz gegen Filterkreisqrequenz $\frac{\omega}{\omega_0}$  ist in Abb. \ref{fig:wien_robinson_plot} zu sehen. Dabei bezeichnet die Filterkreisfrequenz $\omega_0$ die Kreisfrequenz, bei der die Brückenspannung minimal wird.

Im selben Plot ist ebenfalls die von der Theorie durch Formel \eqref{eq:wien_robinson_theo} vorhergesagten Kurve zu finden.

Bei der \name{Wien}-\name{Robinson}-Brücke wird ein Widerstand von R=\SI{1}{\kilo\ohm} und eine Kapazität von C=\SI{294.75}{\nano\farad} verwendet, wodurch sich eine Filterfrequenz von $\nu_0$ = \SI{540}{\hertz} ergibt.
%
\begin{table}[]
  \centering
  \begin{tabular}{SS|SS|SS}
     \toprule
{$\nu$/}\si{\hertz}&{$U_\text{Br}$/}\si{\volt}&{$\nu$/}\si{\hertz}&{$U_\text{Br}$/}\si{\volt}&{$\nu$/}\si{\hertz}&{$U_\text{Br}$/}\si{\volt}\\
\midrule
2&681.3&502&33.12&602&48.12\\
52&675&512&22.5&652&84.38\\
102	&600&522	&14.37&702&118.8\\
152&512.5&532&6.25&802&176.6\\
202&418&542&2.5&1002&271.9\\
252&337&552&9.688&2002&506.3\\
282&300&562&17.5&3002&575\\
302&262&572&25.94&10002&625\\
402	&132.8&582&32.19&20000&593.8\\
452&79.69&592&40.62&30000&556.3\\
    \bottomrule
  \end{tabular}
  \caption{Gemessene Brückenspannungen der \name{Wien}-\name{Robinson}-Brücke}
  \label{tab:wien_robinson}
\end{table}
%
\begin{figure}
\centering
\includegraphics[width=0.8\textwidth]{wien_robinson_plot.pdf}
\caption{Plot der gemessenen und umgerechneten Werte der Brückenspannung gegen die Frequenz}
\label{fig:wien_robinson_plot}
\end{figure}
%

Als letztes wird der Klirrfaktor k des verwendeten Funktionsgenerators bestimmt.
Eine Messung bei der oben genannten Filterfrequenz $\nu_0$ ergibt eine restliche Amplitudenspannung von $U_{Br}$ = \SI{1.4}{\milli\volt}. Es wird angenommen, dass sich diese restliche Spannung nur aus der Spannung der ersten Oberschwingung des eingestellten Sinussignals zusammensetzt. Die Brückenschaltung teilt die Spannung dieser Schwingung, sodass sich der tatsächliche Wert der Amplitudenspannung der ersten Oberschwingung zu $U_2$=\SI{9.4}{\milli\volt} ergibt. 
Der Klirrfaktor k ist in diesem Versuch der Quotient aus $U_2$ und der Amplitudenspannung $U_1$ der Grundschwingung, welche sich zu $U_1$ = \SI{1.99}{\volt} errechnet.
Daraus wird der Klirrfaktor bestimmt. Die Rechnung ergibt, dass
%
\begin{equation*}
k = \SI{4.7e-3}{} = \SI{0.47}{\percent}
\end{equation*}
%
% This work is licensed under the Creative Commons
% Attribution-NonCommercial 3.0 Unported License. To view a copy of this
% license, visit http://creativecommons.org/licenses/by-nc/3.0/.

\section{Auswertung}
\subsection{\name{Wheatstone}-Brücke}
%
Die erste Messreihe wird mit dem unbekannten Widerstand Wert 11
aufgenommen. Anschließend wird eine weiter Messreihe gestartet, diesmal
mit dem unbekannten Widerstand Wert 12.

Die Ergebnisse dieser Messreihen sind in Tabelle~\ref{tab:wheatstone}
zufinden. Ebenfalls in dieser Tabelle sind die errechneten Werte für die
unbekannten Widerstände zu finden. Diese werden mit
Formel~\eqref{eq:abgleich-ohmsch} berechnet.  Als Ergebnis der beiden
Messreihen ergibt sich, dass
%
\begin{align*}
R_\text{x11} &= \SI{496(3)}{\ohm},\\
R_\text{x12} &= \SI{393(1)}{\ohm}.
\end{align*}
%
Die Fehler sind die Standardabweichungen der Mittelwerte, welche aus
den Werten der Messreihe errechnet sind.

\begin{table}
  \centering\footnotesize
  \sisetup{
    table-number-alignment = center,
    table-figures-decimal  = 0
  }
  \begin{tabular}{SSSSSS}
    \toprule
    \multicolumn{3}{c}{Wert 11} & \multicolumn{3}{c}{Wert 12}\\
    {$R_2$/}\si{\ohm} & {$R_3$/}\si{\ohm} & {Wert 11/}\si{\ohm} &
    {$R_2$/}\si{\ohm} & {$R_3$/}\si{\ohm} & {Wert 12/}\si{\ohm}\\
    \cmidrule(r){1-3} \cmidrule(l){4-6}
    50	&912&518&50&888&396\\
    150&768&497&150&721&388\\
    250&665&496&250&612&394\\
    350&586&495&350&531&396\\
    450&524&495&450&468&396\\
    550&474&496&550&418&395\\
    650&432&494&650&378&395\\
    750&397&494&750&345&395\\
    850&362&482&850&312&385\\
    950&342&494&950&293&394\\
    \bottomrule
  \end{tabular}
  \caption{Hier sind die gemessenen und errechneten Widerstände, die
    bei der Untersuchung der \name{Wheatstone}schen Brücke verwendet
    worden sind, gelistet. Dabei sind jeweils die mit Wert 11 und Wert
    12 markierten Widerstände benutzt worden.}
  \label{tab:wheatstone}
\end{table}

\subsection{Kapazitätsmessbrücke}
%
Es werden zwei annähernd ideale Kapazitäten Wert 1 und Wert 3, sowie
eine reale Kapazität Wert 8 mit zu berücksichtigendem \name{Ohm}schen
Widerstand mithilfe einer Kapazitätsmessbrücke untersucht.

In Tabelle~\ref{tab:kapazitaet} sind die gemessenen Werte zu finden.
Mit Formel~\eqref{eq:abgleich-komplex} errechnen sich die unbekannten
Kapazitäten und Widerstände zu
%
\begin{align*}
C_\text{x1} &= \SI{663(1)}{\nano\farad} \text{ für Wert 1},\\
C_\text{x3} &= \SI{421(1)}{\nano\farad} \text{ für Wert 3},\\
C_\text{x8} &= \SI{295.0(5)}{\nano\farad} \text{ für Wert 8},\\
R_\text{x8} &= \SI{569.0(2)}{\ohm} \text{ für Wert 8}.
\end{align*}
%
Die Fehler ergeben sich wieder als statistischer Fehler aus den
gemessenen Werten.
%
\begin{table}
  \centering\footnotesize
  \sisetup{
    table-number-alignment = center,
    table-figures-decimal  = 0
  }
  \begin{tabular}{SSSSSSS}
     \toprule
     \multicolumn{2}{c}{Wert 1} & \multicolumn{2}{c}{Wert 3} &
     \multicolumn{3}{c}{Wert 8}\\
     {$C_2$/}\si{\nano\farad}&{$R_3$/}\si{\ohm}&{$C_2$/}\si{\nano\farad}&
     {$R_3$/}\si{\ohm}&{$C_2$/}\si{\nano\farad}&{$R_2$/}\si{\ohm}&
     {$R_3$/}\si{\ohm}\\
     \cmidrule(r){1-2} \cmidrule(lr){3-4} \cmidrule(l){5-7}
     450&404&992&701&450&373&604\\
     597&475&597&588&597&280&670\\
     992&599&450&516&992&170&770\\
    \bottomrule
  \end{tabular}
  \caption{Gemessene Werte mit der Kapazitätsmessbrücke}
  \label{tab:kapazitaet}
\end{table}

\subsection{Induktivitätsmessbrücke und \name{Maxwell}-Brücke}

In diesem Versuch wird die reale Induktivität Wert 18 analysiert. Die
Messwerte, welche mithilfe einer Induktivitätsmessbrücke gewonnen
werden, sind in Tabelle~\ref{tab:induktivitaet} aufgelistet.

Die Messwerte, welche sich beim Untersuchen von Wert 18 mithilfe einer
\name{Maxwell}-Brücke ergeben, sind ebenfalls in dieser Tabelle zu
finden.  Für die Induktivitätsmessbrücke werden die unbekannten Werte
des \name{Ohm}schen Widerstandes und der Induktivität von Wert 18
mithilfe von Formel~\eqref{eq:abgleich-ind-bruecke} errechnet und
ergeben mit den aufgenommenen Messwerten, dass
%
\begin{align*}
R_\text{x18} &= \SI{360(4)}{\ohm},\\
L_\text{x18} &= \SI{49.6(1)}{\milli\henry}.
\end{align*}
%
Für die selbe Induktivität ergibt die Rechnung mit
Formel~\eqref{eq:abgleich-ind-maxwell} für Wert 18, dass
%
\begin{align*}
R_{x18} &= \SI{349}{\ohm} \pm \SI{3}{\ohm},\\
L_{x18} &= \SI{48.9}{\milli\henry} \pm \SI{0.4}{\milli\henry}.
\end{align*}
%
Die bei diesen Werten angebenen Fehler sind wieder die statistischen
Fehler der Messreihe.
%

\begin{table}
  \centering\footnotesize
  \sisetup{
    table-number-alignment = center,
    table-figures-decimal  = 1,
    table-figures-integer  = 4
    }
  \begin{tabular}{SSSSSSS}
     \toprule
     \multicolumn{3}{c}{Induktivitätsbrücke} &
     \multicolumn{4}{c}{\name{Maxwell}-Brücke}\\
     {$L_2$/}\si{\milli\henry}&{$R_2$/}\si{\ohm}&{$R_3$/}\si{\ohm}&
     {$R_2$/}\si{\ohm}&{$R_4$/}\si{\ohm}&{$R_3$/}\si{\ohm}&
     {$C_4$/}\si{\nano\farad}\\
     \cmidrule(r){1-3} \cmidrule(l){4-7}
     14.6&108&772&1000&311&110&450\\
     20.1&143&712&322&141&151&992\\
    \bottomrule
  \end{tabular}
  \caption{Aufgenommene Werte mit der Induktivitäts- und der
    \name{Maxwell}-Brücke}
  \label{tab:induktivitaet}
\end{table}

\subsection{\name{Wien}-\name{Robinson}-Brücke}
Die gemessene Brückenspannung $U_\text{Br}$ in Abhängigkeit von der
Frequenz $\nu$ der Sinusspannung der Spannungsquelle ist in
Tabelle~\ref{tab:wien_robinson} eingetragen.

Ein halblogarithmitischer Plot des Quotienten aus Brücken- und
Eingangsspannung $\frac{U_\text{Br}}{U_\text{S}}$ gegen den Quotienten
aus Kreisfrequenz gegen Filterkreisqrequenz $\frac{\omega}{\omega_0}$
ist in Abb.~\ref{fig:wien_robinson_plot} zu sehen. Dabei bezeichnet die
Filterkreisfrequenz $\omega_0$ die Kreisfrequenz, bei der die
Brückenspannung minimal wird. Im selben Plot ist ebenfalls die von der
Theorie durch Formel~\eqref{eq:wien_robinson_theo} vorhergesagten Kurve
zu finden.

Bei der \name{Wien}-\name{Robinson}-Brücke wird ein Widerstand von
$R=\SI{1}{\kilo\ohm}$ und eine Kapazität von
$C=\SI{294.75}{\nano\farad}$
verwendet, wodurch sich eine Filterfrequenz von $\nu_0$ =
\SI{540}{\hertz} ergibt.

Als letztes wird der Klirrfaktor k des verwendeten Funktionsgenerators
bestimmt.  Eine Messung bei der oben genannten Filterfrequenz $\nu_0$
ergibt eine restliche Amplitudenspannung von $U_{Br}$ =
\SI{1.4}{\milli\volt}. Es wird angenommen, dass sich diese restliche
Spannung nur aus der Spannungder ersten Oberschwingung des eingestellten
Sinussignals zusammensetzt.  Die Brückenschaltung teilt die Spannung
dieser Schwingung, sodass sich der tatsächliche Wert der
Amplitudenspannung der ersten Oberschwingung zu
$U_2$=\SI{9.4}{\milli\volt} ergibt.  Der Klirrfaktor k ist in diesem
Versuch der Quotient aus $U_2$ und der Amplitudenspannung $U_1$ der
Grundschwingung, welche sich zu $U_1$ = \SI{1.99}{\volt} errechnet.
Daraus wird der Klirrfaktor bestimmt. Die Rechnung ergibt, dass
%
\begin{equation*}
k = \num{4.7e-3} = \SI{0.47}{\percent}.
\end{equation*}
%

\begin{table}
  \centering\footnotesize
  \sisetup{
    table-number-alignment = center,
    table-figures-decimal  = 3,
    table-figures-integer  = 5
  }
  \begin{tabular}{SSSSSS}
    \toprule
    {$\nu$/}\si{\hertz} & {$U_\text{Br}$/}\si{\milli\volt}&
    {$\nu$/}\si{\hertz} & {$U_\text{Br}$/}\si{\milli\volt}&
    {$\nu$/}\si{\hertz} & {$U_\text{Br}$/}\si{\milli\volt}\\
    \cmidrule(r){1-2} \cmidrule(rl){3-4} \cmidrule(l){5-6}
    2&681.3&502&33.12&602&48.12\\
    52&675&512&22.5&652&84.38\\
    102	&600&522	&14.37&702&118.8\\
    152&512.5&532&6.25&802&176.6\\
    202&418&542&2.5&1002&271.9\\
    252&337&552&9.688&2002&506.3\\
    282&300&562&17.5&3002&575\\
    302&262&572&25.94&10002&625\\
    402	&132.8&582&32.19&20000&593.8\\
    452&79.69&592&40.62&30000&556.3\\
    \bottomrule
  \end{tabular}
  \caption{Gemessene Brückenspannungen der
    \name{Wien}-\name{Robinson}-Brücke}
  \label{tab:wien_robinson}
\end{table}

\begin{figure}
  \centering
  \includegraphics[width=0.75\textwidth]{wien_robinson_plot.pdf}
  \caption{Plot der gemessenen und umgerechneten Werte der
    Brückenspannunggegen die Frequenz. Die $x$-Achse ist logarithmisch
    skaliert. Der Zusammenbruch der Brückenspannung, wenn
    $\omega\to\omega_0$ geht, ist deutlich zu erkennen.}
  \label{fig:wien_robinson_plot}
\end{figure}

\FloatBarrier

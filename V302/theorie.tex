\section{Theorie}
\label{sec:theorie}

\subsection{Die allgemeine Brückenschaltung}
\label{sec:allg-bruecke}

Eine Brückenschaltung hat im allgemeinen den in
Abbildung~\ref{fig:allg-bruecke} dargestellten Aufbau.  Die Spannung
zwischen den Punkten $A$ und $B$ ergibt sich mithilfe der Maschen- und
Knotenregeln von \name{Kirchhoff}. Die Knotenregel besagt
\begin{gather}
  I_1 = I_2\\
  \intertext{und}
  I_3 = I_4.
\end{gather}
Die Maschenregel ergibt
\begin{gather}
  U = -R_1 I_1 + R_3 I_3\\
  \intertext{und}
 -U = -R_2 I_2 + R_4 I_4.
\end{gather}
Wenn $U_\text{S}$ die Speisespannung bezeichnet, erhält man daraus
\begin{equation}
  U = \frac{R_2 R_3 - R_1 R_4}{(R_3 + R_4)(R_1 + R_2)} U_\text{S}.
\end{equation}
Die Brückenspannung verschwindet daher bei beliebiger Brückenspannung, falls
\begin{equation}
  \label{eq:abgleich}
  R_2 R_3 = R_1 R_4.
\end{equation}

Sind in der Brückenschaltung auch Kapazitäten und Induktivitäten
verbaut, so kann man die obigen Widerstände als komplexe Zahlen $Z =
X + iY$ auffassen und erhält genauso
\begin{equation}
  \label{eq:abgleich-komplex}
  Z_2 Z_3 = Z_1 Z_4.
\end{equation}
Weil komplexe Zahlen Paare $(a, b)\in\mathbb{R}^2$ reeller Zahlen
sind, müssen für die Gleichheit also die zwei Bedingungen
\begin{gather}
  X_1 X_4 - Y_1 Y_4 = X_2 X_3 - Y_2 Y_3\\
  \intertext{und} 
  X_1 Y_4 + X_4 Y_1 = X_2 Y_3 + X_3 Y_2
\end{gather}
erfüllt sein.  Daher werden bei einer Wechselstrombrücke zwei
voneinander unabhängige Veränderliche eingebaut sein.

\subsection{\name{Wheatstone}-Widerstandsmeßbrücke}
In der \name{Wheatstone}schen Brückenschaltung sind nur \name{Ohm}sche
Widerstände verbaut, weswegen die Schaltung sowohl mit Gleich- als
auch mit Wechselstrom betrieben werden kann. Dieser Brücke kann zur
Bestimmung eines unbekannten Widerstandes verwendet werden. Wenn zum
Beispiel $R_1$ in Abbildung~\ref{fig:allg-bruecke} unbekannt ist,
erhält man aus der Abgleichbedingung~\eqref{eq:abgleich}
\begin{equation}
  \label{eq:abgleich-ohmsch}
  R_1 = R_2 \frac{R_3}{R_4}.
\end{equation}
Da es bei den Widerständen $R_3$ und $R_4$ nur auf das Verhältnis
ankommt, können diese beiden als Potentiometer eingebaut werden.

\subsection{Kapazitätsmeßbrücke}
Ein realer Kondensator wird durch eine ideale Kapazität und einen
\name{Ohm}schen Widerstand in Reihe modelliert, da es immer zu
sogenannten dielektrischen Verlusten kommt, bei denen ein Teil der
gespeicherten elektrischen Energie in Wärme umgewandelt wird. Für den
komplexen Wechselstromwiderstand folgt daher
\begin{equation}
  Z = R - \frac{i}{\omega C}.
\end{equation}
Wie im Abschnitt~\ref{sec:allg-bruecke} über Wechselstromwiderstände
gesagt worden ist, benötigt man in der Schaltung eine weitere
Veränderliche, um den zusätzlichen Freiheitsgrad zu
berücksichtigen. Dazu wird der Widerstand $R_2$ in
Abbildung~\ref{fig:kap-brücke} veränderlich sein. Die
Abgleichbedingung~\eqref{eq:abgleich-komplex} liefern dann
\begin{gather}
  \label{eq:abgleich-kap}
  R_x = R_2 \frac{R_3}{R_4}\\
  \intertext{und}
  C_x = C_2 \frac{R_4}{R_3}.
\end{gather}

\subsection{Induktivitätsmeßbrücke}
Wie ein realer Kondensator durch eine ideale Kapazität und einen
\name{Ohm}schen Widerstand modelliert wird, kann eine Spule durch eine
ideale Induktivität und einen \name{Ohm}schen Widerstand dargestellt
werden (siehe Abbildung~\ref{fig:ind-bruecke}, da auch sie einen Teil
der magnetischen Energie in Wärme umwandelt. Also ist
\begin{equation}
  Z = R + i \omega L
\end{equation}
der zugehörige komplexe Wechselstromwiderstand. Ähnlich wie bei der
Kapazitätsmeßbrücke ergeben sich die Abgleichbedingungen
\begin{gather}
R_x = R_2 \frac{R_3}{R_4},\\
L_x = L_2 \frac{R_3}{R_4}
\end{gather}

\subsection{\name{Maxwell}-Brücke}
Abbildung~\ref{fig:maxwell-bruecke} zeigt die sogenannte
\name{Maxwell}-Brücke. Die Widerstände $R_3$ und $R_4$ sind regelbar.
Die komplexen Wechselstromwiderstände sind also
\begin{gather}
Z_1 = R_x + i \omega L_x\\
\frac{1}{Z_4} = \frac{1}{R_4} + i \omega C_4.
\end{gather}
Mit der Abgleichbedingung~\eqref{eq:abgleich-komplex} kann dann
\begin{gather}
R_x = R_2 \frac{R_3}{R_4},\\
L_x = R_2 R_3 C_4
\end{gather}
gefunden werden.

\subsection{\name{Wien}-\name{Robinson}-Brücke}

Bis jetzt sind die beschriebenen Brücken zumindest der Theorie nach
frequenzunabhängig. Die \name{Wien}-\name{Robinson}-Brücke hingegen
ist eine frequenzabhängige Brücke. Die Schaltung ist in
Abbildung~\ref{fig:wien-robinson-bruecke} zu sehen. Das entscheidende
an dieser Brücke erkennt man, wenn das Verhältnis von Brücken- und
Speisespannung betrachtet wird.
\begin{equation}
\left|\frac{U_\text{Br}}{U_\text{S}}\right|^2 = \frac{(1 - (\omega R
  C)^2)^2}{9(1 - (\omega R C)^2)^2 + 81 (\omega R C)^2}.
\end{equation}
Hieran kann gesehen werden, daß die Brückenspannung verschwindet,
falls 
\begin{equation}
\omega = \omega_0 := \frac{1}{RC}, 
\end{equation}
Mit $\Omega := \omega/\omega_0$ kann das Spannungsverhältnis
umgeschrieben werden, nämlich
\begin{equation}
\left|\frac{U_\text{Br}}{U_\text{S}}\right|^2 = \frac{1}{9} \frac{(1 -
  \Omega^2)^2}{(1 - \Omega^2)^2 + 9 \Omega^2},
\end{equation}
und man erkennt, daß die \name{Wien}-\name{Robinson}-Brücke aus einem
kontinuierlichen Frequenzspektrum die Frequenz~$\omega_0$
herausfiltert.


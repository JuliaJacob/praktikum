
\section{Theorie}

\subsection{Die allgemeine Brückenschaltung}

Eine Brückenschaltung hat im allgemeinen den in
Abbildung~\ref{fig:allg-bruecke} dargestellten Aufbau.  Die Spannung
zwischen den Punkten $A$ und $B$ ergibt sich mithilfe der Maschen- und
Knotenregeln von \name{Kirchhoff}. Die Knotenregel besagt
\begin{gather}
  I_1 = I_2\\
  I_3 = I_4.
\end{gather}
Die Maschenregel ergibt
\begin{gather}
  U = -R_1 I_1 + R_3 I_3\\
 -U = -R_2 I_2 + R_4 I_4.
\end{gather}
Wenn $U_\text{S}$ die Speisespannung bezeichnet, erhält man daraus:
\begin{equation}
  U = \frac{R_2 R_3 - R_1 R_4}{(R_3 + R_4)(R_1 + R_2)} U_\text{S}.
\end{equation}
Man kann also durch Variation dreier Widerstände die Brückenspannung auf
Null regeln und dann gilt
\begin{equation}
  R_2 R_3 = R_1 R_4.
\end{equation}


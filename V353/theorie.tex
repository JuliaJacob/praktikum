% This work is licensed under the Creative Commons
% Attribution-NonCommercial 3.0 Unported License. To view a copy of this
% license, visit http://creativecommons.org/licenses/by-nc/3.0/.

\section{Theorie}

Den Vorgang, der sich abspielt, wenn man ein System aus seinem
Ausgangszustand auslenkt, nennt man, sofern er sich ohne Oszillationen
vollzieht, Relaxation.

\subsection{Auf- und Entladevorgang am Kondensator}
Zunächst wird der Entladevorgang betrachtet. Aus der
Abbildung~\ref{fig:rc-glied} kann man mithilfe der \name{Kirchhoff}schen
Regeln diese Differentialgleichung ableiten:
%
\begin{equation}
  \label{eq:rc-dgl}
  RC\cdot Q'(t) + Q(t) = 0 \text{.}
\end{equation}
%
Die Lösung zum Anfangswertproblem $Q(0) = Q_0$ ist gegeben mit
%
\begin{equation}
  \label{eq:sol-rc-dgl}
  Q(t) = Q_0 \exp(-t/RC) \text{.}
\end{equation}
%
Hier kann man nun das Relaxationsverhalten beim Entladen erkennen,
nämlich daß der Kondensator nach unendlich langer Zeit wieder entladen
ist: $\lim_{t\to\infty} Q(t) = 0$.

Zum Aufladen wird eine Spannungsquelle benötigt, so daß sich die
Abbildung~\ref{fig:rcu-glied} ergibt. Die zugehörige
Differentialgleichung lautet:
%
\begin{equation}
  \label{eq:rcu-dgl}
  RC\cdot Q'(t) + Q(t) = U \text{.}
\end{equation}
%
Die Lösung zum Anfangswertproblem $Q(0) = 0$ führt auf die Funktion $Q$
mit:
%
\begin{equation}
  \label{eq:sol-rcu-dgl}
  Q(t) = UC \Big(1-\exp(-t/RC)\Big) \text{.}
\end{equation}
%
Auch bei dieser Lösung wird nach unendlich langer Zeit der gewünschte
Endzustand erreicht: $\lim_{t\to\infty} Q(t) = UC$. 

Die Zahl $RC$ heißt Zeitkonstante und gibt an, wie schnell die Funktion
gegen ihren entsprechenden Grenzwert, d.\,h. gegen ihren Endzustand
konvergiert.

\begin{figure}
  \centering
  \includegraphics[width=0.3\textwidth]{rc-glied}
  \caption{Schaltung eines RC-Gliedes ohne äußere Spannungsquelle}
  \label{fig:rc-glied}
\end{figure}

\begin{figure}
  \centering
  \includegraphics[width=0.3\textwidth]{rcu-glied}
  \caption{Schaltung eines RC-Gliedes mit äußerer Spannungsquelle}
  \label{fig:rcu-glied}
\end{figure}

\subsection{Relaxationsverhalten bei äußerer Anregung}

Die Spannungsquelle in Abbildung~\ref{fig:rcu-glied} ist bisher konstant
gewesen. Dies soll sich nun ändern. Statt dessen wird ein
Wechselspannung der Kreisfrequenz $\omega$ angelegt:
%
\begin{equation}
  \label{eq:u-ac}
  U(t) = U_0\cos(\omega t) \text{.}
\end{equation}
%
Dies sorgt nun dafür, daß das RC-Glied zumindest bei niedrigen
Frequenzen dem aufgeprägten Signal folgt, daß also die Spannung $U_c$ am
Kondensator zu jedem Zeitzunkt ungefähr gleich der Spannung $U(t)$ der
Spannungsquelle ist. Werden aber die Frequenzen erhöht, so hinkt der
Kondensator mit seiner Auf- und Entladung der Spannungsquelle immer
deutlicher hinterher. Dies führt zu einer Phasenverschiebung $\varphi$
zwischen beiden Spannungen.

Es wird nun die Abhängigkeit der Phase und Amplitude der
Kondensatorspannung von der Frequenz untersucht. Dabei ergeben sich die
folgenden Formeln:
%
\begin{equation}
  \label{eq:phase-frequenz}
  \varphi(\omega) = \arctan(-\omega RC)
\end{equation}
%
\begin{equation}
  \label{eq:amplitude-frequenz}
  A(\omega) = \frac{U_0}{\sqrt{1+(\omega RC)^2}}
\end{equation}


\subsection{Funktion des RC-Gliedes als Integrator}

Unter der Voraussetzung, daß $\omega\ll 1/RC$ ist, funktioniert das
RC-Glied als Integrator. Dies ergibt sich aus dem Zusammenhang:
%
\begin{equation}
  \label{eq:dgl-umformung}
  U(t) = U_R(t) + U_C(t) = R I(t) + U_c(t) = RC U_C'(t) + U_C(t)
\end{equation}
%
Mit der Voraussetzung folgt, daß $|U_C| \ll |U_R|$ und $|U_C| \ll |U|$
und daraus kann \eqref{eq:dgl-umformung} so geschrieben werden:
%
\begin{equation}
  U(t) = RC U_C'(t) \implies U_C(t) = \frac{1}{RC}\int_0^t U(x) \d x
\end{equation}
